\documentclass[11pt]{article}
\usepackage{setspace}
\usepackage{graphicx}
\usepackage{subfigure}
\usepackage{lscape}
\usepackage{flafter}  % Don't place floats before their definition
\usepackage{bm}  % Define \bm{} to use bold math fonts
\usepackage{amsmath}
\usepackage{amsfonts}
\usepackage{amssymb}
\usepackage{MnSymbol}
\usepackage{url}
\usepackage{natbib}
%\usepackage{fullpage}
\bibliographystyle{cbe}
\citestyle{aa}
%\usepackage{algorithmic}
%\usepackage[vlined,algochapter,ruled]{algorithm2e}
\usepackage[vlined,ruled]{algorithm2e}
\SetKwComment{Comment}{$\triangleright\ $}{}

%ROOT PRIORS?
%Add Giribet and Wheeler arth, ratchet cite, Arango sea spiders

\title{Algorithmic Descriptions and Pseudo-Code for Phylogenetic Component Graphs}
\author{Ward C. Wheeler\\
		%Richard Gilder Graduate School,\\
		Division of Invertebrate Zoology,\\
		American Museum of Natural History,\\
		Central Park West @ 79th Street,\\
		New York, NY 10024-5192,\\
		USA,\\
		wheeler@amnh.org}
	%\date{}
\begin{document}

\maketitle
\begin{abstract}
	Algorithmic Descriptions and Pseudo-Code for Phylogenetic Component Graphs (PCG) code base.
\end{abstract}
%\newpage
%\tableofcontents
%\newpage

%\doublespacing
\section{Introduction}
This document contains the descriptions and pseudo-code for core  Phylogenetic Component Graphs (PCG)
algorithms.  This is designed to serve both as documentation for the code and a development tool.  We have tried to keep
the style as consistent as possible,  Functional-type statements (\textit{e.g.} map and fold) and structures (\textit{e.g.} lists, tuples)
are employed.

\section{Labelling of a single graph}
One of the fundamental procedures is in the labelling of individual graphs,  This is the determination of various 
aspects (including preliminary, final, and single assignments) of vertices and edges.   

The basic PCG graph is a forest of directed acyclic graphs (DAG).  These DAGs may be trees in the sense of having
 vertices with (indegree, outdegree): (0, 1) the unique root, (1,0) leaves, and (1,2) ``internal'' vertices.; or
 networks adding (2,1) vertices as ``network'' vertices.  Individual components may have more than one 
 root vertex.   All trees or networks are \textit{components} of a forest.  A forest
 is a list of components,
 $F = [N]$, where $N = (V, E)$ (conventionally $V$ is a set of vertices and $E$ a set of edges).  One or more vertex in each component is a root.
 Vertices with indegree and outdegree 1 may be maintained for computational convenience, but would normally be contracted.
 
 Following \cite{moretetal2004}, the above would be described as a ``model'' phylogenetic network. This has two additional constraints.
 First, that all edges must begin or end in a tree vertex (no network vertex to network vertex edges).  
 Second,  the parents of a network vertex must, at least potentially, be 
 contemporaneous. [Insert path conditions here]  
 
 A less restrictive form of network (``reconstructable'') is defined by \cite{moretetal2004} with vertices (indegree, outdegree): 
 (0, $\ge 1$) the unique root, (1, 0) leaves, and (1, $\ge$1) ``internal'' vertices, and 
 networks adding ($\ge 2$, 1) vertices as ``network'' vertices.  For purposes of searching,
 outdegree of internal nodes is restricted to at most 2. Since edges may have zero length, networks may be output 
 with vertices of outdegree $> 2$.
 
 \section{Optimization of a single graph}
 Optimization is a function that takes a graph ($G$), here a forest $F$, and data ($D$) as inputs and returns a numerical value.  
 This value can be referred to as its score ($S$)
 	\begin{eqnarray*}
 		f & : &  \{\mathcal{F}, D\} \rightarrow \mathbb{R}\\
		g & : & \{  \mathbb{R},  \mathbb{R} \} \rightarrow \mathbb{R}
	 \end{eqnarray*}
 
 \section{Forest Optimization/Labelling}
 The optimization is simply the optimization of each constituent component and their optimization values 
 combined through $f$ (summed for parisomy, if likelihood or such then add logs).  Both local (vertex  specific) and subtree costs (local cost of a vertex plus
 the subtree costs of its descendants) are stored.  The overall component cost is stored at the traversal root.
     \begin{eqnarray*}
     	\mathcal{F} & = & [C]\\
	\mathcal{F}_{labelled} & = & \text{ map } labelNetwork \text{ } [C]\\
	\left[ S_{\mathcal{F}} \right] &=& \text{ map } f \text{ } \mathcal{F}_{labelled}\\
	\mathcal{F}_{optValue} &=& \text{ fold } g \text{ } \left[ S_{\mathcal{F}} \right] 
    \end{eqnarray*}
 
 Define a 4-tuple $L = (a, b, c, d)$ with $a$ a bitvector of the length of leaves, $b$ a tree representation, $c$ a 
 character assignment (of appropriate character type), and $d$ a real (likely double). Each vertex will have a list of these
 tuples, $\left[L\right]$, one element for each network node resolution, created during a postorder traversal of a component.
 
 Define a function $h$ that takes as input two labellings (state sets of a character) and character metadata, returning 
 a labelling and a real denoting the cost of creating the returned labelling.
 
     \begin{eqnarray*}
     	h : \left(l , l \right) \rightarrow \left( l ,  \mathbb{R} \right)
    \end{eqnarray*}

Define a tree representation (e.g. Newick) and a function to ``join'' the tree representations of two descendant vertices
into the representation of the subtree for that vertex.  For a newick representation this would be the string operation
joining representations $A$ and $B$ into $\left( A, B \right)$.

 A full optimization of a forest results in a fully ``decorated'' graph (all vertex and edge information present).  This normally
 proceeds via a map of the optimization operations over the components of a forest.  The forest components are optimized in turn
 by a series of traversals over the vertex and edge sets.
 
% However, multiple components may share vertices.  A vertex with indegree $\ge 2$ where there are $\ge 2$ edges
% that are members of different components are referred to as multi-network edges and the vertices multi-network
% vertices.  The component membership of an edge 
% is determined by tracing the edge back (reverse direction) to its root.  If the roots of two edges are different--they are 
% multi-network edges.  
% Edges can be
% both multi-network and network.  So, if we 
% require ``soft-wired'' forests, the multi-network vertices need to be resolved and the forest into a
% into a list of ``simple'' forests with independent components ($\vline V_{ms} \vline = 0; V_i \cap V_j = \emptyset, \forall V_i, V_j \in F$).
% 	\begin{eqnarray*}
%		\lbrack F \rbrack &=& resolveForest \text{ } \mathcal{F}\\
%		\lbrack F_{score} \rbrack &=& \text{map } simpleForestFullOptimization \text{ } \lbrack F \rbrack \\
%		\mathcal{F}_{score} &=& \min \text{ } \lbrack F_{score}  \rbrack
%	 \end{eqnarray*}
%Resolution into simple forests proceeds by choosing a single indegree edge for each of the multi-network vertices and 
%recursively generating all possible combinations.  The result is a list of simple forests.  
%
%	\begin{algorithm}
%		\caption{resolveForest}
%		\label{alg:resolveForest}
%		\SetAlgoLined
%		\KwData{Input  forest $\mathcal{F}$ (list of DAGs) and data $D$}
%		\KwResult{List of simple forests $[F]$ (no multi-network vertices)}
%		\Comment{recursively choose one indegree edge for each multi-network vertex} 
%		$\lbrack F^{simple} \rbrack \leftarrow $all combinations of multi-network edge choices\\
%		\Return {$\lbrack F^{simple} \rbrack$}
%	\end{algorithm}

	\begin{algorithm}
		\caption{forestFullOptimization}
		\label{alg:orestFullOpt}
		\SetAlgoLined
		\KwData{Input forest $F$ (list of DAGs including components, networks, and trees) and data $D$}
		\KwResult{Optimality value of Forest and and list of optimality value and fully labelled (vertex, edge features etc) forest components}
		\Comment{map optimization over list of components} 
		$optimizedComponentList =$ map (componentFullOptimization $D$) $F$
		
		\Comment{extract and combine optimality values of components} 
		$optimalityValue =$ fold $g$ (map first $optimizedComponentList$)
		
		\Return {(optimalityValue, optimizedComponentList)}
	\end{algorithm}
	
	
	\begin{algorithm}
		\caption{componentFullOptimization}
		\label{alg:compFullOpt}
		\SetAlgoLined
		\KwData{Input component $C$ (DAGs multi-rooted component,  network, or tree) and data $D$}
		\KwResult{Decorated, fully labelled (vertex, edge features etc) component with optimality value}
		\Comment{Post-order traversal, beginning with root as fold right} 
		$preliminaryComponent =$ fold (makePreliminaryVertex $D$) $C$
		
		\Comment{Pre-order traversal beginning at root} 
		$finalComponent =$ fold (makeFinalVertex $D$) $preliminaryComponent$
		
		\Return {(optimalityValue, optimizedComponent)}
	\end{algorithm}

\begin{algorithm}
		\caption{makePreliminaryVertex}
		\label{alg:prelimComp}
		\SetAlgoLined
		\KwData{Input vertex $V$, descendent lists of left $L$ and right $R$ descendants of $V$ (if not a leaf), and data $D$}
		\KwResult{List of label 4-tuple $\left[ \left(a, b, c, d \right) \right]$ as defined above for $V$}
		\Comment{If leaf take observed states and subtree is just name, bitvector has `1' for that leaf and `0' for all others} 
		\If {$V$ is leaf } {$labelList = \left[ \left( V_{bv},  V_{subtreeRep},\text{observed state}, 0 \right) \right] $} 			
		\Comment{Otherwise do all combinations of two lists\\Check if previous in list $L$ and $R$ are same as current--if so then just return previous label}
		\Else {$labelList =$ map2 joinTwoLabelings $L$ $R$}
		\Return {labelList}
	\end{algorithm}

\begin{algorithm}
	\caption{joinTwoLabelings}
	\label{alg:joinTwo}
	\SetAlgoLined
		\KwData{Input two $L$ and $R$ 4-tuple labelings}
		\KwResult{4-tuple labelling, could be null `()'}
		\Comment{If bitvectors have non-empty intersection then incompatible resolution and return null} 
		\If {$L_{bv} \text{ AND } R_{bv} <> 0$} {\Return ()} 			
		\Comment{Otherwise combine elements} 
		\Else {$\Return \left( L_{bv} \text{ OR } R_{bv}, \text {join } L_{subtreeRep} R_{subtreeRep}, h \text{ } \left( L_{assignment} R_{assignment} \right) \right)$}
	\end{algorithm}
%psuedocode
 %resolve soft-wired forest into simple forests.
 %keep track of simple and reticulate forests for optimization purposes.
 
 \section{Forest Rearrangement}
 There are two operations in forest rearrangements.  The first is the operation which splits network components into
 multiple components and the second joins separate components into network components.
 
 \subsection{Break}
 Each tree edge is deleted in turn.  There are two possible outcomes.  The resulting two components
 may or may not be connected (have vertices in common).  The neighborhood can be generated and evaluated
 for overall minimum or breaks evaluated seriatim.   In either case, the new forest elements 
 are diagnosed and compared to the current best.  If better cost, then kept, else discarded.  
 
	\begin{algorithm}
		\caption{forestBreak}
		\label{alg:forestBreak}
		\SetAlgoLined
		\KwData{Input forest $F$ (list of DAGs including networks and trees) and data $D$}
		\KwResult{Forest break neighborhood as list}
		\Comment{Break each forest component at each tree edge.  Head of deleted edge becomes new 
		root, tail becomes (1,1) vertex} 
		$\lbrack F' \rbrack \leftarrow $all combinations of tree edge deletions\\
		\Return {$\lbrack F' \rbrack$}
	\end{algorithm}

 \subsection{Join}
 Pairs of components are joined by creating a new edge between each pair of edges in the input networks,
 subject to the time consistency constraint of network edges.  This basically proceeds as SPR/TBR joins
 with added complexity of network edge issues (ie. time consistency. 

	\begin{algorithm}
		\caption{forestJoin}
		\label{alg:forestJoin}
		\SetAlgoLined
		\KwData{Input forest $F$ (list of DAGs including networks and trees) and data $D$}
		\KwResult{Forest join neighborhood as list}
		\Comment{For each pair of forest components, add edge between each pair of edges (TBR), or edges and rot vertex (SPR)} 
		$\lbrack F' \rbrack \leftarrow $all combinations of tree edge joins\\
		\Return {$\lbrack F' \rbrack$}
	\end{algorithm}

 \section{Network Rearrangement}
 As with forest rearrangement, network edges can be added or subtracted, and underlying (displayed) trees 
 rearranged vis NNI, SPR, TBR etc.
 
 \subsection{Network edge removal}
 The optimization of networks identifies 
 unused edges, so these can be removed during optimization.  Standardly, networks with unused edges are attached an
 infinite cost.  Another option would be to output a new network with the offending edge deleted.  Remaining 
 network edges can be removed in turn.
 
 \subsection{Network edge addition}
 Subject to edge time constraints and that each edge must have at least one tree vertex, new edges are added
 between each pair of edges (except for those where both are network edges).
 
 	\begin{algorithm}
		\caption{networkEdgeAdd}
		\label{alg:networkEdgeAdd}
		\SetAlgoLined
		\KwData{Input network $N$ }
		\KwResult{Network neighborhood as list}
		\Comment{For each pair of edges in input network, add edge between each pair of edges, subject
		to time and edge tree vertex constraints} 
		$\lbrack N' \rbrack \leftarrow $all combinations of available edge joins\\
		\Return {$\lbrack N' \rbrack$}
	\end{algorithm}
	
\subsection{Display tree rearangement}
NNI, SPR, TBR etc.
 
 \section{\textit{Ab initio} build}
 	\begin{enumerate}
		\item{Build trees and create EUN}
		\item{fold over list of leaves = wagner build}
		\item{Build with forest and network rearrangements based on sequential addition of leaves}
	\end{enumerate}
	
\section{Forest Refinement}
Various combinations of forest, network, tree rearrangements.

 \section{Naive Tree Optimization}
 
 \section{Tree ReRoot Optimization}
 
 \section{Naive Soft-Wired Network Optimization}
 
 \section{Less-Naive Network Optimization}
 

	\begin{algorithm}
		\caption{Post-Order Soft-Wired Network Traversal and Optimization for arbitrary in-degree 
		and out-degree vertices}
		\label{alg:PON}
		\SetAlgoLined
		\KwData{Input network $N$ and data $D$}
		\KwResult{The minimum cost and tree displayed by network for each input character in $D$}
		For all characters $\in D cost_{character} \gets 0$\;
		\While {$nodesToDo = True$} {
			\If {$node_{out degree} = 0$} {$node_{prelim} = $observed} 
			\ElseIf {$node_{out degree} = 1$} {$node_{prelim} = node^{desc}_{prelim}$} 
			\ElseIf {$node_{out degree} = 2$} {$node_{prelim} = $ standard if not done and not desc of network edge} 
			\Else  {$node_{prelim} = $ ladderize} 
			\If {$node_{in-degree} = 0$} {$nodesToDo = False$}
			\ElseIf {$node_{in-degree} = 1$} {recurse to parent}
			\Else {split tree then recurse each parent}
		} 
		\Return{$[(tree, [cost])]$ can choose best displayed tree for each character, later
		do reroot thing (3d-Varon and Wheeler (or Malign too) for dynamic characters} 					
	\end{algorithm}
	
% Full (postorder + preorder) Component Optimization	
	\begin{algorithm}
		\caption{FullComponentOptimization}
		\label{alg:FullComponentOpt}
		\SetAlgoLined
		\KwData{Input component $C$ and data $D$}
		\KwResult{Pair of component optimality value and decorated (edge labels, etc) component}
		\Comment{Get optimality value of component and postorder labelling of component}
		$(optimalityValue, postOrderAssignment) =$ PostOrderComponent $C$ $D$
		
		\Comment{Get preorder `final' labelling of component (vertex and edges costs assignments etc)}
		$finalLabelling =$ PreOrderOptimization $postOrderAssignment$ $D$
		\Return{(optimalityValue, finalLabelling)} 					
	\end{algorithm}
	
% Postorder (="down pass") Component Optimization	
	\begin{algorithm}
		\caption{PostOrderOptimization}
		\label{alg:PostOrderOptimization}
		\SetAlgoLined
		\KwData{Input data $D$}
		\KwResult{Blah}
		Comment 
		\Return{(Optimization value, labelled component)} 					
	\end{algorithm}

% Preorder (="down pass") Component Optimization	
	\begin{algorithm}
		\caption{PreOrderOptimization}
		\label{alg:PreOrderOptimization}
		\SetAlgoLined
		\KwData{Input data $D$}
		\KwResult{Blah}
		Comment 
		\Return{labelled component} 					
	\end{algorithm}

%Algorithm basic template
%	\begin{algorithm}
%		\caption{Blah}
%		\label{alg:Blah}
%		\SetAlgoLined
%		\KwData{Input data $D$}
%		\KwResult{Blah}
%		\Comment{Blah Comment}
%		\Return{Bleh} 					
%	\end{algorithm}

\newpage
\singlespacing
\bibliography{/users/ward/home/biblio/ward_new}
%\bibliography{ward_new}

%cost of root for pars = insert all characters, good first aapprox

\end{document}
