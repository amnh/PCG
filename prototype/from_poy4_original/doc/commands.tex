\documentclass[11pt]{book}

\usepackage[htt]{hyphenat}
\usepackage{multind}
\usepackage[pdftitle={POY 4 Documentation},pdfauthor={Andres Varon et. al.},
pdfkeywords={phylogenetic analysis, direct optimization, POY}]{hyperref}
\usepackage{color}
\usepackage{xspace}
\usepackage{framed}
\usepackage{graphicx}
\usepackage{rotating} % to create landscape pages
%\usepackage{makeidx}

%Indexes
\makeindex{general}
\makeindex{poy3}

\newlength\sidebar
\newlength\envrule
\newlength\envborder
\newlength\boxwidth

\setlength\sidebar{1.5mm}
\setlength\envrule{0.4pt}
\setlength\envborder{2.5mm}

\definecolor{exampleborder}{rgb}{0,0,.7}
\definecolor{examplebg}{rgb}{.9,.9,1}
\definecolor{shadecolor}{rgb}{.9,.9,1}
\definecolor{statementborder}{rgb}{.9,0,0}
\definecolor{statementbg}{rgb}{1,.9,.9}

\newsavebox\envbox
\newlength\notelength

\newenvironment{statement}[1][NOTE]{
% Default statement has no title %ilya: Statement is used for NOTES
\SpecialEnv{#1}{statementborder}{statementbg}{statementborder}{}%
}{%
\endSpecialEnv}

\def\Empty{}

% #1 title (if any)
% #2 sidebar (and title bg) color
% #3 background color
% #4 border color (or null for no border)
% #5 \enspace

	\newenvironment{SpecialEnv}[5]{%
	\par
	\def\EnvSideC{#2}% To use later (in end)
	\def\EnvBackgroundC{#3}%
	\def\EnvFrameC{#4}% 
	\flushleft

%\setlength\leftskip{-\sidebar}%
%\addtolength\leftskip{-\envborder}%
\noindent \nobreak
% Check if title is null:
\ifx\delimiter#1\delimiter\else
% If a title is specified, then typeset it in reverse color
\colorbox{\EnvSideC}{%
%\hspace{-\leftskip}% usually positive
%\hspace{-\fboxsep}%
\footnotesize\sffamily\bfseries\textcolor{white}{#1}%
%\hspace{\envborder}}%
}
\par\nobreak
\setlength\parskip{-0.2pt}% Tiny overlap to counter pixel round-off errors
\nointerlineskip 
\fi

% Make side-bar
\textcolor{\EnvSideC}{\vrule width\sidebar}%
% collect body in \envbox:
\begin{lrbox}\envbox 
\setlength{\boxwidth}{\linewidth}
\settowidth{\notelength}{NOT}
\addtolength{\boxwidth}{-1\notelength}
\addtolength{\boxwidth}{-1\sidebar}
\addtolength{\boxwidth}{-1\envborder}
\begin{minipage}[l]{\boxwidth}%

% insert counter, if any:
\ifx\delimiter#5\delimiter\else
%\theexample.\enspace
\fi
\ignorespaces
}{\par
\end{minipage}
\end{lrbox}%
% body is collected. Add background color
\setlength\fboxsep\envborder
\ifx\EnvFrameC\Empty % no frame
\colorbox{\EnvBackgroundC}{\usebox\envbox}%
\else % frame
\setlength\fboxrule\envrule
\addtolength\fboxsep{-\envrule}%
\fcolorbox{\EnvFrameC}{\EnvBackgroundC}{\usebox\envbox}%
\fi
\nobreak \hspace{-2\envborder}\null
\endflushleft
}


\newenvironment{poyexamples}{ \subsubsection{Examples} \begin{itemize}}{\end{itemize}}
% We define a command environment for new command definitions, and store
% whatever command we are dealing with in the @commandname macro. Inside a
% command we can specify the defaults, the syntax, and the arguments to be used.
\newenvironment{command}[2]{
    \def\tmpa{}
    \def\tmpb{#2}
    \def\@commandname{#1}
    \subsection{#1}\index{general}{#1}
    \ifx\tmpa\tmpb 
        \label{comm:#1} 
    \else 
        \label{comm:#2} 
    \fi}
    {}
% Syntax definition. We use the name of the command as stured in @commandname
\newcommand{\syntax}{\subsubsection{Syntax} \@commandname} 
\newcommand{\atsymbol}{@}

% We need to choose properly if we need to open a description or not inside
% an argument or argument group environment. We will use these two variables
% to keep the proper value to be used on each point.
\def\opendescription{\begin{description}}
\def\closedescription{}

% We define a pair of commands to initialize and finalize a description list 
% and set the necessary values of opendescription and closedescription.
\newcommand{\initdescription}{
    \def\closedescription{\end{description}}
    \opendescription
    \def\opendescription{}
}
\newcommand{\finishdescription}{
    \closedescription
    \def\closedescription{}
    \def\opendescription{\begin{description}}
}

% A variable to store a prefix for argument definitions, usually a colon.
\def\argprefix{} 

\newenvironment{script}{\begin{verbatim}}{\end{verbatim}}
\newcommand{\poyexample}[2]{\item \commandstyle{#1} \\#2}
\newcommand{\commandstyle}[1]{\texttt{#1}}
\newcommand{\poycommand}[1]{\commandstyle{#1}}

% Argument specifications
\newcommand{\poyargument}[1]{\commandstyle{#1}}
\newcommand{\obligatory}[1]{\commandstyle{\argprefix#1}}
\newcommand{\optional}[1]{\commandstyle{[\argprefix#1]}}
\newenvironment{arguments} {\subsubsection{Arguments}}{ \finishdescription }
\newenvironment{argumentgroup}[2]{\paragraph{#1} #2}{ \finishdescription }
\newcommand{\argumentdefinition}[4]{
    % Check if we are inside an itemize environment or not, if not, start
    % it.
    \initdescription
    % We will check if the second argument is empty; if so, we don't need
    % the add the : prefix for the argument's value.
    \def\tmpa{}
    \def\tmpb{#2}
    \def\tmpc{#4}
    \def\tmpd{INTEGER}
    \def\tmpe{#1}
    \def\tmpg{FLOAT}
    \def\tmph{STRING}
    \def\tmpi{BOOL}
    \def\tmpj{LIDENT}
    \ifx\tmpd\tmpe
        \index{general}{\@commandname!#1}
    \else
    \ifx\tmpg\tmpe
        \index{general}{\@commandname!#1}
    \else
    \ifx\tmph\tmpe
        \index{general}{\@commandname!#1}
    \else
    \ifx\tmpi\tmpe
        \index{general}{\@commandname!#1}
    \else
    \ifx\tmpj\tmpe
        \index{general}{\@commandname!#1}
    \else
    \index{general}{\commandstyle{\poyargument{#1}}}
    \def\tmpk{(STRING, STRING)}
        \index{general}{\@commandname!#1}
    \fi
    \fi
    \fi
    \fi
    \fi
    \ifx\tmpa\tmpb
        \item[\poyargument{#1}]
            \ifx\tmpa\tmpc
                \label{comm:#1}
            \else
                \label{comm:#4}
            \fi
                #3
    \else
        \def\argprefix{:}
        \item[\poyargument{#1#2}]
            \ifx\tmpa\tmpc
                \label{comm:#1}
            \else
                \label{comm:#4}
            \fi
            #3
    \fi
    \def\argprefix{}
    }

\newenvironment{poydescription}{\subsubsection{Description}}{}

% The primitive types of a POY script.
\newcommand{\poystring}{\commandstyle{STRING}\xspace}
\newcommand{\poyfloat}{\commandstyle{FLOAT}\xspace}
\newcommand{\poyint}{\commandstyle{INTEGER}\xspace}
\newcommand{\poybool}{\commandstyle{BOOL}\xspace}
\newcommand{\poylident}{\commandstyle{LIDENT}\xspace}

\newcommand{\poydefaults}[2]{\subsubsection{Defaults} \commandstyle{\@commandname(#1)} #2}

\newenvironment{poyalso}{\subsubsection{See also}
\begin{itemize}}{\end{itemize}}

% Cross References
\newcommand{\cross}[1]{\item \commandstyle{#1} (Section~\ref{comm:#1})}
\newcommand{\ncross}[2]{\item \commandstyle{#1} (Section~\ref{comm:#2})}

\newcommand{\ccross}[1]{\commandstyle{#1} (Section~\ref{comm:#1})}
\newcommand{\nccross}[2]{\commandstyle{#1} (Section~\ref{comm:#2})}

% The typesetting of POY
\newcommand{\poy}{\commandstyle{POY4}\xspace}
\newcommand{\poyv}{\commandstyle{POY4}\xspace} % poyv stands for "poy version 4"

% Using same footnotes multiple times (used for authorship)
\newcommand{\footnoteremember}[2]{
  \footnote{#2}
  \newcounter{#1}
  \setcounter{#1}{\value{footnote}}
}
\newcommand{\footnoterecall}[1]{
  \footnotemark[\value{#1}]
}

% Hyphenations
\hyphenation{mo-le-cu-lar an-aly-ses an-aly-sis au-to-ma-ti-cally ho-mo-lo-gy chro-mo-so-me chro-mo-so-me-le-vel op-ti-mi-za-tion}


\include{version}

\makeatletter
\def\thickhrulefill{\leavevmode \leaders \hrule height 1pt\hfill \kern \z@}
\renewcommand{\maketitle}{\begin{titlepage}%
    \let\footnotesize\small
    \let\footnoterule\relax
    \parindent \z@
    \reset@font
    \null
    \vskip 50\p@
    \begin{center}
      {{\huge \texttt{POY} \smallbuildnumber}
      \par 
      \vskip \baselineskip
      \hrule height 1pt 
      \par 
      \vskip \baselineskip
      Program Documentation 
      \par
      \small Version \buildnumber }\par
    \end{center}
    \vskip 85\p@
    \begin{flushright}
      \@author \par
    \end{flushright}
    \vskip 85\p@
    \begin{center}
    \includegraphics[width=\textwidth]{doc/figures/amnhlogoblue2.pdf}
    \end{center}
    \vfil
    \null
  \end{titlepage}%
  \setcounter{footnote}{0}%
}

\makeatother

\author{\textbf{Program and Documentation} \\ Andr\'es Var\'on \\ Le Sy Vinh \\
Illya Bomash \\ Ward C. Wheeler \\ \bigskip \textbf{Documentation} \\ Ilya T\"emkin \\ Megan Cevasco \\ Kurt M. Pickett \\ Juli\'an Faivovich \\ Taran Grant \\ William Leo Smith}


%%%%%%%%%%%%%%%%%%%%%%%%%
%	BODY TEXT OF THE DOCUMENT     
%%%%%%%%%%%%%%%%%%%%%%%%%

\begin{document}

\maketitle

\thispagestyle{empty}

\vspace*{5.5cm}

\begin{flushleft}
    \small
{\it
Andr\'es Var\'on}\\
Division of Invertebrate Zoology, American Museum of Natural History, New York, NY, U.S.A. \\
Computer Science Department, The Graduate School and University Center, The City University of New York, NY, U.S.A.  \\
\smallskip 
{\it
Le Sy Vinh\\
Ward C. Wheeler\\
Ilya T\"emkin \\
Megan Cevasco \\
}
Division of Invertebrate Zoology, American Museum of Natural History, New York, NY, U.S.A.\\
\smallskip
{\it
Illya Bomash}\\
Department of Physiology And Biophysics, Weill Medical College of Cornell University, New York, NY, U.S.A.\\
\smallskip
{\it
Kurt M. Pickett}\\
Department of Biology, University of Vermont, Burlington, VT, U.S.A.\\
\smallskip
{\it
Juli\'an Faivovich}\\
Departamento de Zoologia, Instituto de Bioci\^encias, Universidade Estadual Paulista, Brasil\\
\smallskip
{\it
Taran Grant}\\
Faculdade de Bioci\^encias, Pontif�cia Universidade Cat\'olica do Rio Grande do Sul (PUCRS), Brasil\\
\smallskip
{\it
William Leo Smith}\\
Department of Zoology, The Field Museum of Natural History, Chicago, IL, U.S.A.\\

\vspace*{0.5cm}
The American Museum of Natural History\\
\copyright 2007, 2008 by Andr\'es Var\'on, Le Sy Vinh, Illya Bomash, Ward
Wheeler, and The American Museum of Natural History \\
All rights reserved. Published 2008

\vspace*{0.5cm}

\emph{Var\'on, A., L. S. Vinh, W. C. Wheeler.} 2010.
\texttt{POY} version 4: phylogenetic analysis using dynamic homologies.
Cladistics. 26: \textit{in press}. 

\vspace*{0.5cm}

Documentation by Var\'on, A., L. S. Vinh, I. Bomash,
W. Wheeler, I. T\"emkin, M. Cevasco, K. M. Pickett, J. Faivovich, T. Grant, and
W. L. Smith. \url{http://research.amnh.org/scicomp/projects/poy.php}

\vspace*{0.5cm}

Available online at
\url{http://research.amnh.org/scicomp/projects/poy.php}
and
\url{http://code.google.com/p/poy4/}

\end{flushleft}

\tableofcontents

\chapter{What is \poy}

\poy is a flexible, multi-platform program for phylogenetic analysis of molecular and other data. %under various optimality criteria.
An essential feature of \poy is that it implements the concept of dynamic homology \cite{wheeler2001a, wheeler2001} allowing optimization of unaligned sequences. \poy offers flexibility for designing heuristic search strategies and implements an array of algorithms including multiple random addition sequence, swapping, tree fusing, tree drifting, and ratcheting. As output, \poy generates a comprehensive character diagnosis, graphical representations of cladograms and their user-specified consensus, support values, and implied alignments. \poy provides a unified approach to co-optimizing different types of data, such as morphological and molecular sequence data. In addition, \poy can analyze entire chromosomes and genomes, taking into account large-scale genomic events (translocations, inversions, and duplications).

%Currently \poy is beta software, and therefore it has some known glitches. Most
%of them will be worked out in the following months, and updated versions will be
%available on the program's webpage as we produce them. Our current schedule of
%work expect to have a final official version of the parsimony components of the
%program and a release of the beta components of the maximum
%likelihood components in mid may of 2008.

\section{The structure of \poy documentation}
This first chapter, \emph{\poy Quick Start}, will get you started using \poy. The first few sections are intended to provide detailed instructions on how to obtain and install \poy, introduce the user to the program's two working environments, the \emph{Graphical User Interface} and the \emph{Interactive Console}. These sections also show how to initiate a \poy session and point to the various resources to obtain further assistance. Subsequent sections build on that knowledge and give step-by-step examples on how to conduct a basic analysis and visualize the results. The second chapter, \emph{\poy Commands}, describes \poy commands and their valid syntax. It also includes examples of simple operations for every command. The third chapter discusses the heuristic procedures used in \poy. Their understanding helps creating building efficient search strategies. More advanced operations are described in the fourth chapter, \emph{\poy Tutorials}. In addition to the general index, this document contains a \emph{\texttt{POY3.0} Command Line Index}, intended to provide a link between the commands used in \texttt{POY3} and the commands used in \poy. 

\chapter{\poy Quick Start}

\section{Requirements: software and hardware}

\subsection{Software}
\poy is a platform-independent, open-source program that can be compiled for many operating systems  and hardware configurations, including Mac OSX, Microsoft Windows XP, and Linux. \poy \emph{binaries}\index{general}{binaries} (compiled application file) is the only piece of software necessary to run \poy. The intuitive graphical user interface of \poy provides all the functionality for running analyses using pull-down menus and field selections, as well as creating and running \poy scripts. Some utility programs (such as Notepad and Ghostscript for Windows, TextEdit for Mac, or Nano for Linux), can help preparing \poy scripts and formatting datafiles, while others (such as Adobe Acrobat) can facilitate viewing the graphical output in PDF (Portable Document Format).

\subsection{Hardware}
\poy runs on a variety of computers from laptops and desktops to Beowulf clusters 
of various sizes to symmetric multiprocessing hardware. There are no
particular requirements for disk space, but XML and diagnosis report files can be large.

\section{Obtaining and installing \poy}

\poy installers for Linux, Windows XP, and Mac OSX, source code, and documentation in PDF format are available from the \poy website at the American Museum of Natural History Computational Sciences:
\begin{center}
\url{http://research.amnh.org/scicomp/projects/poy.php}
\end{center}
The latest source code can also be obtained from \poy Google Group website:
\begin{center}
\url{http://code.google.com/p/poy4/source}
\end{center}
The following detailed step-by-step instructions will guide you through downloading and installing \poy binaries for various platforms.

\begin{flushleft}
	\begin{minipage}[c]{0.074\textwidth}
	   	\includegraphics[width=\textwidth]{doc/figures/figlogowindows.jpg}
	\end{minipage}
	\quad
	\begin{minipage}[t]{0.88\textwidth}
		   	\subsubsection{Windows}
	\end{minipage}
		\begin{itemize}
			\item
                Download
                \href{http://research.amnh.org/scicomp/projects/poy.php}{\emph{poy4} folder} to the desktop by selecting \emph{WinXP} download link.

			\item 
                Open \emph{POY\_Installer.exe} from the \emph{poy4} folder and follow the installation instructions. You will need Administrator privileges to install the application. 
                By default, \poy is installed without parallel execution support. If you have Windows XP SP2 or Windows Vista and more than one core or processor, you can take advantage of your processing power by installing the parallel components. To do so, instead of a typical installation, select the complete installation. 
                
                \emph{Important:} the complete installation includes MPICH2 1.06 p1. MPICH is used to communicate processes during parallel execution. If you already have MPICH2 installed (if you didn't know what it is you most likely don't have it), select the custom installation option and remove that component. During the first execution in parallel you will be asked by the Windows Firewall to unblock \poy and MPICH, this is necessary for the successful execution of the program.
		\end{itemize}

	\begin{minipage}[c]{0.074\textwidth}
   		\includegraphics[width=\textwidth]{doc/figures/figlogomac.jpg}
	\end{minipage}
	\quad
	\begin{minipage}[t]{0.88\textwidth}
	   	\subsubsection{Mac OSX}
	\end{minipage}
	            \begin{itemize}
			\item Download
            \href{http://research.amnh.org/scicomp/projects/poy.php}{\emph{poy\_parallel\_buildXXXX.dmg} disk image} to the desktop.
            		\item Drag the \poy application from the
            disk \emph{poy4} and drop it into the \emph{Applications}
            folder on the hard drive.
		\end{itemize}

	\begin{minipage}[c]{0.074\textwidth}
   		\includegraphics[width=\textwidth]{doc/figures/figlogolinux.jpg}
	\end{minipage}
	\quad
	\begin{minipage}[t]{0.88\textwidth}
	   	\subsubsection{Linux}
	\end{minipage}
		\begin{itemize}
			\item  Download the 
    \href{http://research.amnh.org/scicomp/projects/poy.php}{gzipped} file.
    			\item Untar and ungzip the \emph{poy4.tar.gz} file.
			\item Run the command \texttt{tar -Pxvzf poy4.tar.gz} as a
    super user in the newly created \emph{poy4} directory.
    The GUI will be installed in \texttt{/opt/poy4/Contents/POY} directory
    and terminal binaries in \texttt{/opt/poy4/Resources/ncurses\_poy} directory.
		\end{itemize} 

\end{flushleft}

\subsubsection{Compiling from the source}

In order to compile \poy the following tools are required:

\begin{enumerate}
    \item \href{http://www.gnu.org/software/make/}{The GNU Make 3.8 utility;}
    \item \href{http://www.ocaml.org}{OCaml version 3.10.0. or later;}
    \item \href{http://gcc.gnu.org/}{A C compiler, for example The GNU Compiler Collection;}
    \item \href{http://www.zlib.net}{The zlib compression library;}
    \item Optionally, the ncurses library for a nice interactive console or the read line library. If none is available,
    the flat interface does not require read line or ncurses.
\end{enumerate}

Download, ungzip, and untar the
\href{http://research.amnh.org/scicomp/projects/poy.php}{\poy source code};
In order to compile under default setting type:
\begin{verbatim}
./configure
make
make install
\end{verbatim}
All the configuration options can be found in {\tt ./configure --help}.

\poy can also be run in parallel environments using the
\href{http://www-unix.mcs.anl.gov/mpi/}{Message Passing Interface}. Your system administrator has likely already one installed and should be able to provide you with proper
paths to set your config file.

\section{The Graphical User Interface}

\poy provides two working environments: the \emph{Graphical User Interface} and the \emph{Interactive Console}.  The \emph{Graphical User Interface} has a user-friendly appearance as other native stand-alone applications where different functions are accessible through menus and windows. Thus, the entire analysis can be carried out clicking on appropriate selections and, where necessary, typing specifications in designated fields. The \emph{Interactive Console}, however, requires a detailed knowledge of \poy commands, their arguments, and the conventions of \poy scripting. All these features are described in the \emph{POY Commands} chapter (\ref{commands}).

Even though the Mac OSX version of the \emph{Graphical User Interface} is used for screen shots throughout this chapter, the Linux(GTK+) and Windows versions contain the same items and functionality, differing only in the generic window format specific to each platform.

When \poy is first opened, two items appear on the screen: the menu bar across the top and the \emph{POY Launcher} window (Figure~\ref{fig:menu_launcher_window}). (Note that in Linux and Windows the menu bar is within the launcher window.)

\begin{figure}[htpb]
    \begin{center}
        \includegraphics[width=0.5\textwidth]{doc/figures/menu_launcher_window.jpg}
    \end{center}
    \caption{The \poy menu bar and the \emph{POY Launcher} window. These items appear when \poy is opened.}
    \label{fig:menu_launcher_window}
\end{figure}

\subsection{POY menu bar}
The menu bar contains the following drop-down menus:
\begin{description}
\item[POY] (Mac OSX only) contains generic items as other Mac OSX applications. It includes \emph{Quit POY} tab that closes the program. This menu also allows contains selection of the \emph{About POY} window (Figure~\ref{fig:about_window}) that lists the current version of \poy, a copywrite statement, and the address of the \poy website.
\item[Analyses] contains options for different types of tree searches, calculation of support values, tree diagnosis, and their respective outputs. Other items in this menu open the \emph{POY Launcher} and the \emph{Interactive Console}.
\item[Edit] contains standard tools for deleting, copying, cutting, pasting, undoing, and selecting.
\item[View] opens the \emph{Output} window to display the results (including warning and error messages) and the current state of the analysis. It also contains the \emph{About POY} menu item in Windows and Linux.
\item[Help] opens the \poy \emph{Program Documentation} in PDF format (requires a PDF viewer).
\end{description}

\begin{figure}[htpb]
    \begin{center}
        \includegraphics[width=0.5\textwidth]{doc/figures/about_window.jpg}
    \end{center}
    \caption{The \emph{About POY} window.}
    \label{fig:about_window}
\end{figure}

\subsection{POY Launcher} 
The \emph{POY Launcher} is the only window that automatically opens upon starting
\poy. This allows the user to import a previously created script,
designate a working directory, specify the number of processors,
and start the analysis.

\begin{description}
	\item[Select the script to run]
     Allows the user to specify the location of a \poy script.
	\item[Select the working directory]
    The working directory is the
    directory that contains the input data and output files. By default, the working directory is set to be the same as the
    directory containing the selected \poy script. 
	\item[Select the number of processors]
    The selection of the number of processors is disabled for Linux
    and Windows platforms. Once specified, the selection is applied
    to all subsequent analyses in the current \poy session. (Parallelization is not supported in interactive sessions, see Section~\ref{interactiveconsole}.)
	\item[Run the analysis]
    Clicking the \emph{Run} button starts the execution of the selected
    script. Once the script is initiated, the \emph{Run} button
    becomes the \emph{Cancel} button that can be used to interrupt
    a \poy session.
\end{description}

If the \emph{Run} button is clicked without the selected script and
working directory, or the names of the scripts and working directory are entered incorrectly, \poy issues an
error message in the upper part of the \emph{POY Launcher} window,
such as \texttt{POY finished with an error}.

\subsection{The \emph{Analyses} menu}
The \emph{Analyses} menu is the main toolbox of the \poy GUI interface (Figure~\ref{fig:simple_search_window}, left). Selections are subdivided into four functional categories. The first three deal with tree searching, support calculation, and tree diagnosis; the fourth one is used for  script management or interactive command execution that bypasses the menu-driven script generating. Each of the menu items is described below in the order it appears on the menu.

Most options are consistently applied through different kinds of analysis. Therefore, all the options are described in detail only for the \emph{Simple Search} analysis. The descriptions of other analyses are made with reference to the the \emph{Simple Search} and focus on unique options.

\subsubsection{\emph{Tree searching options}}

\subsubsection{Simple Search}
A typical search involves a series of steps. First, initial trees are generated by random addition sequence from the imported character data. These trees are then subjected to branch swapping, subsequent to which a subset of trees is selected for the report.
The \emph{Simple Search} window (Figure~\ref{fig:simple_search_window}, right)
provides the most common and basic options for a standard tree search
in \poy that must either (in some cases or) be selected by clicking appropriate buttons or typed. Note that \emph{all the empty fields must be filled in}. The window is subdivided into four sections: 

\begin{figure}
\centering
\begin{minipage}[c]{0.48\textwidth}
   		\includegraphics[width=\textwidth]{doc/figures/simplesearch_menu.jpg}
\end{minipage}
\quad
\begin{minipage}[c]{0.48\textwidth}
	   	\includegraphics[width=\textwidth]{doc/figures/simplesearch_window.jpg}
   	\end{minipage}
	
\caption{The \emph{Simple Search} window. Selecting \emph{Simple Search} from the \emph{Analysis} menu (left) opens the \emph{Simple Search} window options (right).}
\label{fig:simple_search_window}
\end{figure}

\begin{description}
    \item[Input Files]
        Contains the list of files that are to be input into \poy. These include
        character files in nucleotide, Hennig86, and Nexus formats and tree files. (Character data in other formats can be imported by specifying additional argument in the script. See \ccross{read}.)
    \item[Search Parameters]
        Holds one field to set the number of independent random addition replicates to be generated.
    \item[Sequence Alignment Parameters]
        Holds fields to specify the substitution, indel, and gap opening costs. Enter \texttt{0} if no
        gap opening cost is desired. If the value of a parameter is not specified, the default values is used. (See the \emph{POY Commands} chapter (\ref{commands}).)
    \item[Output Files]
        Designates the names and locations of files containing different kinds of results
        (implied by their respective titles) of the analysis. If no names are specified, the default names are generated.
\end{description}

Once all the parameters are selected, click the \emph{Make Script} button and another
window--the \emph{Script Editor} --containing the generated script appears on screen (Figure~\ref{fig:ScriptEditor_Window}). The
script can be edited by typing in the commands directly in the \emph{Script Editor} window,
 saved (by clicking the \emph{Save As} button), or replaced with another script (using 
 the \emph{Open} button). To start the analysis, click the \emph{Run} button in the 
 \emph{Script Editor} window. When the \emph{Run} button is clicked, \poy will issue a
 request to save the script. Thus, not only does \poy execute the script but
 it also creates the record of the type of analysis (including all user-defined specifications) that was performed.
 
 \begin{figure}
\centering
\begin{minipage}[c]{0.48\textwidth}
   		\includegraphics[width=\textwidth]{doc/figures/simplesearch_window_filled.jpg}
\end{minipage}
\quad
\begin{minipage}[c]{0.48\textwidth}
	   	\includegraphics[width=\textwidth]{doc/figures/simplesearch_script.jpg}
   	\end{minipage}
	
    \caption{The \emph{Simple Search} window with specified search parameters and output files (left) and the corresponding \emph{Script Editor} window.}
    \label{fig:ScriptEditor_Window}
\end{figure}

\subsubsection{Timed Search}

Timed search (Figure~\ref{fig:timed_search}) implements a default search strategy that effectively combines tree building with TBR branch swapping, parsimony ratchet, and tree fusing.  The \emph{Timed Search} window has the same four parameter groups described for the \emph{Simple Search}. However, the \emph{Search Parameters} section (called \emph{Search and Perturb Parameters}) contains four fields specifying the search targets instead of the \emph{Repetitions} field. These include the following:

\begin{figure}
\centering
\begin{minipage}[c]{0.48\textwidth}
   		\includegraphics[width=\textwidth]{doc/figures/timedsearch_menu.jpg}
\end{minipage}
\quad
\begin{minipage}[c]{0.48\textwidth}
	   	\includegraphics[width=\textwidth]{doc/figures/timedsearch_window.jpg}
   	\end{minipage}
	
\caption{The \emph{Timed Search} window. Selecting \emph{Timed Search} from the \emph{Analysis} menu (left) and viewing the \emph{Timed Search} window options (right).}
\label{fig:timed_search}
\end{figure}

\begin{description}
    \item[Maximum time] The maximum total execution time for the search. The time is specified as
        days:hours:minutes.
    \item[Minimum time] The minimum total execution time for the search. The time is specified as
        days:hours:minutes.
    \item[Maximum memory] The maximum amount of memory allocated for the search.
    \item[Minimum hits] The minimum number of times that the minimum cost must be reached before aborting the search.
\end{description}

\subsubsection{Search with Ratchet}

The parsimony ratchet is a heuristic strategy to escape the local optima during tree searching~\cite{Nixon1999}. The \emph{Search with Ratchet} (Figure~\ref{fig:search_with_ratchet_window}) follows the same basic steps of a simple search but includes the ratchet step after the swap. In addition to the same four parameter groups described for the \emph{Simple Search} window, the \emph{Search Parameters} section provides the following ratchet parameters fields:

\begin{figure}
\centering
\begin{minipage}[c]{0.48\textwidth}
   		\includegraphics[width=\textwidth]{doc/figures/searchwithratchet_menu.jpg}
\end{minipage}
\quad
\begin{minipage}[c]{0.48\textwidth}
	   	\includegraphics[width=\textwidth]{doc/figures/searchwithratchet_window.jpg}
   	\end{minipage}
	
\caption{The \emph{Search with Ratchet} window. Selecting \emph{Search with Ratchet} from the \emph{Analysis} menu (left) and viewing the \emph{Search with Ratchet} window options (right).}
\label{fig:search_with_ratchet_window}
\end{figure}

\begin{description}
    \item[Ratchet iterations] The number of iterations for the parsimony
        ratchet.
    \item[Severity] The severity parameter of the parsimony ratchet (the weight
        change factor for the selected characters).
    \item[Percentage] The percentage of characters to be reweighted during ratcheting.
\end{description}

\subsubsection{Search with Perturb}

\emph{Search with Perturb} (Figure~\ref{fig:search_with_perturb_window}) provides an alternative means to escape local optima by changing the transformation cost matrix of the molecular characters, a procedure similar in spirit to the parsimony ratchet. In addition to the
same four parameter groups described for the \emph{Simple Search} window, the \emph{Search
with Perturb} window provides three extra fields with the parameters for the
transformation cost matrix perturbation as follows:

\begin{figure}
\centering
\begin{minipage}[c]{0.48\textwidth}
   		\includegraphics[width=\textwidth]{doc/figures/searchwithperturb_menu.jpg}
\end{minipage}
\quad
\begin{minipage}[c]{0.48\textwidth}
	   	\includegraphics[width=\textwidth]{doc/figures/searchwithperturb_window.jpg}
   	\end{minipage} 
\caption{The \emph{Search with Perturb} window. Selecting \emph{Search with Perturb} from the \emph{Analysis} menu (left) and viewing the \emph{Search with Perturb} window options (right).}
\label{fig:search_with_perturb_window}
\end{figure}

\begin{description}
    \item[Perturb iterations] Sets the number of perturb iterations to be performed.
    \item[Substitutions] Specifies the cost of the perturbed substitutions.
    \item[Indels] Specifies the cost of the perturbed indels.
\end{description}

\subsubsection{\emph{Support calculation options}}

None of the support calculation windows include functions for tree building and searching. Therefore, one of the input files must contain trees for which support values are going to be calculated.

\subsubsection{Bootstrap}

The \emph{Bootstrap} window (Figure~\ref{fig:bootstrap}) specifies parameters
for estimating the Bootstrap support values. In addition to the \emph{Simple Search} window fields, it
contains a \emph{Pseudoreplicates} field to specify the number of bootstrap pseudoreplicates.

\begin{figure}
\centering
\begin{minipage}[c]{0.48\textwidth}
   		\includegraphics[width=\textwidth]{doc/figures/bootstrap_menu.jpg}
\end{minipage}
\quad
\begin{minipage}[c]{0.48\textwidth}
	   	\includegraphics[width=\textwidth]{doc/figures/bootstrap_window.jpg}
   	\end{minipage}
\caption{The \emph{Bootstrap} window. Selecting \emph{Bootstrap} from the \emph{Analysis} menu (left) and viewing the \emph{Bootstrap} window options (right).}
\label{fig:bootstrap}
\end{figure}

\subsubsection{Bremer}

The \emph{Bremer} option (Figure~\ref{fig:search_for_bermer_menu}) is divided into two windows: the \emph{Search for Bremer} window, that specifies the Bremer support \cite{Bremer1988, Kallersjoetal1992} calculation parameters, and the \emph{Report Bremer} window to format the output of the results (Figure~\ref{fig:search_report_bremer}). 

\paragraph{Search for Bremer}

\begin{figure}[htpb]
    \begin{center}
        \includegraphics[width=0.65\textwidth]{doc/figures/searchforbremer_menu.jpg}
    \end{center}
    \caption{ Selecting the \emph{Bremer} windows from the \emph{Analysis} menu.}
    \label{fig:search_for_bermer_menu}
\end{figure}

\begin{figure}
\centering
\begin{minipage}[c]{0.48\textwidth}
   		\includegraphics[width=\textwidth]{doc/figures/searchforbremer_window.jpg}
\end{minipage}
\quad
\begin{minipage}[c]{0.48\textwidth}
	   	\includegraphics[width=\textwidth]{doc/figures/reportbremer_window.jpg}
   	\end{minipage}
\caption{Viewing the options of the \emph{Search for Bremer} (left) and the \emph{Report Bremer}(right) windows.}
\label{fig:search_report_bremer}
\end{figure}

The script produced in this window collects trees visited during a search for Bremer support calculations. This
search can take a long time, as the goal of this search strategy is to broadly sample variation among trees, and guarantee that all
clades have Bremer support values. 

In addition to the standard four sections defined for the \emph{Simple Search} window,
note that one of the output files is the \emph{Temporary Trees} file, which 
contains all the information required to produce the bremer support tree
results in the \emph{Report Bremer} window. Make sure to choose a file name that does not overwrite this output.

If the search does not finish within the time frame amenable to the user the search can be interrupted and the intermediate results remain stored in the \emph{Temporary Trees} file.  As Bremer calculations are upper-bound values, terminating the search prior to completion and, thus, storing a smaller pool of visited trees may inflate support values relative to those generated by a more exhaustive search. The trees from the \emph{Temporary Trees} file can then be reported using the \emph{Report Bremer} window.

\paragraph{Report Bremer}
The script produced in this window takes the \emph{Temporary Trees} file generated in the \emph{Search for Bremer} window in the \emph{File with trees for bremer calculation} field. 

\subsubsection{Jackknife}

The \emph{Jackknife} window (Figure~\ref{fig:jackknife}) specifies parameters for estimating the
Jackknife support values. In addition to the \emph{Simple Search} window fields, \emph{Jackknife Parameters} it contains fields to specify
the number of Jackknife pseudoreplicates (\emph{Pseudoreplicates}) and the number of characters to remove (\emph{Remove}).

\begin{figure}
\centering
\begin{minipage}[c]{0.48\textwidth}
   		\includegraphics[width=\textwidth]{doc/figures/jackknife_menu.jpg}
\end{minipage}
\quad
\begin{minipage}[c]{0.48\textwidth}
	   	\includegraphics[width=\textwidth]{doc/figures/jackknife_window.jpg}
   	\end{minipage}
\caption{The \emph{Jackknife} window. Selecting \emph{Jackknife} from the \emph{Analysis} menu (left) and viewing the \emph{Jackknife} window options (right).}
\label{fig:jackknife}
\end{figure}

\begin{description}
    \item[Pseudoreplicates] Specifies the number of resampling iterations.
    \item[Remove] Specifies the percentage of characters being deleted during a pseudoreplicate.
\end{description}

\subsubsection{\emph{Diagnosis}}

\subsubsection{Diagnose Tree}

The \emph{Diagnose Tree} window (Figure~\ref{fig:diagnosetree}) specifies parameters for reporting a diagnosis of the input tree. This window lacks the \emph{Search Parameters} section because the diagnosis is performed on the trees resulted from prior searches and no new trees are generated during the diagnosis procedure.

\begin{figure}
\centering
\begin{minipage}[c]{0.48\textwidth}
   		\includegraphics[width=\textwidth]{doc/figures/diagnose_menu.jpg}
\end{minipage}
\quad
\begin{minipage}[c]{0.48\textwidth}
	   	\includegraphics[width=\textwidth]{doc/figures/diagnose_window.jpg}
   	\end{minipage}
\caption{The \emph{Diagnose} window. Selecting \emph{Diagnose Tree} from the \emph{Analysis} menu (left) and viewing the \emph{Diagnose} window options (right).}
\label{fig:diagnosetree}
\end{figure}

\subsubsection{\emph{Script editing and the Interactive Console}}

\subsubsection{Open POY Script}

Selecting \emph{Open POY Script} (Figure~\ref{fig:open_poy_script}) displays the \emph{POY Launcher} 
window (Figure~\ref{fig:menu_launcher_window}), the function of which is described above.

\begin{figure}[htpb]
    \begin{center}
        \includegraphics[width=0.5\textwidth]{doc/figures/openpoyscript_menu.jpg}
    \end{center}
    \caption{The \emph{Open POY script} selection opens the \emph{POY Launcher} window.}
    \label{fig:open_poy_script}
\end{figure}

\subsubsection{Run Interactive Console}

Selecting \emph{Run Interactive Console} (Figure~\ref{fig:runinteractive}) opens the ncurses interface
that enables the user to run the analysis interactively by entering
\poy commands directly via the command-line interface of the \emph{Interactive
Console}. Note that the \emph{Interactive Console} does \emph{not} run in parallel.

\begin{figure}
\centering
\begin{minipage}[c]{0.48\textwidth}
   		\includegraphics[width=\textwidth]{doc/figures/runinteractive_menu.jpg}
\end{minipage}
\quad
\begin{minipage}[c]{0.48\textwidth}
	   	\includegraphics[width=\textwidth]{doc/figures/create_script_menu.jpg}
   	\end{minipage}
\caption{The \emph{Run Interactive Console} selection (left) opens \poy interactive console in a new window. The \emph{Create Script} selection opens the \emph{Script Editor} window (Figure~\ref{fig:ScriptEditor_Window}).}
\label{fig:runinteractive}
\end{figure}

\subsubsection{Create Script}
The \emph{Create Script} selection opens a blank \emph{Script Editor} window that allows opening, creating, modifying, saving, and executing  a customized script.

\subsection{The \emph{View} menu}

The \emph{View} menu contains the \emph{Output} window which is subdivided into two fields: the upper \emph{Results and Errors} and lower \emph{Status} (Figure~\ref{fig:results_and_status_windows}). These fields display, respectively, the results (including warning and error messages) and the current state of the analysis. These fields are not updated automatically and in order to display the current state of the analysis the user must click the \emph{Update} button. The \emph{View} menu also contains the \emph{About POY} window in Windows and Linux.

\begin{figure}
\centering
\begin{minipage}[c]{0.48\textwidth}
   		\includegraphics[width=\textwidth]{doc/figures/view_menu.jpg}
\end{minipage}
\quad
\begin{minipage}[c]{0.48\textwidth}
	   	\includegraphics[width=\textwidth]{doc/figures/output_window.jpg}
   	\end{minipage}
\caption{Selecting the \emph{Output} window (left) and viewing the \emph{Results and Errors} and  \emph{Status of Search} fields.}
\label{fig:results_and_status_windows}
\end{figure}

\section{Using the Interactive Console} \label{interactiveconsole}

This section will help you get started using the \poy \emph{Interactive Console} and will prepare you for the
more extensive, technical descriptions in the next chapter, \emph{\poy Commands}. Now that you are acquainted with the program's interface, learned how to initiate, and exit or interrupt a \poy session, and how to obtain help, you are well prepared to run your first analysis. This chapter will teach
you how to input datafiles, check the data you are analyzing, generate
a set of initial trees, do basic branch swapping to find a local optimum, and, finally, produce
and visualize the resultant trees, their strict consensus, and generate support values.

For the purpose of this exercise, three datafiles are used available at \\
\texttt{http://research.amnh.org/scicomp/projects/poy.php}:

\begin{itemize}
	\item {\texttt{18s.fas} and \texttt{28s.fas} contain unaligned DNA sequences (partial 18S and 28S ribosomal DNA) in FASTA format.~\cite{pearson1988}}
	\item {\texttt{morpho.ss} contains a morphological data matrix in Hennig86 format.~\cite{farris1988}}
\end{itemize}

Once \poy has been launched and the interface (Figure~\ref{fig:figinterface}) had appeared on the screen, the data can be input and the analysis can proceed. As you follow the instructions, you are encouraged to consult the help file by using the command \commandstyle{help} (see Section~\ref{sec:help} to learn more about \poy commands and their arguments).

\subsection{The interface}

The \emph{Interactive Console} provides a terminal environment with enhanced ability to display the results and the state of the analysis. It is recommended to use the console to explore and verify the data in the early steps of the analysis, and to learn the scripting language. Using the console requires familiarity with \poy commands, their arguments, and the conventions of \poy scripting (which are discussed in the \emph{POY Commands} chapter). It has four windows: \emph{POY Output}, \emph{Interactive Console}, \emph{State of Stored Search}, and \emph{Current Job} (Figure \ref{fig:figinterface}):

\begin{figure}[htbp]
   \centering
   \includegraphics[width=0.7\textwidth]{doc/figures/figinterface.jpg}
   \caption{\poy interface displayed in the Terminal window prior to analysis. Note the cursor at the \poy prompt in the \emph{Interactive Console} and that the \emph{State of Stored Search} and \emph{Current Job} windows are empty.}
   \label{fig:figinterface}
\end{figure}

\begin{description}
\item[POY Output] (Figure \ref{fig:figinterface}, upper box) displays the status of the imported data, outputs the results of the phylogenetic analyses (such as trees, character diagnoses, and implied alignments), reports errors, and displays descriptions of \poy commands.
\item[Interactive Console] (Figure \ref{fig:figinterface}, mid-left box) is used to issue the commands interactively and to execute the commands by clicking the Return key. (See Section~\ref{commands} on the description of \poy commands.)
\item[State of Stored Search] (Figure \ref{fig:figinterface}, mid-right box) displays the time (in seconds) elapsed since the initiation of the current operation. This window also reports the number of trees currently in memory and displays the range of their costs.
\item[Current Job] (Figure \ref{fig:figinterface}, lower box) describes the currently running operation. When the operation is completed, the box is blank.
\end{description} 

\begin{figure}[htbp]
   \centering
   \includegraphics[width=0.7\textwidth]{doc/figures/figprocess.jpg}
   \caption{\poy interactive console during a process. The \emph{POY Output} window displays (by default) the information on the input datafiles. The \emph{Interactive Console} lists the commands that have been consecutively executed. The \emph{Current Job} window shows the state of the current operation and the current tree score. The \emph{State of Stored Search} shows the time elapsed  since the last command, \commandstyle{swap}, was initiated.}
   \label{fig:figprocess}
\end{figure}

%This \poy interface is not available for parallel environments. Once the program is invoked, \poy commands can be executed interactively or scripts can be submitted as when using the \emph{Interactive Console}. By default, \poy will print the output to screen (the same output that is reported in \emph{POY Output} under non-parallelized setting).

\subsection{Starting a \poy session using the \emph{Interactive Console}}

\begin{flushleft}
	\begin{minipage}[c]{0.075\textwidth}
	   	\includegraphics[width=\textwidth]{doc/figures/figlogowindows.jpg}
	\end{minipage}
	\quad
	\begin{minipage}[t]{0.89\textwidth}
		\subsubsection{Windows}
	\end{minipage}
			\begin{itemize}
                \item{Start$>$All Programs$>$POY$>$POY Interactive Console}
			\end{itemize}

	\begin{minipage}[c]{0.075\textwidth}
   		\includegraphics[width=\textwidth]{doc/figures/figlogomac.jpg}
	\end{minipage}%
	\quad
	\begin{minipage}[t]{0.89\textwidth}
	   	\subsubsection{Mac OSX}
	\end{minipage}
			\begin{itemize}
				\item {Double-click \poy application icon to start the program.}
				\item {Select \emph{Run Interactive Console} from the
				\emph{Analyses} menu.}
			\end{itemize}		

	\begin{minipage}[c]{0.075\textwidth}
   		\includegraphics[width=\textwidth]{doc/figures/figlogolinux.jpg}
	\end{minipage}
	\quad
	\begin{minipage}[t]{0.89\textwidth}
	   	\subsubsection{Linux}
	\end{minipage}
	\begin{itemize}
    		\item Add \texttt{/opt/poy4/Resources/} to your \texttt{PATH} and run
    \texttt{ncurses\_poy} from a terminal.
    	\end{itemize}
\end{flushleft}

\begin{figure}[htbp]
   \centering
   \includegraphics[width=0.7\textwidth]{doc/figures/figprelim1.jpg}
   \caption{Specifying the location of datafiles. The folder \texttt{POY-Data} is dragged from the \texttt{POY v3-4} folder directly in the Terminal window.}
   \label{fig:figprelim1}
\end{figure}

\begin{figure}[htbp]
   \centering
   \includegraphics[width=0.7\textwidth]{doc/figures/figprelim2.jpg}
   \caption{Starting \poy. At the folder containing datafiles, entering \texttt{poy} starts a \poy session.}
   \label{fig:figprelim2}
\end{figure}

\subsection{Entering commands}
Once the \poy interface is called, the cursor appears in the \emph{Interactive Console} and \poy is ready to accept commands. The interactive console does not support using the mouse and, as true for most command-line applications, the cursor can be moved using the left and right arrow keys, and the Backspace (in Windows) or Delete (in Mac) keys are used to erase individual characters to the left of the current cursor position. To eliminate the need of retyping commands anew during a \poy session, keyboard shortcuts can be used: Control-P (``previous'') and Control-N (``next'') will scroll through the commands previously entered during the session. In addition, the interactive console is equipped with the autocomplete feature: it involves \poy predicting a command, an argument, of file name that the user wants to type from the first letter(s) entered. Upon typing the first letter or part of the phrase, repeatedly pressing the TAB key scrolls through the list of command, argument, and file names that begin with that letter or phrase. Autocomplete speeds up interaction with the program.

\subsection{Browsing the output}
As more output is reported in the \emph{POY Output} window, only the most recent reports will be seen in the window. Using the Up and Down keys allows the user to scroll up and down the \emph{POY Output} window to see the welcome line, and previously printed reports and help descriptions. Pressing Up and Down keys automatically places the cursor in the lower left corner of the \emph{POY Output} window indicating that you are interacting with that window. Only 1000 lines are stored in the memory and the output that was reported before that will not be accessible by scrolling. The number of lines, however, can be modified by the user using the command \poycommand{set()}, see~\ccross{history}. If the user desires to keep the entire output or specific items in the output, a log can be created using the command \poycommand{set()}, see~\ccross{log}) or specific outputs can be redirected to files (see~\ccross{report}).

\subsection{Switching between the windows}
To return to the \emph{Interactive Console}, start typing and the cursor will automatically be placed back at the \poy prompt. When an operation is in progress (shown in the \emph{Current Job} window), the cursor stays in the upper left corner of the \emph{State of Current Search} window, and switching between the \emph{Interactive Console} and the \emph{POY Output} window is disabled. There are no user interactions in the \emph{Current Job} or \emph{State of the State of Current Search}.

\subsection{Importing data} \label{sec:import}

The basic command to input data in \poy is \commandstyle{read()}, which includes the list of files (in quotation marks and separated by commas) enclosed in parentheses. Suppose that we would like to simultaneously analyze morphological and molecular datasets, contained in separate datafiles, \texttt{morpho.ss} and \texttt{28s.fas}, respectively. In the\emph{Interactive Console} files can also be entered by dragging them into the input window and placing them after a given command which will provide both the correct path and filename.  We can issue a pair of \commandstyle{read()} commands (Figure~\ref{fig:readingexample}):
\begin{quote}
        \commandstyle{read("morpho.ss")}\\
        \commandstyle{read("28s.fas")}
\end{quote}

\begin{figure}
    \begin{center}
        \includegraphics[width=0.6\textwidth]{doc/figures/reading_example.jpg}
    \end{center}
    \caption{Importing datafiles using the \emph{Interactive Console}. Two consecutive \commandstyle{read} commands specify both the morphological datafile in Hennig86 format (\texttt{morpho.ss}), and the molecular datafile in FASTA format (\texttt{28s.fas}). Note that \poy automatically reports  in the \emph{POY Output} window the names and types of files that have been imported.}
    \label{fig:readingexample}
\end{figure}

The syntax of \commandstyle{read}, like every command in \poy, contains two elements: the name of the command (e.g. \commandstyle{read}), followed by an optional list of arguments 
separated by commas and enclosed in parentheses. Typically, the arguments of the command \commandstyle{read()} are names of datafiles, each being enclosed in double quotes (as shown in the example above). Even though there might be only one argument or none in some commands, parentheses (e.g. \poycommand{read()}) always follow the command name. An exhaustive discussion of \poy command structure and detailed descriptions of all commands with examples of their usage are provided in the \emph{POY Commands} chapter (\ref{commands}).

In order to import data by entering the names of the files, the directory containing these files must be identified using the command\commandstyle{cd}; for example \texttt{cd ("/Users/username/docs/poyfiles")}. Alternatively, the full path can be included in the argument of \commandstyle{read}: \texttt{read("/Users/username/docs/28s.fas")}.

Most of the time users are interested in importing multiple datafiles to analyze an entire dataset. In this case, multiple datafiles can be specified as arguments for a single command. For example, importing both files, \texttt{morpho.ss} and \texttt{28s.fas}, can be written more succinctly:
\commandstyle{read("morpho.ss", "28s.fas")}. This is equivalent to sequentially importing each file as shown above (Figures ~\ref{fig:readingexample} and \ref{fig:reading_example2}).

Figure~\ref{fig:readingexample} also illustrates an important feature that makes \poy different from many other phylogenetic analysis programs: every time a file is imported during a \poy session, the input data are \emph{added} to the current data in memory and \emph{do not replace them}. This allows additional analytical flexibility. For example, if only morphological data are read and trees are built based on these data alone, a subsequently imported molecular character dataset will be used in conjunction with the previously imported morphological data, despite the fact that current trees in memory were generated only from morphological data (Figure~\ref{fig:reading_example2}):

\begin{quote}
\commandstyle{read("morpho.ss")}\\
\commandstyle{build()}\\
\commandstyle{read("28s.fas")}\\
\commandstyle{rediagnose()}\\
\commandstyle{swap()}
\end{quote}

It must be noted that if the numbers of terminals differ among datafiles, only the data that correspond to the terminals used to generate the trees (from the morphological datafile in our example) are used. The rest of the character data are ignored, unless the trees are built again with the data files containing the expanded number of terminals.

Also, because \poy appends trees and data in memory, it is a good practice when starting a new analysis to empty the memory using use the command \commandstyle{wipe()}.

\begin{figure}[]
    \begin{center}
        \includegraphics[width=0.6\textwidth]{doc/figures/reading_example2.jpg}
    \end{center}
    \caption{Building trees with morphological data only but continuing analysis using combined morphological and molecular data. This example shows how we can add data to the analysis incrementally by loading files at different points in the search. First, the morphological data are imported from \texttt{morpho.ss} file using \poycommand{read()} the and trees are built based on these data. Then molecular data from the \texttt{28s.fas} file are loaded into memory in addition to previously imported morphological data. Finally, subsequent analyses, \commandstyle{rediagnose()} and \commandstyle{swap()}, are conducted using the data in memory, that is the trees based on morphological data, and both morphological and molecular character sets.}
    \label{fig:reading_example2}
\end{figure}

Valid input files include nucleotide and amino acid sequence files in many formats,
and morphological data in Hennig86 and Nexus formats. (For information on specific formats supported by \poy and other types of input files see \commandstyle{help(read)}.)

\subsection{Inspecting data}

Once a dataset (or multiple datasets) is imported, \poy automatically reports a brief description of contents for each loaded file in the \emph{POY Output} (Figure ~\ref{fig:readingexample}). However, it may be desirable to inspect the imported data in greater detail to ensure that the format and contents of the files have been interpreted correctly. This practice helps to avoid common errors, such as inconsistently spelled terminal names, which may result in bogus results, produce error messages, and aborted jobs.

The basic command for outputting information is \commandstyle{report()}. One of its arguments, \commandstyle{data}, outputs a set of tables showing the list of terminals, the number and types of characters, and the lists of terminals and characters excluded from the analysis. To produce a report of the datafiles that were used in the previous example (\texttt{morpho.ss} and \texttt{28s.fas}), we import the data and execute \commandstyle{report(data)}:
\begin{quote}
    \commandstyle{read("morpho.ss","28s.fas")}\\
    \commandstyle{report(data)}
\end{quote}
This will generate an extensive, detailed output, partial views of which are shown in Figure ~\ref{fig:reportdata}. Obviously, the entire report will not be visible in the \emph{POY Output} window. Therefore, the Up and Down arrow keys and Page Up and Page Down keys can be used to scroll.

\begin{figure}
\centering
\begin{minipage}[c]{0.52\textwidth}
   		\includegraphics[width=\textwidth]{doc/figures/report2.jpg}
\end{minipage}
\quad
\begin{minipage}[c]{0.44\textwidth}
	   	\includegraphics[width=\textwidth]{doc/figures/report3.jpg}
   	\end{minipage}
\caption{Inspecting imported data. The figure shows segments of a data report generated by \commandstyle{report(data)}. The left and right panels demonstrate a typical table output the character and terminal data respectively.}
\label{fig:reportdata}
\end{figure}

In this example, all the imported data are analyzed and, therefore, the report fields that list excluded data will appear empty. One can, however, exclude specific characters or terminals from the analysis using additional commands (see the command~\ccross{select}).

By default, \poy reports the results of executed commands to the \emph{POY Output} window. However, the same output can be redirected to a file simply by adding the name of the output file in the list of argument of the command \commandstyle{report()} \emph{before} the argument specifying the type of the requested report (in this case \commandstyle{data}). For instance, if we would like to output into the file ``data\_analyzed.txt,'' we would enter \commandstyle{report("data\_analyzed.txt", data)}.

Another useful argument of \commandstyle{report} is \commandstyle{cross\_references}. This argument displays whether character data are present or absent for each terminal in each one of the imported data files. This provides a comprehensive visual overview of missing data. Building on the previous example, such output can be generated by the following sequence of commands:
\begin{quote}
    \commandstyle{read("morpho.ss", "28s.fas")}\\
    \commandstyle{report(cross\_references)}
\end{quote}

\begin{figure}[]
    \begin{center}
        \includegraphics[width=0.6\textwidth]{doc/figures/crossref.jpg}
    \end{center}
    \caption{Visualizing missing data. The command \commandstyle{cross\_references} displays a table showing whether a given terminal (in the left column) is present (``+'') or absent (``-'') in each datafile. In this example, the data for all the the taxa listed in the \emph{POY Output} window are present in both datafiles, \texttt{morpho.ss} and \texttt{28s.fas}.}
    \label{fig:crossref}
\end{figure}

A typical output of \commandstyle{cross\_references} command is shown in Figure ~\ref{fig:crossref}.

\subsection{Building the initial trees}

The command to build trees is \commandstyle{build()} (already mentioned in Section~\ref{sec:import}). After importing \texttt{morpho.ss} and \texttt{28s.fas}, executing the command \commandstyle{build()} without specifying any arguments generates 10 trees by random addition sequence (the default setting of the command).

Many \poy commands operate under default settings when executed without arguments. To learn what the default settings are for a particular command, use either \commandstyle{help()} command with the command name of interest inserted in parentheses or consult the \emph{POY Commands} chapter (\ref{commands}).

If the user would like to specify a number of tree building replicates different from the default value of 10, an argument \commandstyle{trees} followed by a colon (``:'') and an integer specifying the number of trees must be included in the argument list of the \commandstyle{build} command: \commandstyle{build(trees:100)}. This command has a shortcut that omits the argument \commandstyle{trees}. Thus, \commandstyle{build(trees:100)} is equivalent to \commandstyle{build(100)}. As defaults, the shortcuts are fully described in Section \ref{commands}. The entire sequence of commands minimally required to import the data and build 100 trees is the following:

\begin{quote}
 	\commandstyle{read("morpho.ss","28s.fas")}\\
 	\commandstyle{build(100)}
\end{quote}

As the tree building advances, the \emph{Current Job} window displays the current status of the operation (Figure~\ref{fig:building}). This window shows how many Wagner builds have been performed out of the total number requested, the number of terminals added in the current build, the cost of the current tree (recalculated after each terminal addition), and the estimated time for the completion of all the builds. When all the trees are generated, the \emph{State of Stored Search} window displays the range of tree costs (the best and worst costs), the number of trees stored in memory, and the number of trees with the best cost (Figure~\ref{fig:building}).

\begin{figure}
\centering
\begin{minipage}[c]{0.507\textwidth}
   		\includegraphics[width=\textwidth]{doc/figures/building1.jpg}
\end{minipage}
\quad
\begin{minipage}[c]{0.453\textwidth}
	   	\includegraphics[width=\textwidth]{doc/figures/building2.jpg}
   	\end{minipage}
\caption{Generating Wagner trees. During the process of tree building (left panel), the \emph{Current Job} window displays how many builds have been performed so far (\texttt{57 of 100}), the number of terminals added in the current build (\texttt{13 of 17}), a cost of a current tree recalculated after each terminal addition (\texttt{362}), and the estimated time (in seconds) for the completion of the operation (\texttt{4 s}). Because the process is not complete, the \emph{State of Stored Search} window contains no trees. Once tree building is complete, the \emph{State of Stored Search} window displays the best (\texttt{451}) and worst (\texttt{472}) costs, the number of trees stored in memory (\texttt{100}), and the number of trees with the best cost (\texttt{2}).} 
\label{fig:building}
\end{figure}

\subsection{Performing a local search}

Now that the trees have been generated and stored in memory, a local search can be performed to refine and improve the initial trees by examining additional topologies of potentially better cost.  The command \commandstyle{swap()} implements an efficient strategy by performing SPR and TBR branch swapping alternately. As with other commands, the arguments of \commandstyle{swap()} allow the customization of the swap algorithm. In the following example, branch swapping is performed under the default settings on each of the 100 trees build in the previous step:

\begin{quote}
 	\commandstyle{read("morpho.ss","28s.fas")}\\
 	\commandstyle{build(100)}\\
	\commandstyle{swap()}
\end{quote}

Branch swapping is performed sequentially on all trees stored in memory. During swapping, the \emph{Current Job} window reports the number of the tree that is currently being analyzed, the method of branch swapping, the specific routine being currently performed, and the cost of the current tree (Figure~\ref{fig:swapping}). When the process is complete, the \emph{State of Stored Search} window displays the range of tree costs (the best and worst costs), the number of trees stored in memory, and the number of trees of the best cost (Figure~\ref{fig:swapping}). Note that the local search had reduced the costs of the initial best (from 451 to 446) and narrowed the range of tree costs.

\begin{figure}
\centering
\begin{minipage}[c]{0.49\textwidth}
   		\includegraphics[width=\textwidth]{doc/figures/swap1.jpg}
\end{minipage}
\quad
\begin{minipage}[c]{0.453\textwidth}
	   	\includegraphics[width=\textwidth]{doc/figures/swap2.jpg}
   	\end{minipage}
\caption{Performing a local search. When searching (left panel), the \emph{Current Job} window reports the number of the tree that is currently being analyzed (\texttt{73 of 100}), a method of branch swapping (\texttt{Alternate}), a function being currently performed (\texttt{SPR search}), and a cost of the current tree (\texttt{456}). When the searching is finished (right panel), the \emph{State of Stored Search} window displays the best (\texttt{446}) and worst (\texttt{463}) costs, the number of trees stored in memory (\texttt{100}), and the number of trees of the best cost cost (\texttt{9}) recovered from independent tree builds. Note these trees may not necessarily have unique topologies.} 
\label{fig:swapping}
\end{figure}

Using different combinations of \commandstyle{swap()} arguments allows designing a  large number of search strategies of different levels of complexity. Some simple options allow the choice between SPR and TBR. More complex strategies allow keeping a specific number of best trees per single initial tree (generated during the building step). For example, the command \commandstyle{swap(trees:10)} will keep up to 10 best trees generated during branch swapping on a single initial tree. Consequently, if 100 trees were built initially, this command will produce up to 1,000 trees. The argument \commandstyle{threshold} allows the retention of suboptimal trees within a specified percent of cost difference from the current best tree. For example, \commandstyle{swap(trees:20, threshold:10)} will execute a swap considering trees within a ten percent cost difference of the current best tree and retain the 20 minimal length swapped trees for each initial build. Other options provide the means to sample trees as they are evaluated, timeout after certain number of seconds, transform the cost regime, and other functions in conjunction with many \poy commands.

\subsection{Selecting trees}

Having performed the basic steps of importing character data, building initial trees, and conducting a simple local search, we obtained a set of local-optima trees in memory. Most of the time, a user would like to select only those trees that are optimal and topologically unique. The default setting of the \commandstyle{select()} does exactly that. Adding \commandstyle{select()} to our example of command sequence for the basic analysis 
\begin{quote}
 	\commandstyle{read("morpho.ss","28s.fas")}\\
 	\commandstyle{build(100)}\\
	\commandstyle{swap()}\\
	\commandstyle{select()}
\end{quote}
selects only unique trees of best cost. The remaining trees are deleted from memory. The \emph{State of Stored Search} window reports the number and the cost of the best tree(s) (Figure~\ref{fig:select}).

\begin{figure}[]
    \begin{center}
        \includegraphics[width=0.6\textwidth]{doc/figures/select.jpg}
    \end{center}
    \caption{Selecting unique best trees. Executing \commandstyle{select()} keeps only unique trees of best cost. The \emph{State of Stored Search} window reports that there is only one unique tree of best cost (\texttt{446}).}
    \label{fig:select}
\end{figure}

\commandstyle{select()} is another multifunctional command the arguments of which are also used to select (include or exclude) specific terminals, characters, and trees.)

Comparing the output reported in the \emph{State of Stored Search} before (Figure~\ref{fig:swapping}) and after (Figure~\ref{fig:select}) executing \commandstyle{select()} shows that swapping on 9 of 100 initial trees produced the trees of best cost (\texttt{446}), but these trees are identical, because only one was retained when filtered using \commandstyle{select()}.

\subsection{Visualizing the results}

There are several ways to visualize results. The command
\commandstyle{report("my\_first\_tree", graphtrees)} outputs a cladogram in PDF format (Figure~\ref{fig:trees}), which can be displayed, edited, and printed using graphics software (such as Adobe Illustrator or Corel Draw). \poy also appends the ``ps'' extension when generating graphic output to a file. A quick way to see the tree(s) on screen is to use the command \commandstyle{report(asciitrees)} that draws a cladogram in the \emph{POY Output} window (Figure~\ref{fig:trees}). The ascii tree can also be reported to a file, if the output file name is specified (in parentheses and separated from the argument \commandstyle{asciitrees} by a comma).

\commandstyle{report("my\_first\_trees.txt", trees)} reports the trees in memory in parenthetical notation to the file \texttt{my\_first\_trees} that can be imported in other programs. Other supported tree output formats include Newick and Hennig86. \commandstyle{report()} can also generate consensus trees in the graphical and parenthetical formats when appropriate arguments are specified (for example, \commandstyle{report("strict\_consensus", graphconsensus)}).

\begin{figure}
\centering
\begin{minipage}[c]{0.45\textwidth}
   		\includegraphics[width=\textwidth]{doc/figures/asciitree.jpg}
\end{minipage}
\quad
\begin{minipage}[c]{0.5\textwidth}
	   	\includegraphics[width=\textwidth]{doc/figures/pstree.jpg}
   	\end{minipage}
\caption{Visualizing trees. An ascii tree (left) is generated using the command
\poycommand{report(asciitrees)}. The same tree is reported to a file in a PDF format (right) using \commandstyle{report("my\_first\_tree", graphtrees)}. Note that both representations of trees  are preceded by their costs.}
\label{fig:trees}
\end{figure}

\subsection{Interrupting a process}
To interrupt a process, press Control-C. By default, an error, \texttt{Error:}\\ \texttt{Interrupted}, is reported in the \emph{POY Output} window. The program does not close, however, and a new command can be entered. Interrupting the analysis cancels the execution of the last command requested by the user and restores the data and trees in memory before that last command. For example, the following two session are equivalent: (1) \texttt{read(``a'') <ENTER>} and (2) \texttt{read(``a'') <ENTER>} \texttt{read(``b'') read(``c'') <ENTER>} \texttt{<CONTROL-C>}.

\subsection{Reporting errors}
If there is an error pertaining to wrong syntax (such as a typo in a command name), \poy will indicate the location of the error by underlining the problematic part of the input with ``\texttt{\^}'' in the \emph{Interactive Console} (Figure~\ref{fig:errors}). The description of the corresponding command, its syntax, and examples of its usage from the help file are automatically printed in the \emph{POY Output} window. As noted above, the Up and Down keys can be used to scroll through the output and determine the source of the error. Certain types of errors are reported explicitly (Figure~\ref{fig:errors}).

\begin{figure}
\centering
\begin{minipage}[c]{0.48\textwidth}
   		\includegraphics[width=\textwidth]{doc/figures/figerror1.jpg}
\end{minipage}%
\quad
\begin{minipage}[c]{0.48\textwidth}
	   	\includegraphics[width=\textwidth]{doc/figures/figerror2.jpg}
   	\end{minipage}
	
\caption{Displaying errors. \poy displays error messages in several ways. In the example in the left panel, the command \commandstyle{build} was entered without parentheses, which is required for a  valid \poy command syntax; the exact place of the error is marked by ``\texttt{\^}'', in this case  following the \commandstyle{build} commands. Examples of the proper usage of the command are automatically displayed in the \emph{POY Output}. In other cases (right panel), error messages are explicitly reported in the \emph{POY Output} window. The first and second error messages indicate that the datafile \texttt{SSU.seq} is not present, which could have been caused either by a mistake in the name of the file, missing file, or the location of the file in a directory, other than the one specified prior to starting the \poy session. The third error message indicates that the valid syntax of \commandstyle{exit} requires the parentheses following the command name (also shown by ``\texttt{\^}'' in  the \emph{Interactive Console}).}
\label{fig:errors}
\end{figure}

\subsection{Exiting}
To finish a \poy session, enter the command \commandstyle{exit()} (Figure~\ref{fig:exithelp}) or \commandstyle{quit()}. This will close the \poy interface and resume the Terminal window (Mac OSX) or the Command Prompt window (Windows).

\begin{figure}[]
    \begin{center}
        \includegraphics[width=0.5\textwidth]{doc/figures/exithelp.jpg}
    \end{center}
    \caption{Exiting \poy}
    \label{fig:exithelp}
\end{figure}

\section{Creating and running \poy scripts}

So far, we have communicated with \poy interactively through the \emph{Graphical User Interface} or by executing commands from the \emph{Interactive Console}. Another way of conducting an analysis is to run a \emph{script}, a simple-text file containing a list of commands to be performed (Figure~\ref{fig:script}). 

Running analyses using scripts has many advantages: not only does it allow for the entire analysis to proceed from the beginning to the end at one click of a button, but it also provides means to examine the logical dependency of the commands and optimize memory consumption (see the description of \poyargument{script\_analysis} argument of the command \poycommand{report} in the \emph{POY Commands} chapter). Submitting jobs using scripts may produce results faster because \poy automatically optimizes the workflow of the entire analysis by taking into account the functional relationships among various tasks and efficiently distributing the jobs and resources (such as memory and multiple processors).

Another advantage of using scripts is that they may contain comments that are ignored by \poy but can be helpful to describe the contents of the files and provide other annotations. The comments are enclosed in parenthesis \emph{and} asterisks. For example, \texttt{(* this is a comment *)}. Comments can be of any length and span multiple lines. Comments can also be entered interactively from the \emph{Interactive Console}.

Obviously, using scripts requires the user to design the workflow of the process prior to conducting the analysis. \poy scripts can be created and saved using the \emph{Script Editor} window of \poy interface or any conventional text editor (such as TextPad, TextWrangler, BBEdit, Emacs, or NotePad).

\poy scripts are extremely useful in cases when operations may take a long time to complete, eliminating the need to wait for a part of the analysis to finish in order to proceed to the next step.

There are two ways to import and run a script:
\begin{itemize}
    \item using the \emph{POY Launcher} in the \emph{Graphical User Interface};
    \item using the command \commandstyle{run()} of the \emph{Interactive Console}; for example, \texttt{run("script.txt")}, where \texttt{script.txt} is the name of the file containing the script.
\end{itemize}

It it critical to include the command \commandstyle{exit()} at the end of the script. Otherwise \poy will be waiting for further instructions to be entered after executing the script's contents.

\begin{figure}
\centering
\begin{minipage}[c]{0.42\textwidth}
   		\includegraphics[width=\textwidth]{doc/figures/commandlist.jpg}
\end{minipage}
\quad
\begin{minipage}[c]{0.53\textwidth}
	   	\includegraphics[width=\textwidth]{doc/figures/script.jpg}
   	\end{minipage}
\caption{Using \poy scripts. The list of commands executed interactively using the \emph{Interactive Console} (left) and a script containing the same list of commands (right). Note, that the header of the script is a comment, enclosed in ``(* *)'', that is ignored by \poy. Also note, that commands can either be listed in a row or in a column (compare \commandstyle{build()} and \commandstyle{swap()} in the console and in the script) and different arguments of the same command can either be specified separately or combined in a single argument list (compare \commandstyle{report()} in the console and in the script). (Both conventions are valid for interactive command submission and for scripts.)}
\label{fig:script}
\end{figure}

\section{Obtaining help} \label{sec:help}
Instructions to run \poy, command descriptions, and the theory behind \poy can be obtained from a variety of sources.
\begin{description}
\item[POY \buildnumber Program Documentation] (this manual) is a comprehensive and detailed manual on all the aspects of using \poy, from installation to output and visualization of results. Included are \emph{Quick Start}, \poy command reference, practical guides and tutorials that make the program immediately accessible for beginners and provide in-depth information for experienced users. The documentation in PDF format can be accessed from the \emph{Help} menu of the graphical user interface or downloaded separately from \poy web site at
\begin{center}
\texttt{http://research.amnh.org/scicomp/projects/poy.php}
\end{center}
\item[POY] interactive help can be obtained by entering \commandstyle{help()} at the \poy interactive console. To obtain help on a particular command, the name of the command must be specified in the parentheses following \commandstyle{help()}. For example, to learn about the command \commandstyle{exit}, type \commandstyle{help(exit)}. (Figure~\ref{fig:exithelp}.)
\item[POY4 Mail Group] is an Internet-based forum for discussing all issues related to \poy and provides the best way to communicate with \poy developers on specific issues (see \emph{WWW resources} below). The website is located at \texttt{http://groups.google.com/group/poy4}.
\item[POY Book] (Wheeler et al., 2006 \emph{Dynamic Homology and Phylogenetic Systematics: A Unified Approach Using POY \cite{wheeleretal2006}}) provides a review of the theory behind \poy, and contains formal descriptions of many algorithms implemented in the program and the descriptions of commands of the earlier version, \texttt{POY3}.
\begin{figure}[htbp]
   \centering
   \includegraphics[width=0.23\textwidth]{doc/figures/figpoybook.jpg}
   \caption{The POY Book.}
   \label{fig:figprocess}
\end{figure}
\end{description}

\section{WWW resources}
\poy is an ongoing project and new versions are being continuously developed to include new procedures, improve performance, and eliminate reported bugs. Therefore, it is imperative to keep up with the program's development and check regularly for updates. There are several Internet-based resources that offer this information.

\begin{description}
\item[POY4 Web Site] has downloadable compressed files of \poy binaries, source code, and documentation in PDF format. It also provides a links to the \emph{POY Mail Group}. The website is hosted by AMNH Computational Sciences at 
\begin{center}
\texttt{http://research.amnh.org/scicomp/projects/poy.php}
\end{center}

\item[POY4 source code repository] contains has downloadable \poy source code.  The site is powered by Google at 
\begin{center}
\texttt{http://code.google.com/p/poy4/source}
\end{center}

\item[POY4 Mail Group] informs registered users via email of new developments, such as new versions and updates. It also provides  additional resources for obtaining help and a way for reporting bugs and other problems with \poy and its documentation. In addition, it allows users to receive and respond to each other's questions thus providing an open forum to  discuss the methods and applications of \poy. The users who choose not to register, have access to the archives of the postings but will not be able to either submit or receive emails from other users and \poy developers. The \emph{POY4 Mail Group} is hosted  by Google at
	\begin{center}
	\texttt{http://groups.google.com/group/poy4}
	\end{center}
	
\end{description}


\chapter{\poy Commands}\label{commands}

\section{\poy command structure}

\subsection{Brief description} \label{commands}

\poy interprets and executes \emph{scripts} issued by the end user.  These can
come from the command line in the \emph{Interactive Console} of the program, or from an
input file. A script is a list of \emph{commands}, separated by any number of
whitespace characters (spaces, tabs, or newlines). Each command consists of a
name  in lower case (\poylident), followed by a list of arguments separated 
by commas and enclosed in parentheses. Most of the arguments are optional, in
which case \poy has default values.

In \poy, we recognize four types of command arguments: \emph{primitive values},
\emph{labeled values}, \emph{commands}, and \emph{lists of arguments}.

\paragraph{Primitive values} can be either an integer (\poyint), a real number
(\poyfloat), a string (\poystring), or a boolean (\poybool).

\paragraph{Labeled values} are a lowercase identifier (which are referred to as
\emph{label}), and an argument, separated by the colon character. ``:''.

\paragraph{List of arguments} are several arguments enclosed in parenthesis and
separated by commas, ``,''.

\paragraph{Commands} are standard commands that can affect the behavior of
another command when included in its list of arguments.

\paragraph{}Thus, certain commands can function as arguments of other commands. Moreover,
some commands share arguments. Although such compositive use of commands
might seem complex, this structure provides much more intuitive
control and greater flexibility. The fact that the same logical operation that functions
in different contexts maintains
the same name (typically suggestive of its function), substantially reduces the number of
commands without limiting the number of operations. Using a linguistic analogy,
\poy specifies a large number of procedures by a more complex grammar (specific
combinations of commands and arguments), rather than by increasing the vocabulary
(the number of specific commands and arguments). For example, the command
\poycommand{swap} specifies the method of branch swapping. This command is
used to conduct a local search on a set of trees. In addition,
\poycommand{swap} functions as an argument for \poycommand{calculate\_support}
to specify the branch swapping method used in each pseudoreplicate during Jackknife or
Bootstrap resampling. \poycommand{swap} can also be used to set the parameters for
local tree search based on perturbed (resampled or partly weighted) data as an argument
of the command \poycommand{perturb}. Therefore, to take the maximum advantage of
\poy functionality, it is essential to get acquainted with the grammar of  \poy.

\subsection{Grammar specification}

The following is the formal specification of the valid grammar of a script in \poy:

\begin{verbatim}
script: = | command
        | command script

command: = LIDENT "(" argument list ")"

argument list: = |
            | arguments

arguments: = |
            | argument
            | argument "," arguments

argument: = | primitive
            | LIDENT
            | LIDENT ":" argument
            | command
            | "(" argument list ")"

primitive: = | INTEGER
            | FLOAT
            | BOOLEAN
            | STRING

LIDENT: = [a-z_][a-zA-Z0-9_]*

INTEGER: = [0-9]+

FLOAT: = | INTEGER
        | [0-9]+ "." [0-9]*

STRING: = """ [^"]* """

\end{verbatim}

The following examples graphically show a typical structure of valid \poy commands
formally defined above. The Figure \ref{simplecommand} illustrates
the syntax of the command \poycommand{swap}. The name of the
command, \poycommand{swap}, is followed by a list of two arguments,
\poyargument{tbr} and \poyargument{trees:2}, enclosed in parentheses
and separated by a comma. Note that \poyargument{trees:2} is a labeled-value
argument that contains a label (\texttt{trees}) and a value (\texttt{2})
separated by a colon.

\begin{figure}[htbp]
   \centering
   \includegraphics[width=0.34\textwidth]{doc/figures/fig-poycommand1.jpg}
   \caption{The structure of a simple \poy command. The entire command (highlighted
   in blue), consists of  a command name followed by a list of arguments (enclosed in red box).
   See text for details.}
   \label{simplecommand}
\end{figure}

Figure \ref{compositecommand} shows a more complex command structure, using the command \poycommand{perturb} as an example. This is a compound command because the list of its arguments contains another command, \poycommand{swap}. This means that executing \poycommand{perturb} performs a set of specified operations that contains a nested set of operations specified by \poycommand{swap}. Note also, that in contrast to the first labeled-values argument \poyargument{iterations}, the second labeled-values argument \poyargument{ratchet} has multiple values (a float and an integer). When multiple values are specified, they must be enclosed in parentheses and separated by a comma. The third argument is a command (\poycommand{swap}), therefore it is syntactically distinguished from other arguments, labeled and unlabeled alike, by having parentheses following the command name. It must be emphasized that the parentheses always follow the command name even if no arguments are specified. (If no arguments are specified, a command is executed under its default settings.)

\begin{figure}[htbp]
   \centering
   \includegraphics[width=0.8\textwidth]{doc/figures/fig-poycommand2.jpg}
   \caption{A structure of a compound \poy command. Note that the list of arguments
   (enclosed in red box) includes a command (highlighted in blue). Also, note that
   \poyargument{ratchet} accepts multiple values, a float and an integer, that are inclosed in
   parentheses and separated by a comma. See text for details.}
   \label{compositecommand}
\end{figure}

\section{Notation}

Some arguments are obligatory, whereas others are not; some commands accept an
empty list of arguments, but others do not; some argument labels have
obligatory values, some have optional values. In the descriptions of
\poy commands below, the elements of \poy grammar are defined in
the text using the following conventions:

\begin{itemize}
    \item A command that could be included in a \poy script (that is can be entered in the
    	interactive console or included in an input file) is shown in \poycommand{terminal type}.
    \item Optional items are inclosed in \poycommand{[square brackets]}.
    \item Primitive values are shown in \poycommand{UPPERCASE}.
\end{itemize}

Each command description entry contains the following sections:

\begin{itemize}
    \item The name of the command.
    \item A brief description of the command's function.
    \item Cross references to related commands.
    \item The valid syntax for the command.
    \item The list of descriptions of valid arguments.
    \item Description of default settings.
    \item Examples of the command's usage.
\end{itemize}

\begin{statement}
    Default syntax. The default syntax for all commands is the same: it includes the
    command name followed by empty parentheses. For example,
    \poycommand{swap()}. The descriptions of default settings, however,
    include the entire argument list for the obvious reason of showing what is
    included in the omitted argument list.
\end{statement}

\begin{statement} \label{commandorder}
    Command order. The effect of the order of arguments in a command depends on the context. 
    If arguments are not logically interconnected, their order is not
    important. For example, the commands \poycommand{build(10,randomized)} and
    \poycommand{build(randomized,10)} are equivalent. However, executing the
    commands \poycommand{transform(tcm:(1,1),gap\_opening:4)} and
    \poycommand{transform(gap\_opening:4,tcm:(1,1))} will produce different results
    because \poycommand{gap\_opening} \emph{modifies} the values set by
    \poyargument{tcm}, while \poyargument{tcm} \emph{overrides} the values set by \poyargument{gap\_opening}.
\end{statement} 

\begin{statement}
    Output files. When an output file is specified, the file name (in double quotes and
    followed by a comma) must precede the argument.
\end{statement}

\begin{statement}
    Certain command arguments are mainly useful to \poy developers, and
    those arguments are preceded by an underscore.
\end{statement}

\section{Command reference}
\begin{command}{build}{buildcommand}

    \syntax{\obligatory{(\optional{argument list})}}
	
 	\begin{poydescription}
        Builds Wagner trees~\cite{farris1970}. Building multiple trees with a randomized addition of terminals allows for
        the evaluation of many more possible tree topologies and generates a
        diversity of trees for subsequent analysis. The arguments of the command
        \poycommand{build} specify the number of trees to be generated and
        the order in which terminals are added during a singe tree building procedure. During tree
        building, \poy reports in the \emph{Current Job} window of the ncurses interface
        which of the terminal addition strategies is currently used.
        

        By default \poy replaces the trees stored in memory with those generated
        in a subsequent build. For example, executing \poycommand{build(10)}
        followed by \poycommand{build(20)} will replace the 10 trees generated
        during the first build with 20 new trees. However, it might be desirable
        (for example, if computer memory were limited) to generate a large number of trees by
        appending trees from multiple separate builds. To keep trees from consecutive
        builds, a tree output file must be specified using the command ~\ccross{report} that must 
        precede the \poycommand{build} command. This will produce a file
        containing the trees appended from all builds. If the same file name is used for reporting
         trees for other analysis, the new trees are going to be appended. Alternatively, trees from different
        builds can be redirected to separate files if different file names are specified.
        
        The command \poycommand{build} is also used as
        an argument for the command \poycommand{calculate\_support}.
   	\end{poydescription}

	\begin{arguments}

        \argumentdefinition{all}{}
            {Turns off all preference strategies for adding branches
            and simply tries all possible addition positions for
            all terminals.}
            {allbuild}

        \argumentdefinition{as\_is}{} 
            {Indicates that in one of the trees to be built, the terminals are
            added in the order in which they appear in the imported datafiles,
            and all others are built using a random addition sequence.}
            {asis}

        \argumentdefinition{branch\_and\_bound}{\optional{\poyfloat}}
            {Calculates the exact solution using the Branch and Bound algorithm
            ~\cite{hendy1982}. By default only one optimal tree is kept but
            the number of optimal trees to be retained can be specified by the
            argument \poyargument{trees}. The optional float value specifies the
            bound (either tree cost or likelihood score).}
            {branchandbound} 

        \argumentdefinition{constraint}{\optional{\poystring}}
            {Builds trees using the set of constraints specified by the consensus
            tree input file. If no input file is provided, the constraint is calculated as
            the strict consensus of the trees in memory. Every tree built
            using this method is subjected to the same randomization as wagner
            builds within each constraint.}
            {buildconstraint}

        \argumentdefinition{lookahead}{\obligatory{\poyint}}
            {The number of trees that can be kept at each build step. If the
            \poyargument{lookahead} argument 
            specifies a number $n$, and the best tree found has cost $c$, then the best $n$
            trees with cost at most $c + threshold$ as specified by
            the \nccross{threshold}{buildthreshold} command are held for the
            next build step. If no \poyargument{threshold} command is specified,
            then it is set to $0$.}
            {lookahead}

        \argumentdefinition{nj}{}
            {Creates a tree using the Neighbor Joining algorithm. If more than
            one tree is requested, all the trees will be the same (the algorithm
            implementation is deterministic).}{}

        \argumentdefinition{random}{}
            {Generates a tree at random.  All possible trees have equal
            probability.}
            {buildrandom}

		\argumentdefinition{randomized}{} 
            {Indicates that terminals are added in random order on every Wagner
            tree built. This is a default tree-building strategy.}
            {randomized}

        \argumentdefinition{threshold}{\obligatory{\poyfloat}}
            {The numerical value specifies the extra cost over the current best
            tree that makes another tree acceptable for the lookahead list. This 
            parameter is only useful if~\nccross{lookahead}{lookahead} is used.}
            {buildthreshold}
            
        \argumentdefinition{trees}{\obligatory{\poyint}}
            {The integer value specifies the number of independent, individual
            Wagner tree builds. The label \poyargument{trees} is optional: it is
            sufficient to specify only the integer value. Therefore, \poycommand{build(5)} is
            equivalent to \poycommand{build(trees:5)}.  Note that  \poyargument{trees} is
            also used as an argument of the command \nccross{swap}{swapcommand}
            but with different meaning.
            
            The value \texttt{0} generates no trees but it \emph{retains} all trees in memory.
            This is useful, for example, in the \ccross{bremer} support calculation,
            where instead of generating new trees per each node, the searches are
            performed on the trees in the neighborhood of the current trees in memory.}
            {treesbuild}

        \argumentdefinition{INTEGER}{}
            {The integer argument specifies the number of independent, individual
            Wagner tree builds. This is a shortcut of the argument \poyargument{trees}.}
            {}

        \argumentdefinition{of\_file}{\obligatory{\poystring}}
            {Imports tree file included in the file path of the argument. This command is
            useful for importing starting trees for calculating \ccross{bremer} support.
            In other contexts the command \ccross{read} can be used with the same effect.}
            {offile}

        \argumentdefinition{STRING}{}
            {This is a shortcut of the argument \poyargument{of\_file}.}
            {}

   \end{arguments}
      
   \poydefaults{trees:10, randomized, lookahead:1, threshold:0}
       {By default, \poy will build 10 trees using a random addition sequence for
       each of them.}

	\begin{poyexamples}
		\poyexample{build(20)}
            {Builds 20 Wagner trees randomizing the order of terminal
            addition (note that because the argument \poyargument{randomized} is specified by default, 
            it can be omitted).}

		\poyexample{build(trees:20, randomized)}
            {A more verbose version of the previous example. By default a build
            is randomized, but in this case the addition sequence is explicitly
            set. For the total number of trees, rather than simply specifying \texttt{20},
            the label \texttt{trees} is used. The verbose version might be desirable
            to improve the readability of the script.}

		\poyexample{build(15, as\_is)}
            {Builds the first Wagner tree using the order of terminals in the first
            imported datafile and generates the remaining
            14 trees using random addition sequences.}
            
            	\poyexample{build(branch\_and\_bound, trees:5)}
            {Builds trees using branch and bound method and keeps up to
            5 optimal trees in memory.}
            
	\end{poyexamples}

\end{command}

\begin{command}{calculate\_support}{calculatesupport}

	\syntax{\obligatory{(\optional{argument list})}}

	\begin{poydescription} 
            Calculates the requested support values. \poy implements support
            estimation based on resampling methods (Jackknife~\cite{Farrisetal1996} 
            and Bootstrap~\cite{Felsenstein1985}) and Bremer support~\cite{Bremer1988, Kallersjoetal1992}. 
            The Jackknife and Bootstrap support values are computed as
            frequencies of clades recovered in strict consensus trees built in each resampling
            iteration. The consensus trees are based on best trees recovered in each replicate
            with zero-length branches collapsed.
            All the arguments of \poycommand{calculate\_support} command are optional and
            their order is arbitrary.  For examples of scripts implementing support measures see 
            tutorials \texttt{4.3} and \texttt{4.4}. 
            
The \poycommand{calculate\_support} command does not output support values by default. The output of
            support values must be requested using the command~\ccross{report}. 
            This is particularly important for
            Jackknife and Bootstrap support values, as these sampling techniques
            do not require the presence of trees in memory. Therefore, it is
            possible to perform the sampling for support values \emph{before}
            the tree of interest has been found.
            
            \begin{statement}
                In the context of dynamic
                homology, the characters being sampled during pseudoreplicates
                are entire sequence fragments, not individual nucleotides.
                Consequently, the bootstrap and jackknife support values
                calculated for dynamic characters are not directly comparable to
                those calculated based on static character matrices. If it is
                desired to perform character sampling at the level of
                individual nucleotides, the dynamic characters must be
                transformed into static characters using \poyargument{static\_approx}
                argument of the command~\ccross{transform}
                prior to executing \poycommand{calculate\_support}.
                Alternatively, an output file in the Hennig86 format can be
                generated based on an implied alignment
                using~\ccross{phastwinclad} that can subsequently be analyzed
                using other programs.
                                
                It is important to remember that the local optimum for the dynamic
                homology characters can differ from that for the static homology characters
                based on the same sequence data. Therefore, it is recommended to perform an extra round of swapping on the
                 transformed data to reach the local maximum for the static
                 homology characters prior to calculating support values.
            \end{statement}
            
            \end{poydescription}

	\begin{arguments}
		\begin{argumentgroup}{Support calculation methods}
            {The following commands allow selecting among several methods for
            calculating support.} 

			\argumentdefinition{bremer}{}
                {Calculates Bremer support~\cite{Bremer1988, Kallersjoetal1992}
                for each tree in memory by performing independent constrained searches for each
                node. The parameters for the searches can be modified using arguments
                described under \emph{Search strategy}.} 
                {}
\begin{statement}
  	  The placement of the root affects calculation of Bremer support values.
	  Therefore, it is critical to specify the root prior to executing
	  \poycommand{calculate\_support}. See the description of the
	  command \nccross{set}{root} on how to specify the root.
	\end{statement}

			\argumentdefinition{bootstrap}{\optional{\poyint}}
                {Calculates Bootstrap support~\cite{Felsenstein1985}. 
                The integer value specifies
                the number of resampling iterations (pseudoreplicates). If the value
                is omitted, 5 pseudoreplicates are performed by default.} 
                {}

			\argumentdefinition{jackknife}{\optional{([argument list])}}
                {Calculates Jackknife support~\cite{Farrisetal1996} using the 
                sampling parameters specified by the arguments. The arguments of
                \poyargument{jackknife} are optional and their order is arbitrary. If
                both values are omitted, the default values of each argument is used.}
                {}
                
                \begin{description}
                    \argumentdefinition{remove}{\obligatory{\poyfloat}}
                        {The value of the argument \poyargument{remove} specifies the
                        percentage of characters being deleted during a pseudoreplicate. The
                        default of \poyargument{remove} is \texttt{36} percent.}
                        {}
                     \argumentdefinition{resample}{\obligatory{\poyint}}
                        {The value of the argument \poyargument{resample} specifies the
                        number of resampling pseudoreplicates. The default of \\
                        \poyargument{resample} is \texttt{5}.}
                        {}
               \end{description}  
		\end{argumentgroup}

        \begin{argumentgroup}{Search strategy}
            {The calculation of the support values requires a local search,
            that is performed under the default settings unless the values
            of the following arguments are specified.}
		 
	     \argumentdefinition{build}{}
             {For calculating Bremer support, the integer value of
             \poyargument{build} specifies the number of independent
             Wagner tree builds per node. The integer value \texttt{0}
             (\texttt{build:0}) specifies that Bremer support values are
             calculated on the starting trees currently
             in memory, rather than on newly generated trees.
             The initial trees for calculating Bremer support
             can also be imported using the argument \poyargument{of\_file}
             of the command ~\nccross{build}{buildcommand}.
             
             For calculating Jackknife
             and Bootstrap supports, it specifies the number of
             Wagner tree builds per pseudoreplicate.  Single best trees from all
             psudoreplicates are used to calculate the support values. If
             multiple best trees are recovered in a pseudoreplicate, one 
             is selected. If \poyargument{build} is
             omitted from the argument list of \poycommand{calculate\_support},
             a single random addition Wagner tree per
             pseudoreplicate is built by default. This is equivalent to 
             \poycommand{build(trees:1, randomized)}. See
             \nccross{build}{buildcommand} for a detailed discussion of
             arguments of the command \poycommand{build}.}
             {buildarg}

        \argumentdefinition{swap}{}
            {Specifies the method and parameters for local tree search. If searching
            parameters are not specified, the search is performed under
            the default settings of ~\nccross{swap}{swapcommand}.} 
            {swaparg}
	     
        		\end{argumentgroup}

	\end{arguments}

    \poydefaults{bremer, build(trees:1, randomized), \\ swap(trees:1)}
    {By default \poy will calculate the bremer support for each tree in memory node by node.
    However, if no trees stored in memory, executing the command
    \poycommand{calculate\_support()} does not have any effect.}

	\begin{poyexamples} 

        \poyexample{calculate\_support(bremer)}
            {Calculates Bremer support values by performing
            independent searches for every node for every tree in memory. This is equivalent to executing \poycommand{calculate\_support()} (the default setting.)}
         
         \poyexample{calculate\_support(bremer, build(trees:0), swap(trees:2))}
            {Calculates Bremer support values by performing swapping on 
            each tree in memory for every node and keeping up to two
            best trees per search round.}
          
          \poyexample{calculate\_support(bremer, build(of\_file:"new\_trees"), \\
          swap(tbr, trees:2))}
            {Calculates Bremer support values by performing TBR swapping on 
            each tree in the file \texttt{new\_trees} located in the current
            working directory for every node and keeping up to two
            best trees per search round.}  
            
         \poyexample{calculate\_support(bootstrap)}
         {Calculates Bootstrap support values under default settings. This command
         is equivalent to \poycommand{calculate\_support(bootstrap:5, \\ build(trees:1,
         randomized), swap(trees:1))}.}
	
        \poyexample{calculate\_support(bootstrap:100, build(trees:5), \\
            swap(trees:1))}
            {Calculates Bootstrap support values performing one random resampling with
            replacement, followed by 5 Wagner tree builds (by random addition sequence)
            and swapping these trees under the default settings of the command 
            \poyargument{swap}, and keeping one minimum-cost tree. The procedure
            is repeated 100 times.}
        
        \poyexample{calculate\_support(jackknife:(resample:1000), 
            build(), \\ swap(tbr, trees:5))}
            {Calculates Jackknife support values randomly removing 36 percent of the
            characters (the default of \poycommand{jackknife}), building 10
            Wagner trees by random addition sequence (the default of
            \poycommand{build}), swapping these trees using \poyargument{tbr},
            and keeping up to 5 minimum-cost tree in the
            final swap per swap (totaling up to 50 stored trees per replicate). 
            The procedure is repeated 1000 times.}

	\end{poyexamples}
            
	\begin{poyalso}
		\cross{report}
        \ncross{supports}{supports}
        \ncross{graphsupports}{graphsupports}
	\end{poyalso}

\end{command}

\begin{command}{clear\_memory}{clearmemory}

	\syntax{\obligatory{(\optional{argument list})}}
	
	\begin{poydescription}
            Frees unused memory. Rarely needed, this is a useful command when the
            resources of the computer are limited. The arguments are optional and
            their order is arbitrary.
	\end{poydescription}
	
	\begin{arguments}
		\argumentdefinition{m}{}
            {Includes the alignment matrices in the freed memory.} 
            {clearmemoryalign}

		\argumentdefinition{s}{}
            {Includes the unused pool of sequences in the freed memory.}
            {clearmemoryseq}
	\end{arguments}
	
    \poydefaults{}{By default \poy clears all memory
    \emph{except} for the pool of unused sequences and the matrices used for the
    alignments.}
	
	\begin{poyexamples}
		\poyexample{clear\_memory(s)}
            {This command frees memory including all alignment matrices but keeping
            unused pool of sequences.}
	\end{poyexamples}

	\begin{poyalso}
		\cross{wipe}
	\end{poyalso}
	
\end{command}

\begin{command}{cd}{cd}

	\syntax{\obligatory{(\poystring)}}

	\begin{poydescription}
            Changes the working directory of the program. This command is useful
            when datafiles are contained in different directories. It also
            eliminates the need to navigate into the working directory before
            beginning a \poy session. To display the path of the current
            directory, use the command~\ccross{pwd}.
	\end{poydescription}

	\begin{arguments}
		\argumentdefinition{STRING}{}
            {The value specifies a path to a directory.}
            {}
	\end{arguments}
	
	\begin{poyexamples}

		\poyexample{cd ("/Users/username/docs/poyfiles")}
            {Changes the current directory to the directory in a Mac environment (when using a 
            PC, the forward slashes will be replaced with backslashes.)}
            

    \end{poyexamples}

    \begin{poyalso}
        \cross{pwd}
    \end{poyalso}

\end{command}

\begin{command}{echo}{}

    \syntax{\obligatory{(\poystring, output class)}} 
	
	\begin{poydescription} 
         Prints the content of the string argument into a specified type of output.
         Several types of output are generated by \poy  which are specified by the
         ``output class'' of arguments (see below). If no output-class arguments are
         specified, the command does not generate any output.
	\end{poydescription}

    \begin{arguments}
           \begin{argumentgroup}{Output class}
        \argumentdefinition{error}{}
            {Outputs the specified string as an error message (\texttt{stderr} in the
            flat interface).}
            {errorecho}

        \argumentdefinition{info}{}
            {Outputs the specified string as an information message (\texttt{stderr} in the
            flat interface).}
            {}

        \argumentdefinition{output}{\optional{\poystring}}
            {Reports a specified string (\texttt{stdout} in the flat interface) to screen or file, 
            if the filename string (enclosed in parentheses) is specified following \texttt{output} 
            and separated from it by a colon, ``:''.}
            {}
           \end{argumentgroup}
    \end{arguments}

	\begin{poyexamples}

        \poyexample{echo("Building with indel cost 1", info)}
            {Prints to the output window in the ncurses interface and to the
            standard error in the flat interface the message \texttt
{Building with indel cost 1}.}

        \poyexample{echo("Final trees", output:"trees.txt")}
            {Prints the string \texttt{Final trees} to the file \texttt{trees.txt}.}

        \poyexample{echo("Initial trees", output)}
            {Prints the string \texttt{Initial trees} to the output window in the
            ncurses interface, and to the standard output (\texttt{stdout} in the flat
            interface).}
    \end{poyexamples}

	\begin{poyalso}
		\cross{report}
	\end{poyalso}

\end{command}

\begin{command}{exit}{} 

	\syntax{\obligatory{()}}

	\begin{poydescription}
         Exits a \poy session. This command does not accept any argument.
         \poycommand{exit} is equivalent to the command \poycommand{quit}.

         \begin{statement}
         To interrupt a process without quitting a \poy session, use Control-C.
         It aborts a currently running operation but keeps all the previously accumulated
         data in memory. It does not abort the current session permitting entering new
         command and continuing the session.
        \end{statement}
	
	 \end{poydescription}
	 
    \begin{poyexamples}
        \poyexample{exit()}
            {Quits the program.}
    \end{poyexamples}

    \begin{poyalso}
        \cross{quit}
    \end{poyalso}

\end{command}

\begin{command}{fuse}{}

    \syntax{\obligatory{(\optional{argument list})}}

    \begin{poydescription}
            Performs Tree Fusing~\cite{goloboff1999} on the trees in memory. Tree Fusing method to escape local optima
            by exchanging clades with identical composition of terminals between
            pairs of trees. Only \emph{one} pair of trees is evaluated during a single iteration.
            The size of the clades being exchanged is not specified.
    \end{poydescription}

    \begin{arguments}
        \argumentdefinition{keep}{\obligatory{\poyint}}
            {Specifies the maximum number of trees to keep between iterations.
            By default, the number of trees retained is the same as the number
            of starting trees.}
            {}

        \argumentdefinition{iterations}{\obligatory{\poyint}}
            {Specifies the number of iterations of tree fusing to be performed. The
            number of iterations is effectively the number of pairwise clade exchanges. 
            The default number of iterations is four times the number of retained
            trees (as specified by \poyargument{keep}).}
            {}

        \argumentdefinition{replace}{\obligatory{argument}}
            {Specifies the method for tree selection. Acceptable arguments
            are:
            \begin{description}
                \item[better] Replaces parent trees with trees of better cost
                produced during a fusing iteration.
                \item[best] Keeps a set of trees of the best cost regardless their origin.
            \end{description}
            The default is \texttt{best}.}
            {}

        \argumentdefinition{swap}{}
            {Specifies tree swapping strategy to follow each iteration of tree fusing.
            No swapping is performed under default settings.
            See the description of the command~\nccross{swap}{swapcommand}.}
            {}

    \end{arguments}
    
    \poydefaults{replace:best}
        {By default \poy performs fusing keeping the same number of trees per
        iterations as the number of the starting trees. The number of iterations is
        four times the number starting trees. During the procedure, only the best
        trees are retained. No swapping is performed subsequent to tree fusing.}
        
    \begin{poyexamples}
	
	\poyexample{fuse(iterations:10, replace:best, keep:100, swap())}
            {This command executes the following sequence of operations. In the
            first iteration, clades of the same composition of terminals are exchanged
            between two trees from the pool of the trees in memory. The cost of the
            resulting trees is compared to that of the trees in memory and a subset of
            the trees containing up to 100 trees of best cost is retained in memory.
            These trees are subjected to swapping under the default settings of
            \poycommand{swap}. The entire procedure is repeated nine more times.}
            
            \poyexample{fuse(swap(constraint))}
            {This command performs tree fusing  
            with modified settings for swapping that follows each iteration. Once
            a given iteration is completed, a consensus tree of the files in memory
            is computed and used as constraint file for subsequent rounds of swapping (see
            the argument \nccross{constraint}{swapconstraint} of the command
            \poycommand{swap}).}

     \end{poyexamples}
        
        \begin{poyalso}
        \cross{swap}
    \end{poyalso}

\end{command}

\begin{command}{help}{}

	\syntax{\obligatory{(\optional{argument})}}
	
	\begin{poydescription}
         Reports the requested contents of the help file on screen.
	\end{poydescription}
	
	\begin{arguments}
        \argumentdefinition{LIDENT}{}
            {Reports the description of the command, the name of which is specified by the
            LIDENT value.}
            {}

        \argumentdefinition{STRING}{}
            {Reports every occurrence in the help file of the expression specified by the string value.}
            {}
	\end{arguments}
	
	\poydefaults{}
        {By default \poy displays the entire content of the help file on screen.}
        
	\begin{poyexamples}
		\poyexample{help(swap)}
            {Prints the description of the command
            \poycommand{swap} in the \emph{POY Output} window of the ncurses
            interface or to the standard error in the flat interface.}

        \poyexample{help("log")}
            {Finds every command with text containing the substring \texttt{log} and
            prints them in the \emph{POY Output} window of the ncurses
            interface or to the standard error in the flat interface.}

	\end{poyexamples}

\end{command}


\begin{command}{inspect}{}

	\syntax{\obligatory{(\poystring)}} 

	\begin{poydescription}
        Retrieves the description of a \poy file produced by the command \ccross{save}. If the description was
        not specified by the user, \poycommand{inspect} reports that the
        description is not available. If the file is not a proper
        \poy file format, a message is printed in the \emph{POY Output}
        window of the ncurses interface or to the standard error of the flat interface.

        \poy files are not intended for permanent storage. They are recommended
        for temporary storage of a \poy session, checkpointing
        the current state of the search (to avoid losing data in case the computer or the
        program fails), or reporting bugs. \poy also automatically
        generates \poy files in cases of terminating errors (important exceptions are
        out-of-memory errors). 

    \end{poydescription}

    \begin{poyexamples}
        \poyexample{inspect("initial\_search.poy")}
            {Prints the description of the \poy file \texttt{initial\_search.poy}
            located in the current working directory in the \emph{POY Output}
            window of the ncurses interface or to the standard error in the flat
            interface. If, for example, the file was saved using
            the command \poycommand{save ("initial\_search.poy", "Results of
            Total Analysis")}, then the output message is: \texttt{Results of
            Total Analysis}.}
    \end{poyexamples}

    \begin{poyalso}
        \cross{save}
        \cross{load}
        \cross{cd}
        \cross{pwd}
    \end{poyalso}

\end{command}

\begin{command}{load}{}

	\syntax{\obligatory{(\poystring)}} 

	\begin{poydescription}
             Imports and inputs \poy files created by the command
             \poycommand{save}.  The name of the file to be loaded
              is included in the string argument. All the information of
               the current \poy session will be replaced with the contents
              of the \poy file. If the file is not in proper \poy file
            format, an error message is printed in the \emph{POY Output}
            window of the ncurses interface, or the standard error in the flat interface.
            See the description of the command \ccross{save} on the \poy file
            and its usage.

            \poy files are not intended for permanent storage: they are recommended
        for temporary storage of a \poy session, checkpointing
        the current state of the search (to avoid losing data in case the computer or the
        program fails), or reporting bugs. \poy also automatically
        generates \poy files in cases of terminating errors (an important exception is
        out-of-memory error).
	
	\end{poydescription}

    \begin{poyexamples}
        \poyexample{load("initial\_search.poy")}
            {Reads and imports the contents of the \poy file
            \texttt{initial\_search.poy}, located in the current working
            directory.}

        \poyexample{load("/Users/andres/test/initial.poy")}
            {Reads and imports the contents of the \poy file \texttt{initial.poy}
             in the absolute path described by the argument.}
    \end{poyexamples}

    \begin{poyalso}
        \cross{save}
        \cross{inspect}
        \cross{cd}
        \cross{pwd}
    \end{poyalso}

\end{command}

	 
\begin{command}{perturb}{}

	\syntax{\obligatory{(\optional{argument list})}}

	\begin{poydescription} 
        Performs branch swapping on the trees currently in memory using a temporarily modified 
        (``perturbed'') characters. Once a local optimum is found for 
        the perturbed characters,
        a new round of swapping using the original (non-modified) characters is
        performed. Subsequently, the costs of the initial and final trees are
        compared and the best trees are selected. If there are $n$ trees in
        memory prior to searching using \poycommand{perturb}, then the  $n$ best
        trees are selected at the end. For example, if there are 20 trees currently in memory,
        20 individual \poycommand{perturb} procedures will be performed (each
        procedure starting with one of the 20 initial trees), and 20
        final trees are produced.        
        This command allows for movement from a local search optimum in the tree space by
        \emph{perturbing} the character space (hence the name). The
        arguments specify the type of perturbation (\poyargument{ratchet},
        \poyargument{resample}, and \poyargument{transform}), the parameters of the
        subsequent search (\poyargument{swap}), and the number of iterations 
        of the \poycommand{perturb} operation (\poyargument{iterations}).
        
        No new Wagner trees are generated following the perturbation of the
        data; the search is performed by local branch swapping (specified by
        \poyargument{swap}). If \poycommand{perturb} is executed with no
        trees in memory, an error message is generated. The arguments of
        \poycommand{perturb} are optional and their order is arbitrary. 
 	\end{poydescription}

	\begin{arguments}

         \argumentdefinition{iterations}{\obligatory{\poyint}}
            {Repeats (iterates) the \poycommand{perturb} procedure for the
            number of times specified by the integer value. The number of iterations
            is reported in the \emph{Current Job} window of the ncurses interface
            and to the standard error in the flat interface.}
            {}

        \argumentdefinition{ratchet}{\optional{(\poyfloat, \poyint)}}
            {Perturbs the data by implementing a variant of the parsimony
            ratchet~\cite{Nixon1999} by reweighting characters listed in \poycommand{report(data)}. 
            For unaligned data,\poyargument{ratchet} randomly selects and reweighs a fraction of
            sequence fragments (\emph{not} individual nucleotides) specified
            by the float (decimal) value, upweighted by a factor specified by the integer
            value (severity). Thus, the number of sequence fragments into which the 
            data is partitioned will impact the effectiveness of using the ratchet on dynamic character matrices.  
            For static matrices, such as those obtained using the 
            command~\nccross{transform}{transformcommand},\poycommand{ratchet} randomly selects and
            reweights individual nucleotide positions (column vectors), as in Nixon's
            original implementation ~\cite{Nixon1999}.
            
            Under default settings,
            \poyargument{ratchet} selects 25 percent of characters and upweights
            them by a factor of 2.  Unless \poyargument{ratchet} is performed
            under default settings (that does not require the specification of the
            fraction of data to be reweighted and the severity value), both
            values must be specified in the proper order and separated by a comma.
            This argument is only used as an argument for \poycommand{perturb}.}
            {}

        \argumentdefinition{resample}{\obligatory{(\poyint, \poylident)}}
            {Resamples the data (characters or terminals) in random order with
            replacement. The \poyargument{resample} string consists of an 
            integer value
            specifying the number of items to be resampled (followed by a comma)
            and a lident value specifying whether characters or terminals
            (values \poycommand{characters} and \poycommand{terminals}, respectively)
            are to be resampled. Specifying both values
            is required. No default settings are available for \poyargument{resample}. This
            command is only used as an argument of \poycommand{perturb}.}
            {}

         \argumentdefinition{swap}{}
            {Specifies the method of branch swapping for a local tree search
            based on perturbed data. If the argument \poyargument{swap}
            is omitted, the search is
            performed under default settings of the
            command~\nccross{swap}{swapcommand}.}
            {swaparg}

        \argumentdefinition{transform}{}
            {Specifies a type of character transformation to be performed
            \emph{before} executing a \poycommand{perturb} procedure.
            See the command~\nccross{transform}{transformcommand} for
            the description of the methods of character type transformations
            and character selection.}
            {}

	\end{arguments}

    \poydefaults{ratchet, swap (trees:1)}
        {When no arguments specified, \poy performs the ratchet procedure under default
        settings.}
	
	\begin{poyexamples}
	
	\poyexample{perturb(resample:(50,terminals), iterations:10)}
	{Performs 10 successive repetitions of random resampling of 50 percents of
	terminals with replacement. Branch swapping is performed using
	alternating SPR and TBR, and and keeping one minimum-cost
	tree (the default of \poycommand{swap}).}
	
	\poyexample{perturb(iterations:20, ratchet:(0.18,3))}
            {Performs 20 successive repetitions of a variant of the ratchet (see
            above) by randomly selecting 18 percent of the characters (sequence
            fragments) and upweighting them by a factor of 3. Branch swapping is
            performed using alternating SPR and TBR, and keeping one
            optimal tree (the default of \poycommand{swap}).}

           \poyexample{perturb(iterations:1, transform (tcm:(4,3)))}
            {Transforms the cost regime of all applicable characters (\emph{i.e.} molecular 
            sequence data) to the new cost regime specified by
            \poyargument{transform} (cost of substitution 4 and cost of indel 3).
            Subsequently a single round of branch swapping is
            performed using alternating SPR and TBR, and and keeping one
            optimal tree (the default of \poycommand{swap}).}
            
            \poyexample{perturb(ratchet:(0.2,5), iterations:25, swap(tbr, trees:5))}
            {Performs 25 successive repetitions of a variant of the ratchet (see
            above) by randomly selecting 20 percent of the characters (sequence
            fragments) and upweighting them by a factor of 5. Branch swapping is
            performed using TBR and keeping up to 5 optimal trees in each iteration.}
            
            \poyexample{perturb(transform(static\_approx), ratchet:(0.2,5), \\
            iterations:25, swap(tbr, trees:5))}
            {Transforms all applicable (\emph{i.e.} dynamic homology sequence characters) using
            \poyargument{transform} into static characters. 
            Therefore, the subsequent ratchet is performed at the level of
            individual nucleotides (as in the original implementation), \emph{not}
            sequence fragments. Thus, ratchet is performed by selecting 20 percent of
            the characters (individual nucleotides) and upweighting them by a factor of 5.
            Branch swapping is performed using TBR and keeping up to 5 optimal trees in 
            each iteration as in the example above.}

	\end{poyexamples}
               
	\begin{poyalso}
		\ncross{swap}{swapcommand}
        		\ncross{transform}{transformcommand}
	\end{poyalso}
	
\end{command}

\begin{command}{pwd}{pwd}

	\syntax{\obligatory{()}}
	
	\begin{poydescription}
         Prints the current working directory in the \emph{POY Output} window of
         the ncurses interface and the standard error (stderr) of the flat interface.
         The command \poycommand{pwd} does not have arguments. The default
         working directory is the shell's directory when \poy started.
	\end{poydescription}
	
	\begin{poyexamples}
		\poyexample{pwd()}
            {This command generates the following message: ``The current
            working directory is /Users/myname/datafiles/''. The actual reported
            directory will vary depending on the directory of the shell when
            \poy started, or if it has been changed using the command
            \poycommand{cd()}.}
    \end{poyexamples}

    \begin{poyalso}
        \cross{cd}
    \end{poyalso}

\end{command}

\begin{command}{quit}{}
	
	\syntax{\obligatory{()}}
	
	\begin{poydescription}
        Exits \poy session. This command does not have any arguments
        \poycommand{quit} is equivalent to the command \poycommand{exit}.
        \end{poydescription}

	\begin{statement}
	 To interrupt a process without quitting a \poy session, use Control-C.
	 It aborts a currently running operation but keeps all the previously accumulated
	 data in memory. It does not abort the current session permitting entering new
	 commands and continuing the session.
	\end{statement}

    \begin{poyexamples}
        \poyexample{quit()}{Quits the program.}
    \end{poyexamples}
    
    \begin{poyalso}
        \cross{exit}
    \end{poyalso}
\end{command}

\begin{command}{read}{}

    \syntax{\obligatory{(\optional{argument list})}}  

	\begin{poydescription} 
        Imports data files and tree files.  Supported formats are ASN1, Clustal, FASTA,
        GBSeq, Genbank, Hennig86, Newick, NewSeq, Nexus, PHYLIP, POY3,
        TinySeq, and XML. Filenames should be enclosed in quotes and, if multiple
        filenames are specified, they must be separated by commas. 
        All filenames read into \poy should include the appropriate suffix (\emph{e.g.} .fas, .ss, .aln).
        \poycommand{read} automatically detects the type of the input file. 
        \poycommand{read} can use wildcard expressions (such as *) to
        refer to multiple files in a single step. For example, \poycommand{read("biv*")} 
        imports all data files the names of which start
        with \texttt{biv} or \poycommand{read("*.ss")} imports all files with
        the extension \texttt{.ss} (given that the data files are in the current directory).
        Specifying filename(s) is
        obligatory: an empty argument string, \poycommand{read()}, results in no
        data being read by \poy. The list of imported files and their content
        can be reported on screen or to a file using \poycommand{report(data)}.
        
        If a file is loaded twice, \poy issues an error message but this will not
        interfere with subsequent file loading and execution of commands.
        
       \poy automatically reports in the \emph{POY Output} window of the ncurses
            interface or to the standard error in the flat interface the names
            of the imported files, their file type, and a brief description of
            their contents. A more comprehensive report on the contents of the imported
            files can be requested (either on screen or to a file) using the argument
            \poyargument{data} of the command \ccross{report}.

        \begin{statement}
            Although \poy recognizes multiple data file formats, it does not
            interpret all of their contents. Instead, it will recognize and import
            only character data and ignore other content (such as blocks of
            commands, \emph{etc.}). For certain data file formats, \poy will interpret
            additional information as detailed for each file type below.
            It is important, however, to verify that the data was interpreted properly (using
            the command \poycommand{report}).
            
            The terminal names, as well as input file names, must not contain
            spaces, ``at'' or percentage symbols.
        \end{statement}
          
        \begin{statement}
            Unlike many phylogenetic programs, \poy does not clear the memory
            upon reading a second file. Instead, any subsequently read files
            will be added to the total data being analyzed.  If a \emph{new} taxon
            appears in a file, then it is be assigned missing data for all
            previously loaded characters. If a taxon does \emph{not} appear in a
            file, missing data are assigned for the characters that appearing in it. 
            
            To eliminate the imported data and then to input a new data
            the \poycommand{wipe()} command must be issued
            first. 
        \end{statement}
        
        \begin{statement}
             If one of the terminal names in an imported molecular file contains
             a space, `` '', \poy issues a warning. This also occurs if a
             taxon name appears to match a nucleotide sequence.

             If one of the terminal names in an imported molecular file contains
             an ``at'' or a percentage symbols, the file will not be loaded because
             it may cause the program to crash when reporting results.
        \end{statement}
	\end{poydescription}

	\begin{arguments}

	  \begin{argumentgroup}{Data file types}
	  To import data files, individual data file names must be included in the list of
	  \poycommand{read} arguments, enclosed in quotes, and separated
	  by commas. If no data file types are specified, the types of the imported
	  files are recognized automatically. To specify the data type,
	  an additional argument explicitly denoting the data type,
	  is included; it is followed by a colon (``:'') and the
	  list of data file names (enclosed in parentheses), separated by commas and enclosed in quotes. This
	  format prevents any ambiguity in importing multiple data file types
	  simultaneously (\emph{i.e.} included in an argument list of a single \poycommand{read})
	  command.

          \argumentdefinition{STRING}{}
              {Reads the file specified in the path included in the string argument.
              A path can be absolute or relative to the current
              working directory (as printed by \poycommand{pwd()}). The file type
              is recognized automatically.

              Molecular files are assumed to contain nucleotide sequences. Valid
              files to read using this command are: tree files using
              parenthetical notation (newick, \poy trees), Hennig86 files, Nona
              files, Sankoff character files as used in POY 3, FASTA files (and
              virtually any file generated by Genbank), and NEXUS files. Only
              taxon names, trees, characters, and cost regimes will be imported
              from each one of this files, no other commands are currently
              recognized.}
              {}
              
            \begin{statement}
            \poy recognizes the characters \emph{x} and \emph{n} as representing any nucleotide 
            base (a,c,t, or g).  The \emph{?} symbol inserted in sequence data signifies missing data,
            a gap, or any nucleotide base may occur in that matrix position.  
            For prealigned data sequence gaps are recognized by dashes.  
                    \end{statement}

	
	\begin{statement}
            Continuous characters can be treated as such by assigning the lower
            and upper bounds of the range as polymorphic additive character 
            states. Because additive characters are integers, such characters 
            need to be re-scaled using the \poyargument{weightfactor} of the 
            \poycommand{transform()}. Consider a continuous
            character \texttt{winglength}, the states of which are ranges of 
            measurements in hundredth of a millimeter, for example 2.53-3.68 
            mm for a given terminal. A corresponding character state in the 
            additive character matrix (in Hennig86 format) is \texttt{[253,368]}. 
            To scale the values, a transformation is applied to the character 
            \texttt{winglength} as follows:
            \poycommand{transform((characters,names:("winglength")),\\(weightfactor:0.01))}.
	\end{statement}
	
        \argumentdefinition{aminoacids}{\obligatory{(\poystring list)}}
            {Specifies that the data listed in the string argument
            are amino acid sequences in FASTA format.} 
            {}
            
	\begin{statement}
            Currently, IUPAC ambiguity codes for aminoacids are \emph{not} supported 
            and inputing files that contain aminoacid data with ambiguities results in
            an error message.
        \end{statement}
        
        \argumentdefinition{annotated}{\obligatory{(\poystring list)}}
            {Specifies that the data listed in the string argument are chromosomal
            sequences with pipes (`` $\vline$ '') separating individual
            loci. This data type allows for locus-level rearrangements specified by
            the argument \nccross{dynamic\_pam}{dynamicpam} of the command
            \nccross{transform}{transformcommand}. Locus homologies are
            determined dynamically, but based on annotated regions~\cite{vinh2006}
            (for a sample script using this data type see tutorial \texttt{4.7}.} 
            {}

        \argumentdefinition{breakinv}{\obligatory{(STRING, STRING, 
        [orientation:\poybool, init3D:\poybool])}}
            {An \\ enhancement of the data file type \poyargument{custom\_alphabet} allowing
            rearrangement events specified using \poycommand{dynamic\_pam()}. Syntactically,
            breakinv data type is identical to custom\_alphabet data type.} 
            {breakinv}

        \argumentdefinition{chromosome}{\obligatory{(\poystring list)}}
            {Specifies that the data in the files listed in the string argument
            are chromosomal sequences without predefined locus boundaries.
            Specifying that imported sequences are chromosome type data enables
            the application of parameter options that optimize chromosome-level
            events such as rearrangements, inversions, and large-scale
            insertions and deletions (including duplications). These parameter
            options (\emph{e.g.} inversion cost) are specified using the
            argument \poycommand{dynamic\_pam} in the
            command~\nccross{transform}{transformcommand}.  
            Unlike when using \poycommand{annotated} data type,
            both locus-level and nucleotide-level homologies
            are determined dynamically~\cite{vinh2007} (see tutorial \texttt{4.6}). If chromosome sequences were imported as
            nucleotide type data, they can be converted to chromosome type data
            using the argument \poyargument{seq\_to\_chrom} of
            \nccross{transform}{transformcommand}.} 
            {}
            
            \argumentdefinition{custom\_alphabet}{\obligatory{(STRING, STRING, 
        [orientation:\poybool, init3D:\poybool])}}
            {Reads the data in the user-defined alphabet format. The first string argument is
            the name of a datafile that contains custom-alphabet sequences in FASTA format. 
            The characters can be (but are not required to be) separated by spaces.
            
            The second string argument is the name of a custom-alphabet input file that contains two parts:
            an alphabet itself, where the alphabet elements are separated by spaces, and a
            transformation cost matrix. The elements in an alphabet can be letters, digits, or
            both, as long as one element is not a prefix of another  (``prefix-free''). For
            example, the following pairs of custom-alphabet elements are \emph{not} valid
            because the first is a prefix of the second (which would prevent the proper parsing of
            an input file): \texttt{AB} and \texttt{ABBA} or \texttt{122} and \texttt{122X}.
            The transformation cost matrix contains the rows and columns in which the
            positions from left to right and top to bottom correspond to the sequence of the
            elements as they are listed in the alphabet. An extra rightmost column and lowermost
            row correspond to a gap. It is impotant that the cost matrix must be symmetrical. An example 
            of a valid custom alphabet input file is provided below:
       	  
	  \texttt{alpha beta gamma delta \\
            0 2 1 2 5 \\
            2 0 2 1 5 \\
            1 2 0 2 5 \\
            2 1 2 0 5 \\
            5 5 5 5 0}
            
           In this example, the cost of transformation of \texttt{alpha} into \texttt{beta} is \texttt{2},
           and cost of a deletion or insertion of any of the four elements costs \texttt{5}.
           
           An example of a corresponding input file:
       
           \texttt{$>$Taxon1\\
	alphabetagammadelta\\
	$>$Taxon2\\
	alphabetabetagammadelta\\
	$>$Taxon3\\
	alphabetabetadelta}
	
	The optional arguments of \poyargument{custom\_alphabet} include \poyargument{orientation}
	and \poyargument{init3D}, both of which require obligatory boolean values. The argument
	\poyargument{orientation} allows the user to specify the orientation of custom-defined alphabet
	characters. The \emph{tilde} symbol (``$\sim$'') preceding an alphabet character indicates
	the negative orientation. The options are \poyargument{orientation:true}
	or \poycommand{orientation:false}. The default option is \texttt{true}.
	
	The argument \poyargument{init3D} indicates that if program will calculate in advance
	the medians for all triplets of characters (a, b, c). The options are \poyargument{init3D:true} or
	\poyargument{init3D:false}. The default option is \texttt{true}.
	
	\poycommand{custom\_alphabet}
	can be transformed into \poycommand{breakinv} using \poycommand{transform}.}
	 {customalphabet}
        
        \argumentdefinition{genome}{\obligatory{(\poystring list)}}
            {Specifies that the data listed in the string argument are
            multichromosomal nucleotide sequences with the ``\atsymbol'' sign 
            separating individual chromosomes. This data type
            allows for chromosome-level rearrangements specified by
            the argument
            \nccross{dynamic\_pam}{dynamicpam} of the command
            \nccross{transform}{transformcommand}. Chromosome
            homologies are determined dynamically using distance
            threshold levels specified by the argument
            \nccross{chrom\_hom}{chromhom}
            of \nccross{transform}{transformcommand}(for a sample script using this data type see tutorial \texttt{4.9}.} 
            {}
            
        \argumentdefinition{nucleotides}{\obligatory{(\poystring list)}}
            {Specifies that the data in the list of files hold nucleotide
            sequences in FASTA format. The sequences can be divided in smaller
            fragments using a pound sign, and each fragment is visible as an
            individual character.} 
            {}

	\begin{statement}
            By default, upon importing prealigned sequence data, all the gaps are
             removed and the sequences are treated as dynamic homology characters.
             To preserve the alignment the data must be imported using the
             \poyargument{prealigned} argument of the command \poycommand{read}.
          \end{statement}
        
        \argumentdefinition{prealigned}{\obligatory{(read argument,
        tcm:\poystring [,gap\_opening:\poyint])}}
            {Specifies that \\ the input sequences are prealigned and
            should be assigned the transformation cost matrix from the 
            input file defined by the string argument. (See the argument~\nccross{tcm}{transformtcm} 
            of the command \poycommand{transform}.)}
            {}
        
        \argumentdefinition{prealigned}{\obligatory{(read argument,
        tcm:(\poyint, \poyint) [,gap\_opening:\poyint])}}
            {Specifies that\\ the input sequences are prealigned and should be
            assigned substitution and indel costs as defined by the
            \poyargument{tcm} and \poyargument{gapopening} arguments. (See the argument~\nccross{tcm}{transformtcm} 
            of the command \poycommand{transform}.)}
            {prealigned2}

	\end{argumentgroup}
		
	\end{arguments}

    \poydefaults{}{If no data files are specified, \poy does nothing. If however,
    data files are listed but character type is not indicated, \poy automatically
    detects data file types and interprets sequence files as nucleotides-type data.}

	\begin{poyexamples}
	
        \poyexample{read("/Users/andres/data/test.txt")}
            {Reads the file \texttt{test.txt} located in the path
            ``/Users/andres/data/''.}

        \poyexample{read("28s.fas", "initial\_trees.txt")}
            {Reads the file \texttt{28s.fas} and loads the trees in parenthetical notation
            of the file \texttt{initial\_trees.txt}.}

        \poyexample{read("SSU*", "*.txt")}
            {Reads all the files the names of which start with \texttt{SSU}, and all the
            files with the extension \texttt{.txt}. The types of the datafiles are determined
            automatically.}
        
        \poyexample{read(nucleotides:("chel.FASTA", "chel2.FASTA"))}
            {Reads the files \texttt{chel.FASTA} and \texttt{chel2.FASTA}, containing nucleotide
            sequences.}

        \poyexample{read(aminoacids:("a.FASTA", "b.FASTA", "c.FASTA"))}
            {Reads the amino acid sequence files \texttt{a.FASTA}, \texttt{b.FASTA}, and
            \texttt{c.FASTA}.}

        \poyexample{read("hennig1.ss", "chel2.FASTA", aminoacids:("a.FASTA"))}
            {Reads the Hennig86 file \texttt{hennig1.ss}, the FASTA file \texttt{chel2.FASTA}
            containing nucleotide sequences, and the amino acid
            sequence file \\ \texttt{a.FASTA}.}
            
        \poyexample{read(custom\_alphabet:("my\_data", "alphabet"))}
        {Reads the first file, \texttt{my\_data}, containing data in the format of a custom
        alphabet, which is defined in the second input file, \texttt{alphabet}. By default, the
        forward and reverse orientation (\texttt{orientation:true}) of custom-alphabet
        characters is considered and prior calculation of medians for their triplets
        (\texttt{init3D:true}) is performed.}
            
           \poyexample{read(annotated:("filea.txt", "fileb.txt"), \\ chromosome:("filec.txt"))}
           {Reads three files containing chromosome-type sequence data.
           The sequences in two files,
            \texttt{filea.txt} and \texttt{fileb.txt}, contain pipes (``~$\vline$~'') separating
            individual loci, whereas the sequences in the third, are without
            predefined boundaries.}
            
            \poyexample{read(genome:("mt\_genomes", "nu\_genomes")}
            {Reads two files containing genomic (multi-chromosomal) sequence data.}

	   \poyexample{read(prealigned:("18s.aln", tcm:(1,2)))}
	    {Reads the prealigned data file \texttt{18s.aln} generated from the nucleotide file \texttt{18s.FASTA}
	    using the the transformation costs \texttt{1} for substitutions and \texttt{2} for indels.}
	
	\poyexample{read(prealigned:(nucleotides:("*.nex"), tcm:"matrix1"))}
	    {Reads character data from all the Nexus files as prealigned data using the the transformation cost
	    matrix from the file \texttt{matrix1}.}

	\end{poyexamples}

	\begin{poyalso}
          \cross{report}
	\end{poyalso}

\end{command}

\begin{command}{rediagnose}{rediagnose}

	\syntax{\obligatory{()}}

	\begin{poydescription}
        Performs a re-optimization of the trees currently in memory. This
        function is only useful for sanity checks of the consistency of the data.
        Its main usage is for the \poy developers. This command does not have
        arguments.
	\end{poydescription}

    \begin{poyexamples}
        \poyexample{rediagnose()}{See the description of the command.}
    \end{poyexamples}

\end{command}

\begin{command}{recover}{}
    \syntax{\obligatory{()}}

    \begin{poydescription}
            Recovers the best trees found during swapping, even if the swap was
            cancelled. This command functions only if the argument \nccross{recover}{recoverarg} 
            was included in a previously executed 
            (in the current \poy session) command \poycommand{swap}. Otherwise, it has no effect.
	
	The trees imported by \poycommand{recover} are appended to those currently
	stored in memory.
	
	Note that using recovered trees is not intended for temporary storage of trees.
	It is useful only as an intermediary operation in a given part of a \poy session. When
	other commands that require clearing memory are executed (such as
	\poycommand{build}, \poycommand{calculate\_support}, or another
	\poycommand{swap}),
	the trees stored by \poyargument{recover} can no longer be retrieved.
            
    \end{poydescription}

    \begin{poyexamples}
        \poyexample{recover()}{If the command \poycommand{swap} (executed
        earlier in the current \poy session) contained the argument \poyargument{recover},
        for example, \poycommand{swap(tbr,recover)}, this command will restore the best
        trees recovered during swapping.}
    \end{poyexamples}

    \begin{poyalso}
        \ncross{swap}{swapcommand}
        \ncross{recover}{recoverarg}
    \end{poyalso}
\end{command}

\begin{command}{redraw}{}

	\syntax{\obligatory{()}}

	\begin{poydescription}
        Redraws the screen of the terminal. This command is only used in the ncurses
        interface, other interfaces ignore it. \poycommand{redraw} clears the
        contents of the \emph{Interactive Console} window but retains the contents
        of the other windows. It does not affect the state of the search and the data
        currently in memory.
	\end{poydescription}

    \begin{poyexamples}
        \poyexample{redraw()}{See the description of the command.}
    \end{poyexamples}

\end{command}

\begin{command}{rename}{}

	 \syntax{\obligatory{(\optional{argument list})}}  
	
	\begin{poydescription} 
        Replaces the name(s) of specified item(s) (characters or terminals). This command allows 
        for substituting taxon names and helps merging multiple datasets without modifying the original
        datafiles. More specifically, it can be used, for example, (1) for housekeeping purposes,
        when it is desirable to maintain long verbose taxon names (such as catalog or GenBank
        accession numbers) associated with the original datafiles but avoid reporting these 
        names on the trees; (2) to provide a single name for a terminal in cases where the corresponding
        data is stored in different files under different terminal names; and (3) to simply change an
        outdated or invalid terminal name.
        
        The command consists of a terminal or character identifier followed by a comma and then by
        either a string containing a synonymy file or a pair (or pairs) of strings containing the names of
        items being renamed.
	\end{poydescription}  
          
	\begin{statement}
            In order to change taxon names, the command \poycommand{rename} must be
            executed \emph{before} importing the datafiles (see command \nccross{read}{read})
            that contain character data the taxa to be renamed.
          \end{statement}
          
          \begin{statement}
            Once the command \poycommand{rename} is applied, subsequent commands 
            must refer to the terminals using the new, substitute names. This is critical, for example,
            when importing a terminals file using the command~\nccross{select}{select} or specifying
             a root using the command~\ccross{set}.
          \end{statement}
          
          \begin{arguments}
		\begin{argumentgroup}{Identifiers}
		{The identifiers specify whether terminals or characters are being renamed. An identifier
		must precede the subsequent arguments.}
		
		\argumentdefinition{characters}{}
                {Specifies that the subsequently items to be renamed are characters.} 
                {}
		\argumentdefinition{terminals}{}
                {Specifies that the subsequently items to be renamed are terminals.} 
                {}
		\end{argumentgroup}
	      
	      \begin{argumentgroup}{Specifying items to be renamed}
	      {These arguments allow to specify the items to be renamed either individually (by 
	      using a pair of string arguments) or in a group (by importing a \emph{synonymy} file.
	      The latter is useful when there are multiple items to be renamed and/or when it is
	      desirable to substitute a single name for  multiple ones.}
	      
	      \argumentdefinition{STRING}{}
                {Specifies the name of the file (a \emph{synonymy} file) that contains the list of
                terminals or characters to be renamed. The synonymy file has the following structure:
                each line contains a list of synonyms (two or more) separated by spaces. The name of the
                item listed first is going to be substituted for all the subsequently listed names. Consider,
                for example, a two-line synonymy file below:
                
                \texttt{alpha beta gamma \\
                delta 1\\}
                 
                 When this file is imported, the items \texttt{beta} and \texttt{gamma} will be
                 renamed as \texttt{alpha} and the item \texttt{1} will be renamed as \texttt{delta}
                 in all subsequently imported datafiles.}
                {}
                \argumentdefinition{(STRING, STRING)}{}
                {Specifies the names of individual items to be renamed. The first item is renamed
                as the second item: specifying \texttt{("alpha",\\"beta")} renames the character or taxon
                \texttt{alpha} to \texttt{beta}. To specify multiple pairwise name substitution, several 
                name pairs can be listed: \texttt{("alpha","beta"),("gamma","delta").}}
                {}
                
                \begin{statement}
                Note that when \poycommand{rename} is applied by specifying pairs of
                 synonyms in the command's argument,
                the substitute name is listed \emph{second}. However,  the substitute name 
                appears \emph{first} in a synonymy file, followed by one or more synonyms.
                \end{statement}
	      \end{argumentgroup}
	      
          \end{arguments}
          
          \begin{poyexamples}
        \poyexample{rename(terminals,"synfile")}{This command renames terminal names
            contained in the synonymy file \texttt{synfile} in all subsequently imported datafiles.}
            \poyexample{rename(terminals,("Mytilus\_sp","Mytilus\_edulis"))}{This command
            renames terminal taxon \texttt{Mytilus\_sp} as \texttt{Mytilus\_edulis} in all subsequently
            imported datafiles.}
    \end{poyexamples}

\end{command}
        
\begin{command}{report}{}

	\syntax{\obligatory{(\optional{argument list})}}

	\begin{poydescription} 
        Outputs the results of current analysis or loaded data in the \emph{POY Output}
        window of the ncurses interface, the standard output of the flat
        interface, or to a file. To redirect the output to a file, the file name in 
        quotes and followed by a comma must be included in the argument list
        of \poycommand{report}. All arguments for \poycommand{report} are
        optional. 
	\end{poydescription}

	\begin{arguments}

        \begin{argumentgroup}{Reporting to files}{}

            \argumentdefinition{STRING}{}
                {Specifies the name of the file to which all the specific types of report outputs,
                designated by additional arguments, are printed. If no additional arguments
                are specified, the data, trees, and diagnosis are reported to that file by
                default.
                
                A string (text in quotes) argument is interpreted as a filename.
                Therefore, \texttt{"/Users/andres/text"} represents the file \texttt{text} in
                the directory \texttt{/Users/andres} (in Windows
                \texttt{C:$\backslash$users$\backslash$andres}). If no path is given, the path
                is relative to the current working directory as printed by
                \poycommand{pwd()}.
                usage.} 
                {}
        \end{argumentgroup}
                
	\begin{argumentgroup}{Terminals and characters}
            {This set of arguments reports the current status of terminals and
            characters from the imported data files. }
		
            \argumentdefinition{compare}{\obligatory{(\poybool, identifiers,
            identifiers)}}
                {If the boolean argument is set to \texttt{false}, the command
                reports the ratios of all pairwise distances to their maximum length
                for the characters specified by character identifiers. If the boolean
                argument is set to \texttt{true}, the complement sequences for
                the characters specified by the second identifier are computed prior
                to reporting the distance.}{}

            \argumentdefinition{cross\_references}{\optional{identifiers\optional{STRING}}}
                {Reports a table with terminals being analyzed in rows, and the
                data files in columns. A plus sign (``+'') indicates that data for a given
                terminal is
                present in the corresponding file; a minus sign (``-'') indicates that it is
                not. \poyargument{cross\_references} is a very useful tool for visual
                representation of missing data.
                
                Under default settings, cross-references are reported for
                all imported datafiles. To report cross-references for some of
                the fragments within a given file, a single character, or a subset
                of characters, optional arguments must be specified. A combination of
                a character identifier (see command  \nccross{select}{select}) and
                the file names (specified in the the string value) is used to select specific
                datafiles to be cross-referenced. For example, if a
                command \poycommand{cross\_references:names:\\("file1")} is
                executed, the output is produced only for \texttt{file1}.
                
                The argument \poyargument{cross\_references:all} generates
                a table that shows presence and absence of fragments contained
                within each file. If each datafile contains a
                single fragment,  executing \poyargument{cross\_references:all}
                is equivalent to executing \poyargument{cross\_references}.
                
                By default, the cross-reference table is printed on screen or to an
                output file, if specified.}
                {crossreferences}

	     \argumentdefinition{data}{}
                {Outputs a summary of the input data.
                More specifically, \poy will report the number of
                terminals to be analyzed, a list of included terminals with
                numerical identification numbers, list
                of synonyms (if specified), a list of excluded terminals, a
                number of included characters in each character-type category
                (\emph{i.e.} additive, non-additive, Sankoff, and molecular) with the corresponding
                cost regimes, a list of excluded
                characters, and a list of input files. If the report is directed
                to a file with extension ``nex'' or ``nexus'' then
                the output is suitable for a nexus file (including the NEXUS
                header). Hennig format is produced if the report is directed
                to a file with extension ``ss'' or ``hen'' or ``hennig''.}
                {data}

            \argumentdefinition{searchstats}{}
                {Outputs a summary of the results of the last search command,
                including the number of builds, fuse, ratchet, and the costs of
                the trees found.}
                {searchstats}

            \argumentdefinition{searchstats}{}
                {Outputs a summary of the results of the last search command,
                including the number of builds, fuse, ratchet, and the costs of
                the trees found.}
                {searchstats}

            \argumentdefinition{seq\_stats}{\obligatory{identifiers}}
                {Outputs a summary of the sequences specified in the argument
                value, for all taxa. The summary includes the maximum, minimum,
                and average length and distance for all terminals.}
                {seqstats}

            \argumentdefinition{terminals}{}
                {Reports a list and number of terminals included and excluded
                per input file. Use the command~\ccross{select} for including and excluding
                terminals .}
                {}

            \argumentdefinition{treestats}{}
                {Reports the number of trees in memory per cost.}
                {treestats}

            \argumentdefinition{treescosts}{}
                {Reports the cost of each tree separated by colons. The output
                contains no formatting for easy processing by any scripting
                language.}
                {}

		\end{argumentgroup}

		\begin{argumentgroup}{Trees}
            {This set of arguments outputs tree representations
            in parenthetical, ascii (simple text), or PDF formats.
            The arguments specify the types of tree outputs. They include
            actual trees resulting from current searches, or imported from
            files, their consensus trees, or trees displaying support values.
            
            To select the root terminal in the tree representation, the command~\ccross{set} is used.
            
            Most analyses produce more than a single tree and it is
            often desirable to report only some of them. To
            report particular trees (for instance all optimal trees,
            randomly-selected trees, or all unique trees, \emph{etc.}), first the
            command~\ccross{select} must be applied to specify (select)
             the desired trees from all those stored in memory.} 

             \argumentdefinition{all\_roots}{}
                {In a tree with $n$ vertices (and therefore $n - 1$ edges),
                calculates the cost of the $n - 1$ rooted trees as implied by a
                root located in the subdivision vertex at each edge in the unrooted
                tree in memory.}
                {allroots}

            \argumentdefinition{asciitrees}{\optional{collapse\optional{\poybool}}}
                {Draws ascii character representations of trees stored in memory. The
                argument \poyargument{collapse} collapses the zero length branches if
                the boolean value is \texttt{true} (the default); if the boolean value is
                \texttt{false}, the zero length branches are not collapsed.}
				{}

			\argumentdefinition{clades}{}
	     {Output a set of Hennig86 files. Each file, named \texttt{file.hen},
                where ``file'' is whatever string you pass to this function
                contains information on each clade for one of the trees
                currently stored. This is similar to the utility jack2hen 
                of \texttt{POY3}.}
				{}

			\argumentdefinition{consensus}{\optional{\poyint}}
                {Reports the consensus of trees in memory in parenthetical notation.
                If no integer value is
                specified, a strict consensus is calculated~\cite{rohlf1982};
                if integer value is specified,
                a majority rule consensus is computed, collapsing nodes with
                occurrence frequencies less than the specified integer~\cite{margush1981}.
                If a value less
                than \texttt{51} is specified, \poy reports an error.} 
                {}

            \argumentdefinition{graphconsensus}{\optional{\poyint}}
                {Same as \poyargument{consensus} except for consensus trees are
                reported in graphical format, either in the ascii format on
                screen or in the PDF format if redirected to a file.}
                {}

            \argumentdefinition{graphdiagnosis}{}
                {Output the diagnosis in PDF format. The PDF is compressed, and
                contains the trees and links to see the diagnosis of each vertex
                in the tree.}
                {}

            \argumentdefinition{graphsupports}{\optional{argument}}
                {This command outputs a tree with support values that have
                been previously calculated using the
                \nccross{calculate\_support}{calculatesupport} either on screen
                in ascii format, or, if specified, to a file in PDF
                format. The argument values are the same as for 
                \poyargument{supports} (\emph{i.e.} \poyargument{bremer},
                \poyargument{jackknife}, and \poyargument{bootstrap}).} 
                {graphsupports}

            \argumentdefinition{graphtrees}{\optional{collapse\optional{\poybool}}}
                {If \poy has been compiled with graphics support, it 
                will display a window in which you can
                browse graphical representations of all the trees in memory.
                When working in this window, using ``j'' and ``k'' keys displays the
                previous or next tree respectively. If no graphical support available, the output 
                is similar to that generated by the \poyargument{asciitrees} argument. Pressing ``q'' key 
                returns to the \emph{Interactive Console} window. The argument
                \poyargument{collapse} will collapse the zero length branches if
                true, otherwise not (default is \texttt{true}.)} 
	     {}

            \argumentdefinition{supports}{\optional{argument}}
                {Outputs a newick format representation of a tree with the
                support values has previously been calculated using the
                command~\nccross{calculate\_support}{calculatesupport},
                either to the screen or to a file (if specified). If no argument
                is given, all calculated support values are printed. The arguments
                \poyargument{bremer}, \poyargument{jackknife}, and
                \poyargument{bootstrap} specify which type of support tree to
                report. 
                
                To print the bremer supports of the trees in memory, using as
                reference trees stored in a file, 
                \poyargument{bremer} accepts an optional string argument
                (as in \poycommand{report(supports:\\bremer:"file.txt")}, 
                or a list of strings encloed in parentheses (as in
                \poycommand{report(supports:\\bremer:("file1.txt",
                "file2.txt"))}. The argument's value specifies the files containing lists of trees and costs (as
                those generated by \ccross{visited}), that should be used with
                their annotated cost to assign the bremer support values. 

                To print the bremer supports of a tree that does not exists in
                memory (or a consensus tree)
                stored in a file, \poyargument{bremer} accepts the value
                \poyargument{of\_file:(\poystring, \poyint, files)}, where the
                first argument value is the file containing the tree for which
                bremer supports should be computed, the second argument is the
                cost of the tree, and the files is that described in the
                previous paragraph.

                If no input file is given, or if \poyargument{bootstrap} or
                \poyargument{jackknife} are requested,
                then the necessary information must have been calculated 
                using \nccross{calculate\_support}{calculatesupport}.
                
                \poyargument{jackknife} and \poyargument{bootstrap} accept an
                optional argument with two possible values:
                \poyargument{individual}, \poyargument{consensus}, or a
                \poystring.
                \poyargument{individual}
                reports the support value for each tree held in memory: if there
                are a hundred trees stored in memory, for each one, the support
                values for each tree are reported. \poyargument{consensus}
                generates a ``consensus'' tree, with the clades that have
                support higher than 50 percent. \poystring labels the branches
                in the input trees contained in the inputfile located in the path of
                the \poystring (e.g.\ to assign support values to the branches of a consensus
                tree). The default behavior, when no
                \poyargument{individual} or \poyargument{consensus} value is
                provided, is \poyargument{individual}.}
                {supports}

			\argumentdefinition{trees}{\obligatory{(argument list)}}
                {Outputs the trees in memory in parenthetical notation. The argument
                \poyargument{trees} receives an optional list of values
                specifying the format of the tree that has to be generated.
                Unless \poyargument{hennig} is specified in the list of values, 
                \poyargument{trees} uses newick format in the tree output. The
                valid optional arguments are:  
                
                \begin{description}
                    \argumentdefinition{total}{}
                        {Includes the total cost of a tree in square brackets
                        after each tree.}
                        {total}

                    \argumentdefinition{\_cost}{}
                        {Include the cost in square brackets for every subtree in the tree. (These 
                        are \emph{not} branch lengths.)}
                        {cost}

                    \argumentdefinition{hennig}{}
                        {Prepends the \poycommand{tread} command to the list of
                        trees and separates them with a star; this format is
                        suitable for Hennig86, NONA, and TNT files.}
                        {}
                        
                    \argumentdefinition{newick}{}
                        {Outputs the trees in the Newick format, with the
                        terminals separated with commas, and trees separated
                        with semicolons.}
                        {}
                        
                    \argumentdefinition{nexus}{}
                        {Outputs the trees in the NEXUS format, inside a TREE
                        block.}
                        {nexusreport}
	    \begin{statement}
	     The \poyargument{hennig} and \poyargument{newick} arguments are 
	     mutually exclusive.
	     \end{statement}
	     
                    \argumentdefinition{margin}{\obligatory{INTEGER}} 
                        {Sets the margin width of the generated trees.}
                        {}

                    \argumentdefinition{nomargin}{}
                        {Outputs the trees in a single line. This is useful for
                        some programs (such as TreeView) that cannot read
                        trees broken in several
                        lines.}
                        {}

                    \argumentdefinition{collapse}{\optional{\poybool}}
                        {If \poycommand{true}, zero length branches are collapsed (the
                        default), but if \poycommand{false}--no branches are
                        collapsed.}
                        {}
                    \end{description}
                 If the report is directed
                to a file with extension ``nex'' or ``nexus'' then
                the output is suitable for a nexus file (trees inside a TREES
                block). Hennig format is produced if the report is directed
                to a file with extension ``ss'' or ``hen'' or ``hennig''. In
                these two cases, all other formatting options are ignored.}
                {treesreport}

		\end{argumentgroup}

		\begin{argumentgroup}{Implied alignments}
            {This set of arguments outputs implied alignments~\cite{wheeler2003}.} 

            \argumentdefinition{fasta}{\obligatory{identifiers}}
            {The same as \nccross{implied\_alignments}{impliedalignment} but no additional headers
                are added, producing a valid FASTA file. Intended for easy
                automation, by producing a file that other programs can read
                immediately.}
                {fasta}

            \argumentdefinition{implied\_alignments}{\optional{identifiers}}
                {Outputs the implied alignments of the specified
                set of characters in FASTA format. The optional value of the
                argument specifies the characters included
                in the output, using the same identifiers described for the
                character specification in the entry for the command~\ccross{select}. If no
                characters are specified, then the implied alignment of all the
                sequence characters is generated. The output is reported on
                screen unless a name of an output file (in parentheses) is
                specified, preceding the command name and separated from it by a
                comma. This argument is synonymous with the argument
                \poyargument{ia}.}
                {impliedalignment}

            \argumentdefinition{ia}{\optional{identifiers}}
                {Synonym of \poyargument{implied\_alignments}.}
                {}

        \end{argumentgroup}

        \begin{argumentgroup}{Exporting static homology data}
            {The following commands export the static homology characters
            currently in memory.}

            \argumentdefinition{nexus}{}
                {Produces a file in the NEXUS format that contains all the
                characters currently in memory.  In
                order to export an implied alignment as a NEXUS file, the
                characters must first be transformed into static characters
                using the \poycommand{transform} command (see the Hennig86 example in tutorial \texttt{4.2}): 
                \begin{flushleft}
                    \poycommand{transform ((all, static\_approx))} \\
                    \poycommand{report ("report.nexus", nexus, trees:(nexus))}
                \end{flushleft}}
                {}

            \argumentdefinition{phastwinclad}{}
                {Produces a file in the Hennig86 format that contains the
                additive and nonadditive characters currently in memory.  In
                order to export an implied alignment as a Hennig86 file, the
                characters must first be transformed into static characters
                using the \poycommand{transform} command (see example in tutorial \texttt{4.2}): 
                \begin{flushleft}
                    \poycommand{transform ((all, static\_approx))} \\
                    \poycommand{report ("report.ss", phastwinclad)}
                \end{flushleft}}
                {}
	\begin{statement}
	      To generate a file that contains implied
                alignments only for a subset of fragments, an identifier must be
                included in the argument list of \poycommand{transform}. For
                example, 
                \begin{flushleft}
                \poycommand{transform ((names:("fragment\_1", "fragment\_2"),
                static\_approx))} \\
                \poycommand{report ("myfile.ss", phastwinclad)}
                \end{flushleft}
                will produce Hennig86 files only for
                \texttt{fragment\_1} and \texttt {fragment\_2}. The resulting file can be imported into other programs,
                such as WinClada.  This is equivalent to the
                \poycommand{phastwincladfile} command in \texttt{POY3}.
	\end{statement}
	
		\end{argumentgroup}

		\begin{argumentgroup}{diagnosis}
			{This set of arguments will output the diagnosis.} 

			\argumentdefinition{diagnosis}{}
                {Outputs the diagnosis of each tree on screen or redirects it to a file, if
                specified. If the extension \emph{.xml} is appended to the name of the
                output file, the diagnosis is reported in XML format, rather than in
                simple text format.} 
                {}

		\end{argumentgroup} 

		\begin{argumentgroup}{Other arguments}
			{} 

	     \argumentdefinition{ci}{}
	      {Calculates the ensemble consistency index (CI ~\cite{farris1989,
	      klugeandfarris1969}) for additive, nonadditive, and Sankoff
	      characters. Dynamic homology characters are ignored in calculating
	      the CI, therefore, the dynamic homology characters must be converted
	      to static homology characters using the argument \poyargument{static\_approx} 
	      of the command \nccross{transform}{transformcommand}.}
	      {ci}
	      
	      \argumentdefinition{memory}{}
                {Reports on screen, the statistics of
                the garbage collector. For a precise description of each memory parameter, see
                the Objective Caml documentation.}
                {}
	     
	      \argumentdefinition{ri}{}
	      {Calculates the ensemble retention index (RI; ~\cite{farris1989}) for additive,
	      nonadditive, and Sankoff characters. Dynamic homology characters are ignored in calculating
	      the RI, therefore, the dynamic homology characters must be converted
	      to static homology characters using the argument \poyargument{static\_approx} 
	      of the command \nccross{transform}{transformcommand}.}
	      {ri}
	      
	     \argumentdefinition{script\_analysis}{\obligatory{\poystring}}
                {Reports the order in which commands listed of the imported
                script (specified by the string argument) are going to be executed.
                Unlike executing individual commands interactively, when commands are submitted in a 
                script, \poy determines the logical interdependency of operations
                and processes the commands in the order that yields the same
                results as if they were executed sequentially. This substantially
                optimizes parallelization and reduces memory consumption.
                
                The colored output in the \emph{POY Output} window of the ncurses
                interface facilitates reading the output of \poyargument{script\_analysis}:
                red lines mark hard constraints that allow neither
                parallelization nor memory optimizations, blue lines mark 
                constraints that allow the program to pipeline commands in
                parallel, and green lines mark fully parallelizable commands. When \poy
                is compiled with parallel off, all the operations are
                sequential, therefore, each potentially parallel operation is
                done as sequential repetitions of the subscripts described in
                the output of the command, reducing memory consumption.}
                {scriptanalysis}
                
                \argumentdefinition{timer}{\obligatory{STRING}}
                {Reports the value and the user time (in seconds) elapsed between
                two consecutive timer reports. The string value provides a label
                (typically a textual description) that precedes the time report
                in the output produced.
                The first timer report displays the time elapsed since the beginning of the
                \poy session. This command is useful for monitoring the execution time
                of specific tasks.}
                {}
                
                \argumentdefinition{xslt}{\obligatory{(STRING, STRING)}}
	     {Applies a user-defined xslt stylesheet to the XML output. The first string is
	     the filename of the output, the second string is the name of the stylesheet
	     requested to generate it.}
	     {}
	     
	     \begin{statement}
	     Extensible Stylesheet Language Transformations (XSLT) are used
	     for the transformation of XML output into other formats. Because the XML output contains all the information regarding data and trees, using XSLT stylesheets greatly expand the capabilities of \poy to use and display results.
	     Examples of potential applications includes graphical display of trees with proportional branch lengths,
	     integration of tree topologies with geographical coordinate data for spatial mapping, and
	     generating input files for other programs.
	     \end{statement}
	     
	     \end{argumentgroup}
	\end{arguments}

	\poydefaults{data, diagnosis, trees}
        {By default, \poy will print on screen the following items: the tree(s)
        in parenthetical notation with corresponding tree cost(s), diagnosis of
        each tree, and a graphical representation on the tree(s) in ascii
        format. This output can be re-directed to a file by specifying a file
        name enclosed in quotation marks, for example:
        \poycommand{report("filename")}.}

	\begin{poyexamples} 

		\poyexample{report("my\_results")}
		{This commands outputs the data, trees, and diagnosis (the default) to the
		file \texttt{my\_results}. Because no path is specified, the
		file is located in the current working directory.}
		
		\poyexample{report(data)} 
            {This command displays on screen a list of included and excluded terminals, their
            names and codes, gene fragments, file names, and other relevant data.}
            
		\poyexample{report(treestats)}
            {This example displays on screen the costs of all trees in memory and the
            number of trees for each cost.}

		\poyexample{report("filename", treestats)} 
            {This commands outputs the costs of all trees in memory and the
            number of trees for each cost to a file \texttt{filename}.}

		\poyexample{report(cross\_references:names("file1", "file3"))}
		{This command produces a table showing presence
		and absence of data corresponding to all terminals contained
		in files \texttt{file1} and \texttt{file3}. Because an output
		file is not specified, the table is displayed on screen.}
		
		\poyexample{report("taxa", terminals)}
		{This command generates a file \texttt{taxa} that contains the
		lists and numbers of excluded and included terminals for each of the previously
		imported datafiles.}
		
		\poyexample{report(trees)}
            {This command displays on screen the trees in memory in parenthetical
            notation with zero-length branches collapsed and terminals
            separated by spaces.}

        	\poyexample{report(trees:(total))}
            {This command produces the same output as the example above
            but also includes the total tree cost in square brackets
            following each tree.}

	\poyexample{report("filename", trees:(collapse:false, newick))} 
            {This command produces a file \texttt{filename} that contains
            all trees in Newick format with zero-length branches \emph{not}
            collapsed.}
		
		\poyexample{report("filename", graphtrees)} 
            {This command saves all trees in memory in
            PDF format to the file \texttt{filename.pdf}.}

		\poyexample{report(asciitrees, "file1", trees:(newick, nomargin), \\ "file2", graphtrees)}
		{This command displays a tree in ascii format on screen and outputs
		to \texttt{file1} trees with zero-length branches collapsed in Newick format
		in a single line (using no margin, the format compatible with \emph{TreeView}). It
		also writes to \texttt{file2} the graphical representation of these trees in
		PDF format.}

        \poyexample{report("hennig.ss", phastwinclad, trees:(hennig, total))}
            {This command outputs all the static homology characters, including their cost
            regime, in the file \poycommand{hennig.ss}; then append to the same
            file the trees currently in memory using the Hennig format, 
            including the total cost of each tree in square brackets. The
            generated \poycommand{hennig.ss} is compatible with NONA, TNT, and
            Hennig86.
            \index{general}{export!hennig}\index{general}{export!nona} \index{general}{export!tnt}}
            
         \poyexample{report("my\_results", data, diagnosis, consensus, \\consensus:75,
         "consensus", graphconsensus)}
         {This command reports the requested types of outputs (\emph{i.e.}
        reports on the data, diagnosis, and strict consensus and 75 percent
         majority-rule consensus trees in parenthetical notation) to the file
         \texttt{my\_results}. In also outputs a strict consensus tree to the file
         \texttt{consensus}.}
         
         \poyexample{report(graphsupports, "bremertree", graphsupports:bremer)}
         {This command reports on screen all previously calculated support values
         placed at the nodes of ascii trees and outputs to file the \texttt{bremertree}
         only the tree(s) with bremer support values.}
         
         \poyexample{report(implied\_alignments)}
         {This command reports the implied alignments for all dynamic homology
         characters on screen.}
         
          \poyexample{report("align\_file", ia:names:("SSU", "LSU"))}
          {This command generates the file \texttt{align\_file} that contains
          the implied alignments only for characters contained in datafiles
          \texttt{SSU} and \texttt{LSU}.}
          
          \poyexample{report("scipt1\_analysis", script\_analysis:"/users/datafiles/\\script1.poy")}
          {This command produces the file \texttt{scipt1\_analysis} that lists the commands from
          the input script file \texttt{script1.poy} in the order that optimizes parallelization and
          memory consumption. In this example the complete path (\texttt{/users/datafiles/script1.poy})
          is provided, which is not necessary if the directory containing the file \texttt{script1.poy}
          has already been assigned using the command \ccross{cd} in the same \poy session.}
          
          \poyexample{report("swapping", timer:"swap end")}
          {This command generates the file \texttt{swapping} that contains
          the string \texttt{swap end} followed by the number of seconds (in
          decimals) elapsed since the execution of the previous \poyargument{timer}
          argument.}
          
          \poyexample{report("new\_tree\_diagnosis.xml", diagnosis)}
         {This command reports the diagnosis to the \texttt{new\_tree\_diagnosis.xml}
         file in XML format.}

	\end{poyexamples}

	\begin{poyalso}
        \ncross{calculate\_support}{calculatesupport}
	 \end{poyalso}

\end{command}

\begin{command}{run}{}

	\syntax{\obligatory{(\poystring)}}

	\begin{poydescription}
        Runs \poy script file or files. The filenames must be included in
        quotes and, if multiple files are included, they must be separated by commas.
        The script-containing files are executed in the order in which they are listed
        in the string argument.
        Executing scripts using \poycommand{run} is useful in cases when
        operations take take long time or many scripts need to be executed automatically,
        for example, when conducting sensitivity analysis\cite{wheeler1995}.
        There are no default settings of \poycommand{run}.
        \end{poydescription}
        
        \begin{statement}
  	Note that if any of the scripts contain the commands \poycommand{exit()} or
	\poycommand{quit()}, \poy will quit after executing that file. Therefore, if
	multiple files are submitted, only the last one must contain \poycommand{exit()}
	or \poycommand{quit()}.
	\end{statement}
	
	\begin{poyexamples}
        \poyexample{run("script1", "script2")}
            {This command executes \poy command scripts contained in the files \texttt{script1}
            and \texttt{script2} in the same order as they are listed in the list of arguments of
            \poycommand{run}.}
          \end{poyexamples}
          
          \begin{poyalso}
        		\cross{exit}
		\cross{quit}
	\end{poyalso}

\end{command}

\begin{command}{save}{}

	\syntax{\obligatory{(\poystring \optional{, \poystring})}}

	\begin{poydescription}
            Saves the current \poy state of the program to a file (\poy file). The first, obligatory string argument
            specified the name of the \poy file. The second, optional string argument specifies a
            string included in the \poy file, that can be retrieved using the command \ccross{inspect}. 

            \poy files are not intended for permanent storage: they are recommended
            for temporary storing of a \poy session by a user, checkpointing the
            current state of a search to avoid loss work in case the computer or the
            program itself fails, or to report bugs. \poy will also automatically
            generate the file in many cases when a terminating error occurs (an
            important exception is out-of-memory errors). The format of these files might differ among different versions of \poy; consequently, these files might not be interchangeable between all the versions of the program.
	\end{poydescription}

    \begin{poyexamples}
        \poyexample{save("alldata.poy")}
            {This command stores all the memory contents of the program in the file
            \texttt{alldata.poy} located in the current working directory, as
            printed by \poycommand{pwd()}.}

        \poyexample{save("alldata.poy", "My total evidence \\data")}
            {This command performs the same operation as described in the example above,
            but, in addition, it includes the string \texttt{My total
            evidence data} with the file \texttt{alldata.poy},
            which can later be retrieved using the command \ccross{inspect}.}

        \poyexample{save("/Users/andres/test/alldata.poy", "My total evidence \\data")}
            {This command performs the same operation as the command described above
            with the important difference that the file \texttt{alldata.poy} generated in the
            directory \texttt{/Users/andres/test/} instead of the current working directory.}
    \end{poyexamples}

    \begin{poyalso}
        \cross{inspect}
        \cross{load}
    \end{poyalso}

\end{command}

\begin{command}{search}{}

	\syntax{\obligatory{(\optional{argument list})}}

	\begin{poydescription}
            \poycommand{search} implements a default search strategy that
            includes tree building, swaping using TBR, perturbation using
            ratchet, and tree fusing. The strategy involves specifying targets for 
            a driven search, such as maximum and minimum execution times, 
            maximum allowed memory consumption for tree storage, minimum number of times the
            shortest tree is found, and an expected cost for the shortest tree.  When executing
             \poycommand{search} using parallel processing trees are exchanged upon the
              completion of the command (after fusing).  Because the lowest cost unique trees 
              generated are selected and stored at the end of a \poycommand{search} 
              (defined by the user with \poyargument{max\_time}), aggressive use of this 
              command in a parallel environment consists of including few sequential
               \poycommand{search} commands that will allow the processes to
               exchange trees and add the pool of selected
               best trees to subsequent iterations of the command (see the example for parallel processing).

               Trees that exists in memory prior the \poycommand{search} command
               are included in the set of trees available for the
               \poycommand{fuse} but are not swapped.
	\end{poydescription}

	\begin{arguments}

        \argumentdefinition{constraint}{\obligatory{\poystring}}{For a complete
        description see \nccross{constraint}{buildconstraint}.}{searchconstraint}

        \argumentdefinition{hits}{\obligatory{\poyint}}{Specifies the minimum number of
        times that the minimum cost must be reached before aborting the search. 
        The \poyargument{hits} argument is not used in parallel processing}{}
        	      
        
        \argumentdefinition{max\_time}{\obligatory{\poyfloat:\poyfloat:\poyfloat}}
        {Maximum total execution time for the search. The time is specified as
        days:hours:minutes. For example, executing the search for 1.5 days can
        be expressed as 1:12:00 or 1.5:00:00.}{maxtime}
        
        \argumentdefinition{memory}{\obligatory{\poylident:\poyfloat}}{Specifies the maximum amount of
        memory allocated for the stored trees during the search per processor. \poy \emph{attempts} to consume memory within
        the specified limit, but it may surpass it in certain operations (most
        notably during the ratchet). The lident value expresses the 
        units of memory (\poyargument{gb} for Gigabytes and \poyargument{mb} for Megabytes), 
        whereas the float value specifies the actual value. Keeping memory consumption 
        within the limit is approximate and is used as a rough guide to \poy, preventing 
        the program from overflowing the memory.  Furthermore, it is important to note that 
        when running \poy in parallel the maximum amount of memory specified by the user is 
        allocated to each processor being used.  Under certain circumstances, however, 
        it might be required to use more memory to avoid program failures.}{}

        \argumentdefinition{min\_time}{\obligatory{\poyfloat:\poyfloat:\poyfloat}}
        {Specifies the minimum total execution time for the search. The time is specified as
        days:hours:minutes. This command is useful when
        the number of \poyargument{hits} is specified but the actual cost of the
        tree is unknown. In this case, \poy performs the search for at least the time
        specified by this argument.}{mintime}

        \argumentdefinition{target\_cost}{\obligatory{\poyfloat}}{Specifies the 
        upper limit for the cost of the shortest tree.}{targetcost}

        \argumentdefinition{visited}{\optional{\poystring}}{For a complete
        description see \ccross{visited}. Note that this argument has a
        tremendous execution time cost, as printing the trees becomes a
        bottleneck for the application.}{searchvisited}

	\end{arguments}

    \poydefaults{max\_time:0:1:0, min\_time:0:1:0, memory:gb:2}{Under default
    parameters, the program performs a search for at most one hour using at most
    2 GB of memory. \emph{If the user does not specify the value of \poyargument{max\_time}, 
    the search will be aborted after one hour.}}
    
        \begin{statement}
                In order to maximize computational efficiency when using \poycommand{search} 
                in parallel processing environments the \poyargument{hits}
                argument is ignored. However, a diverse set of trees which
                include the current best trees found among all the processes is
                desirable to improve the potential of tree fusing.

                POY will \emph{only exchange trees between processes at the end
                of each search command}. Therefore,
                to guarantee that separate processes seed each other with the
                best trees they have found every number of hours,
                it is advisable to use few successive search commands
                when executing the program in parallel. Each search will still
                be run in parallel, but after each one, trees will be exchanged
                between processors, to initiate each successive round of search.
        \end{statement}

        
    	\begin{poyexamples}
        	
	\poyexample{search(hits:100, target\_cost:385, max\_time:1:12:13)}
            {This command will attempt as many builds, swaps, ratchets, and tree
            fusings as possible within the specified time of 1 day, 12 hours, and 13
            minutes, finding at least 100 hits (whichever occurs first, the time
            limit or the number of hits), knowing that the expected cost of the
            best hits is at most 385 steps.}
	
	\poyexample {For Parallel Implementation of \poycommand{search}}
            {{search(max\_time:0:6:0)}\\
	                      {search(max\_time:0:6:0)}\\
	                      {search(max\_time:0:4:0)}\\
            {This series of commands will attempt as many builds, swaps, ratchets, and tree
            fusings as possible within the specified total time of 16 hours.  Trees are exchanged among
             processors at the end of each \poycommand{search} and the best unique trees
              are then selected and included in the following
              \poycommand{search} command.}}
            \end{poyexamples}

	\begin{poyalso}
  		\ncross{build}{buildcommand}
		\ncross{swap}{swapcommand}
		\ncross{transform}{transformcommand}
	\end{poyalso}

\end{command}

\begin{command}{select}{}

	\syntax{\obligatory{(\optional{argument})}}

	\begin{poydescription} 
            Specifies a subset of terminals, characters, or trees from those
            currently loaded in memory to use in subsequent analysis.
	\end{poydescription}
	

	\begin{arguments}
		
		\begin{argumentgroup}{Select terminals and characters}
            {Specifies terminals and characters to use in subsequent
            analysis. 
            The arguments in this group specify whether terminals or characters
            are being selected.
            \emph{Identifiers} are used to specify which characters or
            terminals are being selected (see
            the \emph{Character and terminal identifiers} argument group below
            for the description of methods for selecting specific terminals or characters).}
 
            \argumentdefinition{terminals}{}
                {Specifies that the subsequently listed identifiers
                refer to \emph{terminals} to be selected. By default, \poy
            assumes that the specification refers to terminals. For example, to
            analyze only those terminals listed in the file \texttt{opiliones} using
            the character data currently loaded in memory, use the command 
            \poycommand{select(files:("opiliones"))}. This  command is
            equivalent to \poycommand{select(terminals,files:("opiliones"))}.
            
            When the command is executed, the list of selected terminals is
            printed on screen.  \poycommand{terminals} is only valid as an
            argument of commands \poycommand{select} and \ccross{rename}.} 
                {}
	
	\begin{statement}
  	Note that once specific terminals and/or  characters are selected, the excluded
	data cannot be restored. To be able to reconstitute the original data set or to
	experiment with various character and terminal selections within a given \poy
	session, use the commands ~\ccross{store}{} and \ccross{use}.
	\end{statement}
	
	\argumentdefinition{characters}{}
                {Specifies that the subsequently listed identifiers
                refer to \emph{characters} to be selected.}
                {}

            \argumentdefinition{STRING}{}
                {Selects terminals listed in the file specified by the string argument.}
                {}

		\end{argumentgroup}
		
        \begin{argumentgroup}{Character and terminal identifiers}\label{identifiers}
        {\emph{Identifiers} specify which characters or terminals are analyzed.
        In addition to the command \poycommand{select}, identifiers are used as
        arguments for other commands that require selection of specific terminals or
        characters, such as commands \ccross{report} and
        \nccross{transform}{transformcommand}.}

            \argumentdefinition{all}{}
                {Specifies all characters or terminals.}
                {allidentifier}

            \argumentdefinition{names}{\obligatory{(\poystring list)}}
                {Specifies the names of the characters or terminals.}
                {}

            \argumentdefinition{codes}{\obligatory{(\poyint list)}}
                {Specifies the codes of characters or terminals. The codes are unique
                numbers that are generated by \poy when data files are first imported.
                The codes can be reported using the argument \ccross{data}
                of the command \poycommand{report}. The codes are generated anew
                when a given data file is reloaded; therefore, they can effectively be used
                only within a current \poy session.}
                {}

            \argumentdefinition{files}{\obligatory{(\poystring list)}}
                {Specifies the filename list containing lists of terminals or
                characters.}
                {}

            \argumentdefinition{missing}{\obligatory{\poyint}}
                {Selects terminals or characters to be included in the analysis
                based on the proportion of missing data. The
                integer value sets the minimum percentage of missing
                data to be allowed in the analysis. Terminals or characters that have \emph{more} missing data
                than the value are included in the analysis.}
                {}
               
            \argumentdefinition{not missing}{\obligatory{\poyint}}
                {The complement of the previous argument.}
                {notmissing}

                \begin{statement}
                For dynamic homology characters, the missing data refers to
                sequence fragments, whereas for static characters it refers to
                individual matrix positions. Therefore, when excluding
                terminals with missing data, the resulting set of selected
                terminals depends on the character type and might, or
                might not, be identical. For example, if a data file (containing
                sequences corresponding to a single fragment) includes
                a very short sequence, this sequence is not treated as
                missing data regardless of its length. This is because in the
                context of dynamic homology a fragment, rather than an
                individual nucleotide position, constitutes a character.
                On the other hand, if the same data are treated as static characters,
                the taxon represented by a very short sequence
                might be excluded if the length of the sequence exceeds the
                threshold defined by the value of \poyargument{missing}.
                \end{statement}

            \argumentdefinition{static}{}
                {Specifies the static homology characters.}
                {}

            \argumentdefinition{dynamic}{}
                {Specifies the dynamic homology characters.}
                {}

            \argumentdefinition{not names}{\obligatory{(\poystring list)}}
                {Specifies the characters or terminals other than those the
                names of which are listed in the string list.}
                {notnames}

            \argumentdefinition{not codes}{\obligatory{(\poystring list)}}
                {Specifies the characters or terminals other than those the
                codes of which are listed in the string list.}
                {notcodes}

          
        \end{argumentgroup}

		\begin{argumentgroup}{Select trees}
			{The following arguments are used to select trees from the pool of trees currently in memory.}

			\argumentdefinition{optimal}{}
				{Selects all trees of minimum cost.} 
                			{}
			
            \argumentdefinition{best}{\obligatory{\poyint}}
				{Selects the number of best trees specified by the integer value.
				Best trees are not equivalent to optimal trees because best trees
				can include suboptimal trees in case the value of
				\poyargument{best} exceeds the number of optimal (minimal-cost)
				trees. If the number of optimal trees exceeds the value of
				\poyargument{best}, only a subset of optimal trees (equal to the
				value of \poyargument{best} is selected in an unspecified order).} 
                			{}
	
	\begin{statement}
               There is no special command in \poy to clear trees from memory. However,
               selecting zero best trees using the command \poycommand{select(best:0)}
               effectively removes all trees currently stored in memory.
          \end{statement}
            
            \argumentdefinition{within}{\obligatory{\poyfloat}}
                {Selects all optimal and suboptimal trees the costs of which do not exceed
                the current optimal cost by the float value. For example, if the current
                optimal cost is 507 and the float value of \poyargument{within} is
                \texttt{3.0}, all trees with costs 507--510 are selected.} 
                {}

            \argumentdefinition{random}{\obligatory{\poyint}}
	{Randomly selects the number of trees specified by the integer
	value irrespective of cost.} 
                {}

	\argumentdefinition{unique}{}
	{Selects only topologically unique trees (after collapsing zero-length
	branches) irrespective of their cost.} 
                {}

		\end{argumentgroup}
	
	\end{arguments}
	 	 	 	 	  
    \poydefaults{unique, optimal}
        {By default \poy selects all unique trees of optimal (best) cost. The rest of
        the trees are removed from memory.}

	\begin{poyexamples}
        
        
        \poyexample{select(terminals,names:("t1", "t2", "t3", "t4", "t5"), \\ 
            characters, names:("chel.aln:0"))}
            {This command selects only terminals \texttt{t1},  \texttt{t2},
             \texttt{t3},  \texttt{t4}, and  \texttt{t5} and use data only from the
              fragment  \texttt{0} contained in the file \texttt{chel.aln}.}
	
	\poyexample{select(terminals, missing:50)}
	{This command excludes from subsequent analyses all the terminals that
	have more than 50 percent of characters missing. The list of included and excluded
	terminals is automatically reported on screen.}
	
	\poyexample{select(optimal)}
            {Selects all optimal (best cost) trees and discards suboptimal trees from
            memory. The pool of optimal trees might contain duplicate trees (that can
            be removed using \poyargument{unique}).}
            
	\poyexample{select(unique, within:2.0)}
	{This command selects all topologically unique optimal and suboptimal trees
	the cost of which does not exceed that of the best current cost by more than
	2. For example, if the best current cost is 49, all unique trees that fall within
	the cost range 49--51 are selected.}
	
	\end{poyexamples}

	\begin{poyalso}
        \cross{characters}
        \ncross{transform}{transformcommand}
	\end{poyalso}

\end{command}

\begin{command}{set}{}

	\syntax{\obligatory{(\optional{argument list})}}

	\begin{poydescription}
            Changes the settings of \poy. This command performs diverse auxiliary 
            functions, from setting the seed of the random number generator to
            selecting a terminal for rooting output trees.
            
            There is no default setting for \poycommand{set} and the order of its
            arguments is arbitrary.
            
	\end{poydescription}

	\begin{arguments}

        \begin{argumentgroup}{Application settings}
            {Some generic application settings. Have no effect in the analyses
            themselves.}

            \argumentdefinition{history}{\obligatory{\poyint}}
                {Sets the size of the \poy output history displayed in the
                \emph{POY Output} window to the number of lines specified by the
                integer value. The size of the history must be greater than
                zero. This command has effect only in the ncurses interface. The
                default size of the output history is 1000 lines.}
                 {}

            \argumentdefinition{log}{\obligatory{\poystring}}
                {Directs a copy of a partial output to the file specified by the
                string  argument. The output includes the  information in the
                \emph{POY Output}, \emph{Interactive Console}, and \emph{State
                of Stored Search} windows of ncurses interface.  Timers and
                current state of the search are not included in the log. If the
                log
                file already exists, \poy will append the text to it; if the log
                file does not exist, then \poy creates a new file. If the user
                would like to delete the contents of a pre-existing file, then
                the argument \poyargument{log:new:"logfile"} creates a new
                initially empty file named \texttt{logfile}.}
                {log}

            \argumentdefinition{nolog}{}
                {Stops outputting the log to any previously selected
                file. See the description of the argument \poyargument{log}
                above.}
                {}

            \argumentdefinition{root}{\obligatory{\poylident}}
                {Specifies the terminal to root output trees.
                 The terminal can either be indicated as a taxon name (a
                \poystring, which must appear in quotes, such as
                \texttt{"Genus\_species"}) or the code, that is automatically
                assigned to the taxon by \poy at the beginning of each \poy
                session (for example, \poycommand{set(root:45)}. The codes can
                be obtained using the command \poycommand{report(data)}).  The
                terminal codes, however, are unique only within a current
                session.}
                {root}

            \argumentdefinition{timer}{\obligatory{\poyint}}
                {Specifies the lapse of time in seconds that should have passed
                between reporting the total execution time of a swap and build
                command. If the timer is set to 0, then no time messages are
                generated.}{}

        \end{argumentgroup}

        \begin{argumentgroup}{Cost calculation}
            {These arguments set the tree cost estimation routines and
             are applied to all character types. The arguments
            are mutually exclusive: only the argument of 
            \poycommand{set} specified last is used.}

            \argumentdefinition{normal\_do}{}
                {Applies a standard Direct Optimization algorithm for the tree
                cost estimation. This is the default and fastest technique.}
                {normaldo}

            \argumentdefinition{exhaustive\_do}{}
                {Applies a standard Direct Optimization algorithm for the tree
                cost estimation~\cite{wheeler1996,wheeler2002a}. The difference with \poyargument{normal\_do} is
                that the calculation of the tree costs during a search is much
                more intense, always looking for the best possible alignment 
                for every single topology (instead of a lazy and greedy strategy
                used by the \poyargument{normal\_do}).}
                {exhaustivedo}

                \argumentdefinition{iterative}{\obligatory{\poylident}}
                {Applies the Iterative Pass optimization for the tree cost
                calculations. There are two forms of iterative pass: if the
                argument value is \poyargument{exact} (the default), then a complete three
                dimensional alignment is computed, as described for POY 3.0 in~\cite{wheeler2003a}. 
                Otherwise, if the argument value is \poyargument{approximate}, then the iterations
                approximate the three dimensional alignment using pairwise
                alignments. 
                For chromosome characters, 
                the three dimentional alignment for three chromosomes
                is computed by separating into smaller three dimentional
                alignments of short sequences whose lengths are smaller 
                or equal to \poyargument{max\_3d\_len}. The default value
                is \texttt{200}.

                If the argument value is \poyargument{exact}, this method improves the tree
                cost estimation but at the expense of a tremendous execution
                time. When approximate the execution time footprint is much
                smaller, and far less memory consumption.
                A heuristic strategy is to apply  \poyargument{iterative} at the 
                very end of an analysis to polish the final set of trees and perform a final search. 
                
                Both arguments accept an optional integer, stating the maximum
                number of iterations that can be performed. If no integer is
                given, then the procedure iterates until no further tree cost
                improvement can be made.}
                {}
                
	\begin{statement}
  	  Due to the complexity of heuristics of the iterative pass optimization, there is no
	  guarantee that the tree cost recovered from the search would be exactly the same
	  as produced by the diagnosis of the same tree. However, the cost of the tree found during
	  the search can be verified by outputting the medians from the diagnosis
      (see the description of the argument \nccross{diagnosis}{diagnosis})
      of the command \poycommand{report} and determining edge costs by hand. The cost of the tree found during the search might
	  differ from that obtained by the rediagnosing the same tree (see \nccross{rediagnose}{rediagnose}), but will
	  recover the same tree cost in subsequent rediagnoses.

  \end{statement}
	
        \end{argumentgroup}

        \begin{argumentgroup}{Randomized routines}
            {}

            \argumentdefinition{seed}{\obligatory{\poyint}}
                {Sets the seed for the random number generator using the integer 
                value. If unspecified, \poy uses the system's time as seed.}
                {}
	\begin{statement}
  	 It is critical to set a seed value to insure reproducibility of the results
	 of the analyses that require randomization routines (such as tree
	 building).
	\end{statement}
	
            \end{argumentgroup}
	\end{arguments}

    \poydefaults{history:1000, normal\_do}{Under default settings the size of
    the history buffer is limited to 1,000 lines, the Direct Optimization is used for
    tree cost calculation, and the current time is used to specify the seed.}

    \begin{poyexamples}
    	\poyexample{set(history:1500, seed:45, log:"mylog.txt")}
            {This command increases the size of the history in the ncurses
            interface to 1,500 lines, sets the random number generator to 45,
            and initiates a log file \texttt{mylog.txt}, located in the current
            working directory.}
            
            \poyexample{set(root:"Mytilus\_edulis")}
            {This commands selects terminal \texttt{Mytilus\_edulis} as a root
            for the output trees.}
            
            \poyexample{set (iterative:exact)}
            {Turns on the iterative exact algorithm in all the nucleotide
            sequence characters. The program will iterate on each vertex of the
            tree until no further tree cost improvements can be made.}

            \poyexample{set (iterative:approximate:2)}
            {Turns on the iterative approximate algorithm in all the nucleotide
            sequence characters. The program will iterate either two times, or
            until no further tree cost improvements can be made, whichever
            happens first.}

            \poyexample{set (iterative:exact:2)}
            {Same as the previous, but using the exact algorithm insted.}
     \end{poyexamples}
     
	\begin{poyalso}
		\cross{report}
	\end{poyalso}

\end{command}

\begin{command}{store}{}

	\syntax{\obligatory{(\obligatory{\poystring})}} 

	\begin{poydescription}
            Stores the current state of \poy session in memory. The stored information
           includes character data, trees, selections, \emph{everything}. Specifying the
           name of the stored state of the search (using the string argument)  does
           \emph{not}, however, generate a file under this name that can be examined;
           the name is used only to recover the stored state using the command \poycommand{use}.

            In combination with \poycommand{use}, the command \poycommand{store}
            is extremely useful when exploring alternative  cost regimes and terminal sets
            within a single \poy session.
	\end{poydescription}
	
	\begin{arguments}
		\argumentdefinition{STRING}{}
                {Specifies the name of the stored search state of the current \poy session.}
                {}
	\end{arguments}

    \begin{poyexamples}
        \poyexample{store("initial\_tcm") \\ transform(tcm:(1,1)) \\ use("initial\_tcm")}
            {The first command, \poycommand{store}, stores the current
            characters and trees under the
            name \texttt{initial\_tcm}. The second command,
            \poycommand{transform}, changes the cost regime of molecular characters,
            effectively changing the data being analyzed. However, the third
            command, \poycommand{use}, recovers the initial state stored under the
            name \texttt{initial\_tcm}.}
    \end{poyexamples}

    \begin{poyalso}
        \cross{use} 
        \cross{transform}
    \end{poyalso}

\end{command}

\begin{command}{swap}{swapcommand}

	\syntax{\obligatory{(\optional{argument list})}}

	\begin{poydescription} 
            \poycommand{swap} is the basic local search function in \poy. This
            command implements a family of algorithms collectively known in systematics as
            branch swapping and in combinatorial optimization as hill climbing. They proceed by clipping parts of a given tree and
            attaching them in different positions.  It can be
            used to perform a local search from a set of trees loaded in memory.

            Swapping is performed on all trees in memory. During a search,
            \poycommand{swap} can collect information about the
            visited trees and perform various kinds of checkpoints to reduce
            information loss in case if \poy crashes.

            \poycommand{swap} is also used as an argument for other
            commands to specify a local search strategy in other contexts, for example,
            in calculating support values using the command
            \nccross{calculate\_support}{calculatesupport}.
            
            All arguments of \poycommand{swap} are optional and their order
            is arbitrary. The argument of different groups can be combined to tune the search heuristics, but the arguments within each group of are mutually exclusive. (If more than one arguments of one group is listed, only the last one is executed.)
            
        \end{poydescription}

	\begin{arguments}

	\begin{argumentgroup}{Neighborhood}
	{A neighborhood is a subset of topologies reachable from a given one by a given search method. The basic standard procedures for local search in phylogenetic analysis are SPR and TBR~\cite{swofford1990}. The nearest-neighbor interchanges (NNI) swapping strategy is implemented by combining 
            the arguments \poyargument{spr} and \poyargument{sectorial} (see \emph{Join method} group of
            arguments): \poycommand{swap(spr, sectorial:1)}.
            \index{general}{NNI|see{swap}}
            \index{general}{nearest-neighbor interchanges|see{swap}}}
            \label{swap_neigh}

	      \argumentdefinition{alternate}{}
	     {Performs \poyargument{spr} and \poyargument{tbr}
                swapping iteratively until a local optimum is found.
                This is a specific strategy of performing \poyargument{tbr},
                as the trees visited by \poyargument{spr} are a subset
                of those visited by \poyargument{tbr}.}
                {}

            \argumentdefinition{spr}{\optional{once}}
	{This argument performs \poyargument{spr} swapping, starting
                from the current trees in memory and subsequently repeating
                the SPR procedure until  a local optimum is found. If the optional value
                \poyargument{once} is specified, \poyargument{spr} 
                stops once the first tree with better cost is found.} 
                {}

            \argumentdefinition{tbr}{\optional{once}}
	      {This argument performs \poyargument{tbr} swapping, starting
                from the current trees in memory and subsequently repeating
                the TBR procedure until  a local optimum is found.  If the optional value
                \poyargument{once} is specified, \poyargument{tbr} 
                will stop once the first tree with better cost is found.}
                {}

        \end{argumentgroup}

    \begin{argumentgroup}{Trajectory}
        {The following arguments define the direction of the search in the defined
        neighborhood.}

        \argumentdefinition{around}{}
            {Changes the trajectory of a search by
            completely exploring the neighborhood of the current
            tree in memory and choosing the best swap position
            before continuing.
            The default in \poy is to choose the first one
            available that shows a better cost than the current
            best cost.}
            {}

        \argumentdefinition{annealing}{\obligatory{(\poyfloat, \poyfloat)}}
            {Uses simmulated annealing~\cite{Kirkpatrick1983}. If the argument's value is $(a, b)$, 
            \poy accepts a tree with cost $c$ when the best known tree has
            cost $d$ with probability $\exp{(- (c - d) / t)}$, where
            $t = a \times \exp{- i / b}$ and $i$ is the number of tree
            evaluated in the local search.}
            {}

        \argumentdefinition{drifting}{\obligatory{(\poyfloat, \poyfloat)}}
            {Uses \poy drifting function~\cite{goloboff1999}. If the argument's value is
            $(a, b)$, then \poy always accepts a tree with better cost than
            the current best cost, with probability $a$ a tree with equal cost,
            and with probability $1 / b + d$ a tree with cost $d$ greater
            than the current best cost.}
            {}

	\end{argumentgroup}

	\begin{argumentgroup}{Branch break order}
		{During the local search, a branch is broken and local branch swapping
		is performed (see \emph{Neighborhood} group of arguments), the
		precise choice of which branches
        should be broken first can affect both the speed and the local
        optimum found by the program. The following arguments select among
        the different strategies available in \poy.}
        
        \argumentdefinition{once}{}
            {Breaks each branch only once during a local search; that is, if a
            broken branch does not yield a better tree, it is never broken again,
            no matter how many changes occur along the search trajectory.}
            {once}

        \argumentdefinition{randomized}{}
            {Chooses branches uniformly at random for breakages.}
            {}

        \argumentdefinition{distance}{}
            {Gives higher priority to those branches with the greatest length.}
            {}

    \end{argumentgroup}

    \begin{argumentgroup}{Join method}
        {After breaking a tree (using SPR or TBR), the following 
        arguments control the selection of the positions to join the broken
        clades.}

            \argumentdefinition{constraint}{\optional{\poyint | 
            (depth:\poyint, file:\poystring)}}
                {Sets
                constraints on the join locations during the search using both a tree and an optional maximum distance
                from the break branch. Only sets defined either in the input
                file, or in the strict consensus of the files in memory to consider during swapping. An integer value of
                \poyargument{depth} specifies the maximum distance from the
                break branch to attempt joins. The string value for
                \poyargument{file} specifies an input file containing a single
                tree that defines topological constraints. Under default settings,
                \poyargument{constraint}
                will use a consensus tree from the files in memory and perform
                swapping with the value of \poyargument{depth} set to \texttt{0}
                (no maximum distance is specified).}
                {swapconstraint}

            \argumentdefinition{all}{\optional{\poyint}}
                {Turns off all preference strategies to make a join, simply try
                all possible join positions for each pair of clades generated
                after a break, in a randomized order.}
                {}

            \argumentdefinition{sectorial}{\optional{\poyint}}
                {Join in edges at distance equal or less than the value of the argument
                from the broken edge, where the distance is the number of edges
                in the path connecting them. If no argument is given, then no
                distance limit is set.}
                {}
		
    \end{argumentgroup}

    \begin{argumentgroup}{Reroot order}
        {During TBR, the following options control the order of the rerooting.}

        \argumentdefinition{bfs}{\optional{\poyint}}
            {Reroots using breath first search~\cite{cormen2001} from the broken edge, within the
            arguments value distance from the root of the clade. If no value is
            given, there is no limit distance for the rerooting. By default, \poyargument{bfs}
            is used with no limit distance for the rerooting.}
            {}

    \end{argumentgroup}

	\begin{argumentgroup}{Trajectory samples}
	{During the search, \poy visits a large number of trees. For some applications
	 it might be desirable to  collect information about the trees examined during a
	 search: for example, to provide backups of the state of a
        search (in an unlikely crash), or to examine the characteristics of the alignments.
        The difference from the \poycommand{swap} arguments is that the user can
        choose any combination of trajectory samples, and that can be 
        used during the search. None of the trajectory samples is used by
        default.}

	\argumentdefinition{recover}{}
	{Stores the current best tree in memory that can be recovered in
            case of failure. If it is necessary to recover such
            trees after an aborted command, use ~\ccross{recover}. 
            If the program terminates normally, the stored trees are exactly
            those produced at the end of the \poycommand{swap}. Using
            \poyargument{recover}, however, requires twice as much memory
            compared to swapping without it.}
            {recoverarg}

     \argumentdefinition{timeout}{\obligatory{\poyint}}
            {Specifies the number of seconds after which tree branch
            swapping is stopped. The current best tree is the result of the
            swap after the timeout.} 
            {}

        \argumentdefinition{timedprint}{\obligatory{(\poyint, \poystring)}}
	{\poyargument{timedprint:(n, "trees.txt")} prints the current
            best tree in memory to the file \texttt{trees.txt}, at least every 
            \texttt{n} seconds. However, \poy typically underestimates the amount of
            time and, therefore, the samples can be slightly sparcer. \poyargument{timedprint} 
            can only be used in combination with the argument \poyargument{recover}.}
            {}

		\argumentdefinition{trajectory}{\optional{\poystring}}
			{\poyargument{trajectory:"better.txt"} will store every new tree
                found with a better score during the local search in the file
                \texttt{better.txt}. The string is the filename where the
                trajectory is to be stored, which is optional (indicated by
                brackets); if not added, the trees are printed in the standard
                output (flat interface) or the output window (ncurses
                interface).}
            {} 

        \argumentdefinition{visited}{\optional{\poystring}}
            {\poyargument{visited:"visited.txt"} will store every visited tree
            and its cost during the local search in the file
            \texttt{visited.txt}. The (optional) string is the filename where the
            trajectory is to be stored. If not included, the trees are printed
            in the standard output (flat interface) or the output window (ncurses
            interface).}
            {}
    
    \end{argumentgroup}

    \begin{argumentgroup}{Character transformation}
    	{}
	
	\argumentdefinition{transform}{}
            {Specifies a type of character transformation to be performed
            \emph{prior} to swapping.
            See the command~\nccross{transform}{transformcommand} for
            the description of the methods of character type transformations
            and character selection.}
            {}
    \end{argumentgroup}
            
    \begin{argumentgroup}{Tree selection}
        {As the tree search proceeds, a tree may or may not be selected to
        continue the search or to return as a result. The following
        arguments determine under what conditions can a tree be acceptable
        during the search.}

        \argumentdefinition{threshold}{\obligatory{\poyfloat}}
            {Sets the percentage cost for suboptimal
            trees that are more exhaustively evaluated during the swap,
            meaning that trees within the threshold are subject to an extra
            round of swapping. For example, if the current
            optimal tree has cost 450, and \poyargument{threshold:10} is specified, trees
            with cost at most 495 are swapped.  \poyargument{threshold} is
            equivalent to \emph{slop} of  \texttt{POY3}.}
            {thresholdswap}

        \argumentdefinition{trees}{\obligatory{\poyint}}
            {Maximum number of best trees that are retained in a search round,
            per tree in memory.}
            {treesswap}

    \end{argumentgroup}
    
	\end{arguments}

    \poydefaults{trees:1, TBR, threshold:0, bfs}{By default, current trees are
    submitted to a round of TBR using breadth first search and one best tree per starting tree is kept.}

	\begin{poyexamples} 
		\poyexample{swap()}
            {This command performs swapping under default settings.}

		\poyexample{swap(trees:5)}
            {Submits current trees to a round of SPR followed by TBR. It keeps
            up to 5 minimum cost trees for each starting tree.}

		\poyexample{swap(transform((all, static\_approx)))}
            {Submits current trees to a round of SPR followed by TBR, using
            static approximations for all sequence characters.}
            
		\poyexample{swap(trees:4, transform((all, static\_approx)))}
            {Submits current trees to a round of SPR followed by TBR, using
            static approximations for all characters, keeping up to 4 minimum
            cost trees for each starting tree.}
            
        \poyexample{swap(constraint:(depth:4))}
            {Calculates a consensus tree of the files in memory and uses it as
            constraint file, then joins at distance at most 4 from the breaking
            branch. This is equivalent to \texttt{swap(constraint:(4))}.}
	
        \poyexample{swap(constraint:(file:"bleh"))}
            {Reads the tree in file \texttt{bleh} and use it as constraint for the
            search. This is equivalent to \texttt{swap(constraint:("bleh"))}.}	
		
        \poyexample{swap(constraint:(file:"bleh", depth:4))}
            {Uses the tree in the file \texttt{bleh} as a constraint tree and joins at
            distance at most 4 from the breaking branch during the swap.}
        
         \poyexample{swap(recover, timedprint:(5, "timedprint.txt"))}
            {Saves the current best tree to file \texttt{timedprint.txt} every 5 seconds.}
            
	\end{poyexamples}

	\begin{poyalso}
        \ncross{transform}{transformcommand}
	\end{poyalso} 

\end{command}

\begin{command}{transform}{transformcommand}

	\syntax{\obligatory{(\optional{argument list})}} 

	\begin{poydescription} 
            Transforms the properties of the imported characters from one type into
            another type. This includes changing in costs for indels and substitution,
            modifying character weights, converting dynamic into static homology characters,
            and transforming nucleotide into chromosomal characters
            among other operations.
            
            The essential arguments of the command
            \poycommand{transform} include identifiers and methods. The methods
            specify what type of transformation is applied to the set of characters
            specified by identifiers as defined in the description of the command~\ccross{select}.
            Identifiers and methods are included in parentheses and separated by
            a comma. It is important to remember that only identifiers of
            \emph{characters} (such as \poyargument{names}, \poyargument{codes}, among
            others) can be used. The parentheses separate these essential
            arguments from all other optional arguments that might be included
            in the list. Thus, if only identifiers and methods are specified,
            the argument list of \poycommand{transform} is included in double
            parentheses. For example, the command \poycommand{transform((all,
            gap\_opening:1))} contains only an identifier (\poyargument{all}) and a
            method (\poyargument{gap\_opening}).  Minimally, only methods can be
            specified; in that case, the transformation is applied to all
            characters to which the transformation method can be applied and only a
            single set of parentheses is used. For instance,
            \poycommand{transform(gap\_opening:1)}, where
            \poyargument{gap\_opening} defines
            the transformation method.

            There are no default values for \poycommand{transform}, that is if
            no methods are specified (\poycommand{transform()}), the command does nothing.
	\end{poydescription}

	\begin{arguments}
	
        \begin{argumentgroup}{Identifiers}
            Identifiers specify which characters are transformed. Only
            identifiers of characters (\emph{not} terminals) can be used. If
            identifiers are omitted, the transformation to is applied to all
            applicable characters. For example,
            \poycommand{transform((all,tcm:(1,1)))} is equivalent to
            \poycommand{transform((tcm:(1,1)))}. See the command~\ccross{select}
            for detailed description of identifiers.
        \end{argumentgroup}
           
           \begin{argumentgroup}{Methods}
            This set of arguments specifies different transformations that can be applied
            to selected characters. If multiple transformation methods are applied
            sequentially in the same list of arguments, the effect of the methods listed
            earlier might be altered or canceled by methods listed after that. Thus, caution
            must be used in designing complex strategies with multiple character
            transformations. See the note on command order (Section~\ref{commandorder}).

        \argumentdefinition{auto\_static\_approx}{}
            {Evaluates each selected fragment and, if the number of indels
            appear to be low and stable between topologies, then the character
            is transformed to the equivalent character using static homologies
            with the implied alignment~\cite{wheeler2003}.
            This method greatly accelerates searching and is applicable only to nucleotide sequences under dynamic homology analysis.}
            {autostaticapprox}

        \argumentdefinition{auto\_sequence\_partition}{}
            {Evaluates each fragment and if a long region appears 
            to have no indels, then the fragment is broken inside that region.
            Any number of partitions can occur along a fragment. Fragmenting
            long sequences greatly accelerates searching. This method is
            applicable only to dynamic homology characters, and requires a tree
            in memory.}
            {autosequencepartition}

        \argumentdefinition{direct\_optimization}{}
          {Transforms the characters specified so that the initial assignment of sequences to 
          the internal vertices of a tree use direct
          optimization~\cite{wheeler1996}. This method is recommended for small
          alphabets (less than 7 elements). Otherwise fixed states is
          recommended. It is only applicable to dynamic homology
          characters.}
          {directoptimization}

        \argumentdefinition{fixed\_states}{}
          {Transforms the characters specified in fixed state characters~\cite{wheeler1999a}
          where the initial assignment of sequences to the internal vertices of
          the tree is one of the observed sequences. If the observed sequences
          contain ambiguities, only those that resolve closest to another
          sequence are added to the set of valid states. This method is
          recommended for large alphabets (more than or equal to 7 elements). It is only
          applicable to dynamic homology characters.}
          {fixedstates}
          
        \argumentdefinition{gap\_opening}{\obligatory{\poyint}}
            {Sets the cost of opening a block of gaps to the specified value. Note that
            this cost is in addition to the standard cost of the insertion as
            specified by a given transformation cost matrix.
            The default in \poy is not to have extension
            gap cost (\poyargument{gap\_opening:0}). If the gap
            opening cost is
            $a$, and $indel(x)$ is the cost of inserting (or deleting) a
            base $x$ according to the tcm assigned to the character, the total
            cost of inserting (or deleting) the sequence $s[0...n]$ is $a +
            indel(s[0]) + indel(s[1]) + ... + indel(s[n - 1]) + indel(s[n]).$
            This method is applicable only to dynamic homology characters with
            the nucleotide alphabet.} 
            {gapopening}

        \argumentdefinition{multi\_static\_approx}{}
            {Calculates the implied alignment for each tree in memory
            and convert them to static homology characters using the alignment's
            cost regime. The new character set will be the union of all those
            characters generated for all the trees~\cite{wheeler1995a}. This option is intended only
            for heuristic search purposes and is applicable only to dynamic homology characters.}
            {multistaticapprox}

            \argumentdefinition{prealigned}{}
                {Treats the sequences as prealigned and uses the
                cost regime according to the specified transformation cost
                matrix. All other cost parameters are ignored (including affine
                gap costs). This command requires that all the specified sequences have the same length.}
                {prealignedtransform}
            
        \argumentdefinition{sequence\_partition}{\obligatory{\poyint}}
            {Partitions the sequences in the argument's value number of
            fragments of roughly the same length. This method is applicable only to dynamic homology characters.}
            {sequencepartition}

        \argumentdefinition{static\_approx}{\optional{\poylident}}
            {Transforms the sequences to the static homology characters
            corresponding to their implied alignments and their transformation
            cost matrix~\cite{wheeler2003}. The resulting characters and their number will vary
            depending on the characteristic of transformation coGst matrix
            assigned to each sequence. For example, if the cost of both substitutions
            and indels is \texttt{1}, then one non-additive character is created per
            each homologous position in the implied alignment. If the cost of
            substitutions is \texttt{1} and the cost of indels  is \texttt{2}, then
            one character is created for each homologous position, and one extra character for
            each homologous position with gaps. In more complex cases, a Sankoff character is
            created.
            
            The LIDENT value \texttt{remove} excludes all uninformative characters
            information (except autapomorphies), whereas the value \texttt{keep}
            retains these characters. The default is \texttt{remove}. This
            method is applicable only to dynamic homology characters. If
            a non-metric transformation cost matrix is in use, this
            transformation will assume that the non-metricity is due to the
            individual insertion and deletion cost.}
            {staticapprox}
            
            \begin{statement}
  	  The transformation of dynamic into static homology characters cannot be reverted.
	  Therefore, caution must be taken when the transformation is applied. For example,
	  if sequence characters have been transformed into static characters at top level using
	  the command \poycommand{transform((all, static\_approx))}, all commands executed 
	  subsequently will be applied to the transformed data. However, if the transformation has 
	  been applied \emph{within} another command (as an argument of \poycommand{swap}, 
	  for instance, \poycommand{swap(transform((all, static\_approx)))}), the characters will transformed only for that specific operation.
	   \end{statement}
	
	\begin{statement}
  	  It is important to remember that the local optimum for the dynamic homology
	  characters can differ from that for the static homology characters based on the
	  same sequence data. Therefore, performing additional searches on the transformed
	  data (for example, in calculating support values based on individual nucleotides
	  rather than on sequence fragments) can produce a discrepancy in tree costs.
	\end{statement}

             \argumentdefinition{tcm}{\obligatory{(\poyint, \poyint)}}
            {Defines transformation cost matrix. The first integer value specifies
            substitution cost, the second integer value defines indel cost. By default,
            the cost of substitution is \texttt{1}, and the cost of an indel is \texttt{2}
            (\poyargument{tcm:(1,2)}).}
            {transformtcm}

            \argumentdefinition{tcm}{\obligatory{\poystring}}
            {Defines the transformation cost matrix by importing a file (specified by
            the string value) that contains a user defined nucleotide
            transformation cost matrix. This method is applicable only to dynamic homology characters.
            The transformation cost matrix file contains five rows and columns
            with values listed in the following order (left to right and top to
            bottom): adenine, cytosine, guanine,
            thymine/uracil, and indel.  A similar pattern is followed for amino acids
            where the matrix columns and rows reflect all the amino acid names in alphabetical order
            (read left to right and top to bottom) with the last row and column containing a gap cost. 
            The costs must be symmetrical (that is, the
            cost of the A to T substitution is equal to the cost of T to A
            substitution). For example:
	        \begin{center}
            \texttt{0 2 1 2 4 \\
            2 0 2 1 4 \\
            1 2 0 2 4 \\
            2 1 2 0 4 \\
            4 4 4 4 0} 
            \end{center}
            }
            {transformtcmmatrix}

            \argumentdefinition{weight}{\obligatory{argument}}
            {Changes the cost of specified characters by a
            constant value (weight) which is specified by either a
            float or an integer value. This method is applicable to any character type.} 
            {weight}

            \argumentdefinition{weightfactor}{\obligatory{argument}}
            {Changes the cost of specified characters by a
            multiplicative factor (weight factor) which is specified by either a
            float or an integer value. This method is applicable to any characters.} 
            {weightfactor}

      \end{argumentgroup}
           
       \begin{argumentgroup}{Chromosomal transformation methods}
           For chromosome and genome character types, \poy optimizes nucleotide-, 
           locus-, and chromosome-level variation simultaneously. The arguments in this group
           transform nucleotide characters into chromosomal character
           to allow for translocations, inversions, and indel events both at the locus-level for chromosomal data
           and at the chromosome-level for genomic data.
           
          The functions to calculate breakpoint and inversion distances between two
	sequences of gene orders are taken from GRAPPA, Genome
	Rearrangements Analysis under Parsimony and other Phylogenetic Algorithms ~     \cite{baderetal2002},
	available at \texttt{http://www.cs.unm.edu/\~{}moret/GRAPPA/}.
	
	 \argumentdefinition{breakinv\_to\_custom}{}
            {Transforms \poyargument{breakinv} character type into \poyargument{custom\_alphabet} characters.
            This transformation prevents the use of rearrangement operations.} 
            {breakinvtocustom}

           \argumentdefinition{custom\_to\_breakinv}{\obligatory{([argument list])}}
            {Transforms \poyargument{custom\_alphabet} characters into the \poyargument{breakinv} character type to 
            allow for rearrangement operations (translocations and inversions; duplications are not currently supported).  
            This argument is useful, for example, when \poyargument{custom\_alphabet} characters are used to define a sequence of
            individual genes and one is interested in detecting potential change in their order within a chromosome. 
            See the command~\ccross{read} for the description on how to load the \poyargument{custom\_
            alphabet} and \poyargument{breakinv} character types. The optional list of arguments
            includes the arguments of the \poyargument{dynamic\_pam} that can also
            be specified subsequently, as a separate step (for a sample script using this character transformation see tutorial \texttt{4.8}).}
            {customtobreakinv}
             
        \argumentdefinition{seq\_to\_chrom}{\obligatory{([argument list])}}   
           {Transforms nucleotide type data into \\ chromosome type data to allow
            rearrangements, inversions, and locus-level indel operations.  The
            chromosome-specific options (\emph{e.g.}  \poyargument{locus\_breakpoint}, 
            \poyargument{locus\_inversion}, \poyargument{locus\_dcj}, and \poyargument{locus\_indel}) can be specified by the argument
            \poyargument{dynamic\_pam} If no \poyargument{dynamic\_pam} values
           are specified, its default values are applied.}
           {seqtochrom}
           
            \argumentdefinition{dynamic\_pam}{\obligatory{([argument list])}}
            {Specifies parameters for creating chro\-mosome- and genome-level HTUs (medians).
            The arguments of \poyargument{dynamic\_pam} define homologous blocks within unannotated chromosome sequences using
            (\poyargument{min\_seed\_length}, \poyargument{min\_loci\_len}, and \poyargument{min\_rearranged\_len}); specify the cost
            of locus-level transformation events: (i.e.
            \poyargument{locus\_inversion}, \poyargument{locus\_dcj}, or \poyargument{locus\_breakpoint},
            and \poyargument{locus\_indel});  specify the cost of chromosome-level transformation events: 
            (i.e. \poyargument{chrom\_breakpoint}, and \poyargument{chrom\_indel}); 
            take into account whether the chromosome is linear or circular (\poyargument{circular}); and implement a number of heuristic
            procedures to accelerate computations (\poyargument{median}, \poyargument{swap\_med}, and \poyargument{med\_approx}).
            Under default settings, the pairwise distance between two chromosome segments or two chromosomes is determined 
            using breakpoint rather than inversion calculations and the rest of the arguments are executed under their default settings.}
            {dynamicpam}

		 \begin{figure} [!htbp]
   		 \begin{center}
        		\includegraphics[width=0.6\textwidth]{doc/figures/genomeRearrangement.pdf}
    		\end{center}
    		 \caption{Examples of gene rearrangements: inversions and translocations.}
		 \label{fig:genomeRearrangement}
		\end{figure}

		\begin{figure}[!htbp]
		\begin{center}
        		\includegraphics[width=0.4\textwidth]{doc/figures/breakpointDis.pdf}
      		\end{center}
		  \caption{Rearrangement calculations between chromosomal or genomic data of six genes $g_1, \ldots, g_6$,
                 	 where the rearrangement events are detected as either two breakpoints $(g_2, g_3), (g_5, g_6)$
                   or a single inversion $(g_1, g_2, g_3)$.}
                   \label{fig:distance}
		\end{figure}
		
	\begin{description}
            
            	\argumentdefinition{med\_approx}{\obligatory{\poybool}}
                        {Approximates chromosome medians using a fixed-states
                        approach. This is most useful to accelerating tree
                        building and searching operations for large chromosomal
                        data sets. The boolean value \texttt{true} applies the
                        fixed-states optimization. The default value is
                        \texttt{false}.}
                        {medapprox}
                        
                        \argumentdefinition{locus\_breakpoint}{\obligatory{\poyint}}
                        {Calculates the breakpoint distance \cite{blanchetteetal1997}
                        between two pairs of chromosomes given the cost for rearrangement
                        specified by an integer value.  The breakpoint distance calculation considers
                        a chromosome or genome $G = (x_1, \ldots, x_n)$of $n$ gene, wherein each
                        gene appears exactly once and its orientation is either positve or negative.  Gene
                        orders are altered by gene rearrangement operations: gene inversion, gene translocation,
                        gene inversion and translocation (see Figure \ref{fig:genomeRearrangement}).  
                        The breakpoint distance takes into account rearrangements but not inversions.
                        Given $G$ and $G'$, a pair of genes $(g_i, g_j)$ is a breakpoint if $(g_i, g_j)$ occur 
                        consecutively in $G$ but neither $(g_i, g_j)$ nor $(-g_j, -g_i)$ occur
                        consectively in $G'$  \cite{sankoffandblanchette1998}.  The breakpoint distance between $G$
		     and $G^\prime$ is the number of breakpoints between them.  Figure \ref{fig:distance} 
                        shows two breakpoints between $G$ and $G^\prime$. The breakpoint can be calculated 
                        easily in linear time.  This argument \emph{cannot} be used in
                        conjunction with \poyargument{locus\_inversion} or
                        \poyargument{locus\_dcj}. The default
                        value of \poyargument{locus\_breakpoint} is \texttt{10}.} 
                        {locusbreakpoint} 
                        
                        \argumentdefinition{locus\_dcj}{\obligatory{\poyint}}
                        {Calculates the double cut and join
                        distance~\cite{yancopoulos:2005}
                        between two chromosome segments given the cost for each
                        event specified by the integer value. The inversion distance
                        takes in consideration rearrangements and
                        inversions. The dcj distance is normally smaller than
                        the inversion and breakpoint distances.
                        This argument \emph{cannot} be used in conjunction with
                        \poyargument{locus\_breakpoint} or
                        \poyargument{locus\_inversion}.} 
                        {locusdcj}  

                        \argumentdefinition{locus\_inversion}{\obligatory{\poyint}}
                        {Calculates the inversion distance~\cite{hanenhalliandpevzner1995}
                        between two chromosome segments given the cost for inversion
                        specified by the integer value. The inversion distance
                        takes in consideration rearrangements and
                        inversions. Given $G$ and $G^\prime$, the inversion distance between
                        them is the number of inversions to convert chromosome or genome $G$ 
                        into $G^\prime$ \cite{hanenhalliandpevzner1995}. Figure \ref{fig:distance} shows one inversion  
                        between $G$ and $G^\prime$. The inversion can be calculated in linear time.
                        The breakpoint distance is normally larger than
                        inversion distance and dcj distance.
                        This argument \emph{cannot} be used in conjunction with
                        \poyargument{locus\_breakpoint} or
                        \poyargument{locus\_dcj}.} 
                        {locusinversion}  

		     \argumentdefinition{locus\_indel}{\obligatory{(\poyint, \poyfloat)}} 
                        {Specifies the cost for insertion/deletion of a
                        chromosome segment. The integer value sets the gap opening
                        cost ($o$), whereas the float value sets the gap extension
                        cost ($e$).  The indel cost for a fragment of length $l$ is
                        specified by the following formula:
                       $o + l \times e$. The default values are $o=10, e=1.0$.}
                        {locusindel}

		     \argumentdefinition{min\_seed\_length}{\obligatory{\poyint}}
                        {Specifies the minimum length of identical (invariant,
                        completely conserved) contiguous sequence fragments
                        during comparison between two chromosomes. The integer
                        value of \poyargument{min\_seed\_length} is the number of
                        nucleotides. Correct identification of such fragments
                        facilitates detecting chromosome rearrangement events and
                        accelerates other operations (such as tree building and
                        swapping). However, if \poyargument{min\_seed\_length} value
                        is set too low (detecting many short fragments) or
                        too high (such that no identical fragments are detected)
                        the time required for subsequent searching procedures
                        may significantly increase. The optimal \poyargument{min\_seed\_length} 
                        value depends on the specifics of  a given dataset (see Figure \ref{fig:chrom}).
                        The default value of \poyargument{min\_seed\_length} is \texttt{9}.}
                        {minseedlength}

                          \argumentdefinition{min\_loci\_len}{\obligatory{\poyint}}
                          {Creates a pairwise alignment between two chromosomes to
                          detect conserved areas (``blocks''). However, only blocks of
                          lengths (in number of nucleotides) greater or equal to the specified \poyargument{min\_loci\_len}
                          value are considered as hypothetically
                          homologous blocks and used as anchors to divide chromosomes into
                          fragments. Thus, increasing the value of \poyargument{min\_loci\_len} decreases
                           the chance of inferring small-size rearrangements (see Figure \ref{fig:chrom}). 
                           The default value is \texttt{100}.} 
                           {minlocilen}
                           
                           \argumentdefinition{min\_rearrangement\_len}{\obligatory{\poyint}} 
                            {Two seeds are said to be \emph{non-rearranged}, 
               	           if their distance is less than the predefined threshold value set
	                    for \poyargument{min\_rearrangement\_len}.  In other words, it is unlikely 
	                    that rearrangement operations can occur between two non-rearranged seeds
	                    if they are connected (see Figure \ref{fig:chrom}).
	                    The default value is \texttt{100}}
                             {minrearrangementlen}


                           \argumentdefinition{max\_3d\_len}{\obligatory{\poyint}} 
                            {The three dimentional alignment for three chromosomes
                            is computed by separating into smaller three dimentional
                            alignments of short sequences whose lengths are smaller 
                            or equalt to \poyargument{max\_3d\_len}. The default value
                             is \texttt{200}}
                             {minrearrangementlen}


 			\begin{figure} [!htbp]
   		 	\begin{center}
        			\includegraphics[width=0.8\textwidth]{doc/figures/chromfig1.jpg}
    			\end{center}
    			\caption{The effect of dynamic parameter arguments (\poyargument{seed\_length, min\_rearrangement\_len}, 
			and \poyargument{min\_loci\_len}) on determining homologous blocks in unannotated chromosomal sequences.
			 \emph{Case A} allows for rearrangements between the short homologous blocks within the 				     
			blue brackets constructed upon six nucleotide seeds (red and green), \emph{Case B} 
			restricts rearrangement between seeds that are less than \texttt{28} nucleotides apart.
			 \emph{Case C} restricts rearrangement by requiring homologous blocks at least \texttt{50} nucleotides
			  in length. \emph{Case D} restricts recognition of homology between sequences by setting seed length
			   of conserved seeds at \texttt{15}.}
    		\label{fig:chrom}
		\end{figure}

                       \argumentdefinition{chrom\_breakpoint}{\obligatory{\poyint}} 
      		     {Calculates the breakpoint distance~\cite{blanchetteetal1997}
                        between two sequences of multiple chromosomes given the cost for
                        rearrangement specified by an integer value. The breakpoint distance
                        takes into account locus rearrangements between non-homologous
                        chromosomes (translocations) but not inversions. For further discussion on 
                        how breakpoint distance is calculated see the argument \poyargument{locus\_breakpoint}.  
                        The default value of \poyargument{chrom\_breakpoint} is \texttt{100}.} 
                        {chrombreakpoint}

       
                   \begin{statement}
  		Note that the arguments \poyargument{locus\_breakpoint} and \poyargument {chrom\_breakpoint} cannot be used
		simultaneously with the arguments \poyargument{locus\_inversion} and \poyargument{chrom\_inversion}
		as they designate \emph{alternative} methods of calculating distance between two chromosomes.
		If both arguments are specified, the latter will be executed. The order of other arguments of
		\poyargument{dynamic\_pam} is arbitrary. 
		\end{statement}

		     \argumentdefinition{chrom\_hom}{\obligatory{(\poyfloat)}}
                       {Specifies the lower limit of distance between two chromosomes
                        beyond which the chromosomes are not considered to be
                        homologous. The default value of \poyargument{chrom\_hom}
                        is \texttt{0.75}.}
                        {chromhom}
                        
                         \argumentdefinition{chrom\_indel}{\obligatory{(\poyint, \poyfloat)}}
                        {Specifies the cost for insertion and deletion of a chromosome in analysis of
                        multiple chromosomes. The integer value sets gap opening
                        cost ($o$), whereas the float value sets gap extension
                        cost ($e$).  The indel cost for a fragment of length $l$ is
                        specified by the following formula:
                       $o + l \times e$. The default values are $o=10, e=1.0$.}
                        {chromindel}
 
 	               \argumentdefinition{circular}{\obligatory{\poybool}} 
                        {Specifies if chromosome is circular (boolean value 
                        \texttt{true}) or linear (boolean value \texttt{false}).
                        The default value of \poyargument{circular} is
                        \texttt{false} (linear chromosome).}
                        {circular}

       		      
		      \argumentdefinition{median}{\obligatory{\poyint}}
                        {Specifies the number alternative locus and chromosome
                        rearrangements of the best cost selected (randomly) for
                        each HTU (hypothetical taxon unit) or median. Limiting the number of rearrangements
                        stored in memory (smaller value of \poyargument{median})
                        is heuristic strategy to accelerate calculations at the
                        expense of thoroughness of the search. By default, only 1
                        rearrangement is retained (the first one found). If more than
                        one rearrangement is specified, the selected number of
                        rearrangements is selected in random order from the pool of
                        all generated rearrangements.}
                        {median}

        		      \argumentdefinition{max\_kept\_wag}{\obligatory{\poyint}}
                        {Defines the maximum number of Wagner\-based possible ancestral sequence
                        alignments kept to create the next set of alignments during the pairwise alignment
                        with rearrangement process.  The default value of \poyargument{max\_kept\_wag} is 1,
                        however, at every step in the pairwise alignment with rearrangement process, the original
                        order (1...n) is always considered as a potential solution.}
                        {maxkeptwag}
                        
                        
       		     \argumentdefinition{swap\_med}{\obligatory{\poyint}}
                        {Specifies the maximum number of swapping iterations
                        to search for best pairwise alignment of two chromosomes
                        taking into account locus-level rearrangement events. Limiting the number of swapping
                        iterations accelerates the search at the expense of
                        thoroughness. The default value is \texttt{1}.}
                        {swapmed}

       
          \end{description}
	\end{argumentgroup}
	\end{arguments}
	

    \poydefaults{}{If no arguments are given, this command does nothing.}

	\begin{poyexamples} 
		\poyexample{transform((all, tcm:(1,1)))} 
             	{Applies the transformation cost matrix (1,1) to all characters,
             	meaning that substitutions and gaps receive the same weight.}

		\poyexample{transform((all, tcm:"molmatrix"))} 
           	 {Applies the character transformation matrix "molmatrix" to all
            	characters.}
            	
		\poyexample{transform((all, tcm:(1,1)))}{This command
		is equivalent to \poycommand{transform((dynamic, tcm:(1,1)))}.}
		
		\poyexample{transform(tcm:(1,1), gap\_opening:1)}
		{Applies the transformation cost matrix and the gap opening cost
		to all characters. In this example the cost for substitutions is \texttt{1},
		the gap opening cost is \texttt{2} (\texttt{1} set by \poyargument{gap\_opening}
		+ \texttt{1} set by \poyargument{tcm}), and the gap extension cost is \texttt{1}
		(set by \poyargument{tcm}).}
		
		\poyexample{transform(tcm:(2,2), ti:(1,1,1,1,0), td:(1,1,1,1,0))}
		{Assigns to all characters the symmetric transformation cost
		matrix with cost \texttt{2} for every indel and substitution, but for those
		insertions and deletions at the ends of the sequences, the cost
		assigned will only be \texttt{1}.}
		
		\poyexample{transform((static, weightfactor:2))}
            	{This command reweights all the static homology characters
            	by a multiplicative factor of \texttt{2}, while keeping the weighting
            	scheme that has been specified before.}
		
		\poyexample{transform((static, weight:4.2))}{Applies the same
		weight (a float value \texttt{4.2}) to all static homology characters.}
		
		\poyexample{transform((dynamic, weight:4))}{Applies the same
		weight (an integer value \texttt{4}) to all dynamic homology characters.}

        		\poyexample{transform((all, tcm:(1,1)), (names:("gen1",
        		"gen2"), \\ static\_approx), (names:("gen3"), tcm:"molmatrix"))}  
            	{Applies the substitution and indel costs \texttt{1} to all characters, then applies static approximation
            	using that tcm to characters in files \texttt{gen1} and \texttt{gen2}, and for file
            	gen3, it invokes a different transformation cost matrix, contained
            	in the file molmatrix. Beware that the file name should be exactly
            	as it was reported with \poycommand{report(data)}, which differs from the actual
            	file name (\poycommand{report (data)} reports files as fileX:N).}

        		\poyexample{transform((all, tcm:(1,1)), (names:("gen1:3",
        		"gen2:10", \\"gen3:1", "gen4:5"), static\_approx), (names:("gen5", \\
        		"gen6"), tcm:"Molmatrix1"))}
            	{Applies \poyargument{tcm (1,1)} to all characters, then applies
            	static approximation to the sequence data contained in files \texttt{gen1}, \texttt{gen2},
            	\texttt{gen3}, and \texttt{gen4} according to this transformation cost
            	matrix, and applies the custom transformation cost matrix contained in the file
            	\texttt{Molmatrix1} to the sequence data contained in files \texttt{gen5} and
            	\texttt{gen6}.}
         
         		\poyexample{transform(fixed\_states)}
        		 {Transformed all sequence characters into fixed states characters.}
            
          	\poyexample{transform((names:("gen1", "gen4"), fixed\_states))}
           	{Transformed only specified sequence characters (\texttt{gen1} and
           	\texttt{gen4}) into fixed states characters.}
           
           	\poyexample{transform(custom\_to\_breakinv:(circular:true))}
           	{In this example all custom\_alphabet data is transformed into the breakinv data type 
		and is treated as a circular chromosome.}
          
          	\poyexample{transform(seq\_to\_chrom:(locus\_indel:(50, 1.0), min\_seed\_length:15))} 
          	{All applicable (\emph{i.e.} sequence) data are transformed into chromosome
          	data with the minimum length of identical contiguous sequence fragments which form
	          the seeds of homologous blocks set at  \texttt{15} nucleotides and the locus-level gap
	         opening cost is set at \texttt{50} with a gap extension cost at \texttt{1.0}.}
              
          	\poyexample{transform((all, dynamic\_pam:(locus\_breakpoint:10, max\_kept\_wag:2, min\_rearrangement\_len:60, median:1, circular:false)))} 
             	{This example shows the transformation of chromosomal data using the argument \poyargument{dynamic\_pam}
		to set the locus rearrangement (breakpoint) cost at \texttt{10}, and only strings of \texttt{60} or more nucleotides 
		are considered in determining possible rearrangements between identified seeds.  The chromosome data are
		treated as linear and only a single set of median rearrangements are stored.}
            
	\end{poyexamples}	    

\end{command}


\begin{command}{use}{}

	\syntax{\obligatory{(\poystring)}}

	\begin{poydescription}
         Restores from memory the state of a \poy session (that includes character data,
         selections, trees, all other data and specifications) that had previously been
         saved during the session using the command~\ccross{store}{}. The recalled
         session replaces the current session. The string argument specifies the name
         of the stored state.
         
         In combination with ~\ccross{store}{}, the command \poycommand{use}
         is very useful for exploring alternative  cost regimes and terminal sets
         within a single \poy session.
            
	\end{poydescription}
	
	\begin{poyexamples}
        \poyexample{store("initial\_tcm") \\ transform(tcm:(1,1)) \\ use("initial\_tcm")}
            {The first command, \poycommand{store}, stores the current
            characters and trees under the
            name \texttt{initial\_tcm}. The second command,
            \poycommand{transform}, changes the cost regime of molecular characters,
            effectively changing the data being analyzed. However, the third
            command, \poycommand{use}, recovers the initial state stored under the
            name \texttt{initial\_tcm}.}
    \end{poyexamples}

     \begin{poyalso}
        \cross{store}
        \cross{transform}
    \end{poyalso}

\end{command}

\begin{command}{version}{}

	\syntax{\obligatory{()}}

	\begin{poydescription}
            Reports the \poy version number in the output window of the ncurses
            interface, or to the standard error in the flat interface.
	\end{poydescription}

    \begin{poyexamples}
        \poyexample{version ()}{}
    \end{poyexamples}
\end{command}

\begin{command}{wipe}{}

	\syntax{\obligatory{()}}

	\begin{poydescription}
        Eliminates the data stored in memory (all character data, trees, \emph{etc.}).
	\end{poydescription}

    \begin{poyexamples}
        \poyexample{wipe ()}{}
    \end{poyexamples}
\end{command}
      


% \chapter{\poy Heuristics: a practical guide}
% \section{Introduction}

As the level of phylogenetic analysis increases---from individual loci to chromosomes to genomes containing multiple chromosomes---so does computational complexity. In \poy, a significant increase in computational time results from combining in single process cladogram searching with co-optimiza\-tion of nucleotide pairwise alignments, rearrangements of loci within a chromosome, and rearrangements of chromosome fragments within the genome . As a result, a phylogenetic analysis involves a set of nested computationally ``hard'' (NP-complete) problems that makes finding the exact solution impossible. In addition, the increasing sequence length heterogeneity (at the levels of nucleotides, loci, and chromosomes) and the ever-growing sizes of datasets further contribute to computational complexity making it impossible to obtain an exact solution in a reasonable time.

To circumvent the problem computational intractability, and, hence, the speed of the analyses, \poy employs a battery of approximate, or heuristic, methods that function at different levels of analysis. As with all heuristic procedures, a tradeoff is involved: a substantial decrease in execution time comes at a price of obtaining possibly less accurate  and less precise results (however, the extent of the tradeoff is difficult to evaluate in the analyses of real large datasets). Therefore, it becomes important to understand the combined effect of different heuristic methods, so that the chosen search strategy balances the computational time with a ``reasonable'' accuracy of the result.

Here we provide general guidelines for using different heuristic methods, explore their combined effect, and suggest the choice of parameters that can be explored to provide the best result for specific cases. Real datasets differ greatly in size and complexity, so that no single optimal strategy can be suggested. These guidelines, however, should enable the investigator to design an efficient strategy that will tailor to the peculiarities of a given dataset.

In addition to heuristic methods, this chapter attempts to assist with the selection of transformation cost regimes. Alternative cost regimes can significantly affect the outcome of the analysis, that becomes particularly apparent in dealing with large, genome-level datasets, where multiple cost regimes are used simultaneously to specify transformations at different levels of analysis. Most difficulties stem from selecting the most reasonable combination of parameters that affect optimization of DNA sequence data at the levels of nucleotides, loci, and chromosomes.

\section{Data treatment}

Direct optimization (see \emph{Character optimization} section below) involves comparing all potential nucleotide homologies between two sequences. Consequently, the time it takes is proportional to the product of the lengths of the sequences compared. This procedure can be time consuming for long and greatly differing in length DNA fragments. In cases where unambiguous (such as long completely conserved regions) sequence fragments can be identified, partitioning the long sequences into smaller fragments delimited by these regions can significantly reduce computational time. Such economy is reached because nucleotide homologies are not examined over the separate partitions. This strategy assumes that the fragments are mutually exclusive and are putatively homologous across terminals.

At the level of nucleotides, individual fragments in a locus can be separated by the pound symbols (``\#'') or contained as individual files (that is, treated as partitions). When ``\#'' are used, their number must be the same across homologous sequences. Alternatively, the argument of \poyargument{auto\_sequence\_partition} of the command \ccross{transform}. At the chromosome level, individual loci can be separated by pipes (``$\vert$'').

\begin{center}
\begin{tabular}{| l  l  p{.35\textwidth}|}
	\hline
	Level of analysis & Heuristic & Implementation \\ \hline \hline
	Nucleotides & Fragment sequences & Manually separating fragments or use
	\poycommand{transform (auto\_sequence\_partition)}\\
	Locus & Fragment chromosome & Manually insert pipes separating loci \\
	Chromosomes & NA & NA \\
	\hline	
\end{tabular}
\end{center}

\section{Character optimization}
Minimizing overall cladogram cost is an NP hard problem dependent on the lowest cost assignment of HTU sequences.  POY implements direct optimization (DO; ~\cite{wheeler1996}) and fixed-states optimization (FSO; ~\cite{wheeler1999a}) heuristics to determine the set of HTU sequences comprising the internal nodes of each cladogram constructed.  Direct Optimization decomposes the problem into a series of two-node comparisons, calculating locally optimal solutions, which generates the total cladogram cost.  An advantage of direct optimization is that it allows for the exploration of a large diversity of putative homologies and selects the scheme that yields the most optimal solution. This is useful in analyzing sequences of different length, where site-to-site homologies are uncertain.  Because the procedure is based on a greedy algorithm, it requires multiple iterations (independent initial cladogram builds) and extensive tree searches to reach a potentially global minimum.  In contrast, fixed-states optimization does not calculate HTU sequences but rather optimizes those observed in terminal taxa. These internal node sequences then are diagnosed using dynamic programming based on a matrix of edit costs between sequences.  In the fixed-states implementation cladogram optimization is independent of sequence lengths, and as the number of sequences increase so to does the pool from which the HTU sequences are drawn, thereby improving cladogram cost estimation. Because of these properties fixed-states optimization is recommended as an initial approximation strategy for large data sets of variable length sequences.  

\begin{center}
\begin{tabular}{| l  l  p{.35\textwidth}|}
	\hline
Level of analysis&Heuristic&Implementation \\ \hline \hline
Nucleotides&DO&Default strategy\\
Nucleotides&FSO&\poycommand{transform(fixedstates)}\\
Loci&FSO&\poycommand{transform(dynamic\_pam:(approx))}\\
Chromosomes&NA&NA\\
\hline	
\end{tabular}
\end{center}

Further approximations and economies can be achieved by varying parameters of commands, such as selecting a limited subset of trees for subsequent analysis limiting the number of replicates, and examining intermediary results from an interrupted analysis.

\section{Tree searching}
The heuristic approaches to cladogram searching include random addition of taxa, branch swapping (TBR and SPR), simulated annealing (the ratchet and tree-drifting), and genetical algorithms (tree fusing). These techniques, frequently used in combination, allow a more efficient exploring of tree space and provide the means of finding more globally optimal solutions. These methods are widely used in phylogenetics \cite{felsenstein2004a, wheeleretal2006}, although \poy implements additional modifications of these procedures.

Typical search strategy in \poy involves consecutive application of tree search algorithms that begin with generating multiple, randomly selected starting points [Random Addition Sequences (RAS) or Wagner trees]. The importance of multiple starting trees cannot be overemphasized and a successful search shall maximize the number of RAS. However, making a tree search more exhaustive by increasing the number of starting trees comes at a price of longer computation time. Therefore, it is advised here to estimate the amount of time it takes to complete a single replicate and takes this information in consideration when designing a more exhaustive strategy. The  number of replicates used by \poy practitioners for datasets of moderate size (70-100 terminals) ranges from 100 to 250. Here are some examples of search strategies:
\begin{description}
\item[RAS+SPR/TBR+Ratchet] The strategy is for a thorough search for a data set of 100 or fewer taxa. A diversity of starting points is generated by multiple RAS, each refined by a local search (TBR or a combination of SPR and TBR, the latter is an efficient default strategy in \poy). Ratcheting is used to examine tree space that potentially has not been explored by the local searches.
\item[RAS+SPR/TBR+Ratchet+Tree Fusing]  Adding tree fusing step allows for combining the best sectors of cladograms that can potentially yield a tree of shorter length. Empirical studies showed that adding tree fusing after replicate rounds enhances the results only when dealing with data sets with more than 50 taxa.
\item[RAS+SPR/TBR+Ratchet+Tree Drifting+Tree Fusing] Tree Drifting can be used in place of or in addition to the Ratchet.
\item[Input Trees+SPR/TBR+Ratchet+Tree Drifting+Tree Fusing] For more exhaustive searches, the best trees obtained from the initial searches using the strategies outlines above, can be used as input trees for subsequent analyses. In doing so, the RAS step can be omitted because searching starts with trees approximating the globally optimal tree(s).
\end{description}

The aggressiveness of searches can be adjusted by varying parameters of the branch swapping, ratchet, tree fusing, and tree drifting commands.

Further economies can be reached by using a combination of different character optimization methods. For example, initial searches can be conducted under faster static approximation (that converts sequence data into static homology characters; see \emph{Character optimization} section), whereas the final refinement can be performed using direct optimization.

\section{Chromosomal heuristics}
Analysis of chromosomal data requires heuristic procedures to estimate
rearrangement events in addition to nucleotide transformations. Chromosomal
data are divided into four different classes, namely breakinv, annotated, 
chromosome and genome. In the following we discuss the complexity of
each class.

\subsection{Breakinv character}
Breakinv character is the most simple form of chromosome data. Each breakinv
character (chromosome) is presented by a sequence of general alphabet characters each codes for
one gene. The transformation cost matrix among characters is calculated in advance
and provided by users. During the tree search, \emph{pairwise alignments with rearrangements} (PAR) 
between two breakinv characters is constructed.
The PAR generalizes the ordinary pairwise alignment by allowing rearrangements of character order. 
Since an exact solution for the PAR problem is likely intractable, we developed
a heuristic approach that is a compromise between computational expense and
alignment quality~\cite{vinh2006}. The method is comprised of two phases: first, it
creates an initial PAR using stepwise addition strategy; second, it improves the 
initial PAR by pairwise position swapping techniques. The second step is
repeated several iterations until either no improvement is found or the number
of swap iterations exceeds a user-defined maximum number,
\emph{swap\_med parameter}. 
The runtime complexity of the approaches is O($n^4 \times swap\_med$) where $n$ is 
the number of genes.

\begin{table}[t]
\caption{The influence of \emph{swap\_med} to running time and tree cost
         on a dataset containing 22 taxa with approximate 20 genes}
\label{swapMedComp} 
\begin{center}
\begin{tabular}{l c c}
\hline
	swap\_med & tree cost & time (seconds) \\
\hline
 	   0 		& 954 &   28 \\
 	   1 		& 948 &   52 \\
 	   2 		& 871 &   77 \\
 	   4 		& 882 &   97 \\
 	   8 		& 852 &   102 \\
\hline
\end{tabular}
\end{center}
\end{table}

Table \ref{swapMedComp} shows that the increase of \emph{swap\_med} 
results in increases the runtime. However, it does not guarantee the
improvement of tree cost. The \emph{swap\_med} default is one.


\subsection{Annotated chromosome character}
The annotated character type is a more general presentation of chromosome data 
than the breakinv character. Each annotated character consists of 
a sequence of loci/genes separated by pipes (`` $\vline$ ''). 
This data type allows for locus-level rearrangements as well as
nucleotide transformations. Locus homologies are
determined dynamically, but based on annotated regions~\cite{vinh2006}.
Given that two annotated characters each have $m$ genes, the pairwise alignment method
first calculates pairwise distances
among genes are calculated and then applies the algorithm to reconstruct
the PAR. 
The runtime complexity of the algorithm is O($m^2 \times l^2 + n^4 \times
swap\_med$) where $l$ is the average length of genes. Note that annotated
character does not require characters to have the same number of genes.
Although a large of number equally optimal PARs could be constructed
between two annotated characters, only a user-defined maximum number of PARs, \emph{median},  are kept
during the tree search. Our experience is that the increase of \emph{median}
does not usually result in the improvement of tree cost. The default value of
\emph{median} is one. 

It takes approximately 4 minutes to construct a Wagner tree of 10 taxa each contains
8 genes of length approximate 300 nucleotides. We conducted a 
SPR search on the constructed Wanger tree to examine the runtime and tree
improvement. SPR search takes approximately 13 minutes and reduces 
the tree cost about one percent.



\subsection{Chromosome character}

Unannotated chromosomal sequences are the most general presentation of
chromosome data where each chromosome consists of a long sequence of nucleotides.
To analyze this character type we developed an approach to construct a \emph{comprehensive chromosome pairwise
 alignment} that fulfills four conditions:
(1) all putative homologies among loci are determined automatically, 
(2) each locus is either aligned with only one putatively homologous locus or
considered as a locus indel, 
(3) loci are allowed to rearrange
(4) the total cost to transform one genome into another genome
(i.e. nucleotide transformation costs, locus indel costs, 
and locus order rearrangement costs) is minimized.  To this end, 
the approach consists of two phases. First, reliable homologies 
between two chromosomes are detected automatically. Second,
conserved areas serve as anchors to divide each chromosome into 
a sequence of separated loci.  To construct comprehensive chromosome
pairwise alignments the same method used for annotated chromosomes is applied~\cite{vinh2007}. 
Note that only a user-defined maximum number of PARs and medians are considered
during the tree search.


To find reliable homologies between two chromosomes, we apply a three-step algorithm.
First, identical segments, called \emph{seeds}, with lengths greater or equal to
user-defined \emph{seed\_length} between two genomes are detected using suffix tree
structure. Detected seeds whose distance is not greater than
\emph{rearranged\_len} are connected to construct larger conserved areas, called
\emph{blocks}. Blocks whose lengths are greater than the user-defined significant block length 
threshold \emph{sig\_block\_len} are considered as reliable homologies. 

A discussion of the influence of \emph{seed\_len},
\emph{sig\_block\_len} and \emph{rearranged\_len} follows:

The best default value of \emph{seed\_len} is \texttt{t}.
The higher the \emph{seed\_len} value, the fewer seeds are detected,
that, in turn influences the number of blocks recognized. Conversely, if the
value of seed\_length is low, an increased number of seeds and, consequently, a
greater number of short blocks are detected.

The default value of \emph{sig\_block\_len} is \texttt{100}. 
If the value of \emph{sig\_block\_len} is low, small-size rearrangements are allowed;
whereas if the value of \emph{sig\_block\_len} is high only large-size
rearrangements can be detected.

The \emph{rearranged\_len} parameter sets a threshold
value under which homologous blocks separated by non-homologous regions can be
considered as a single block. 
The default for this parameter is \texttt{100}.  Therefore, if two inferred
homologous blocks are separated by less than 100 nucleotides they will be
treated as a single block in calculating of rearrangement events.

Thus, the combination of parameters \poycommand{seed\_length}, 
\poycommand{sig\_block\_len}, 
and \poycommand{rearranged\_len} significantly influence the estimation of inferred rearrangements.

\begin{table}[t]
\caption{The influence of \emph{seed\_length} to the runtime and tree cost
or 11 corona viruses 27-32kb in length}
\label{seedLength} 
\begin{center}
\begin{tabular}{l c c}
\hline
	\emph{seed\_length} & tree cost & runtime (minutes) \\
\hline
         7             & 109085   & 16\\
         9 (default)   & 91099   & 12\\
         11            & 91524   & 13\\
\hline
\end{tabular}
\end{center}
\end{table}

To examine the running time, we collect 11 corona viruses 
27-13kb in length. The program takes about 12 minutes
to reconstruct a Wagner tree, and around one hour 
for SPR swapping.



\begin{table}[t]
\caption{The default and suggested values of different parameters for chromosome
characters}
\label{defaultPam} 
\begin{center}
\begin{tabular}{l c c}
\hline
	name & default value cost & suggested value \\
\hline
    \poycommand{seed\_length}     & 9   & 5-15\\
    \poycommand{sig\_block\_len}  & 100 & 60-150\\
    \poycommand{rearranged\_len}   & 100 & 50-1000\\
    \poycommand{breakpoint}       &10   & 10-50\\
    \poycommand{inversion}        &none & 15-35\\
    \poycommand{approx}           &false& large data sets\\
    \poycommand{median}           &1    & 1-2\\
    \poycommand{swap\_med}        &1    & 1-2\\
    \poycommand{locus\_indel}     &opening 10, extension 1  & opening 10, extension 1\\
\hline
\end{tabular}
\end{center}
\end{table}


Table \ref{defaultPam} summarizes the default and suggested values of different
parameters for chromosome characters.


\section{Transformation cost regimes}
In analyses at the level of nucleotides, there are three general approaches to selecting transformation cost regimes most commonly used by \poy practitioners.
\begin{description}
\item[Equal costs] This approach assigns the same cost to all substitutions and indels, and does not take into account gap extension cost. For rationale for using this cost regime see Frost et al. \cite{frost2001} %and for other examples of its application see.
\item[Homology maximization] This approach, developed by De Laet \cite{delaet2005}, assigns costs \texttt{2, 3, and 1} to transformations, gap opening, and gap extension respectively. %For examples using this methods see
\item[Parameter sensitivity analysis] This method, suggested by Wheeler \cite{wheeler1995}, explores the effect of varying transformation costs by comparing results of analyses conducted under different cost regimes. Partition inconguence can subsequently be computed  for each cladogram and the parameter set that minimizes incongruence is selected as optimal. %For examples using this methods see
\end{description}
More specifically, it depends on relative costs of nucleotide- and locus-level transformations. Nucleotide-level transformations are specified by tcm argument, the locus-level rearrangements are specified by locus\_breakpoint or inversion costs. If locus\_level rearrangement costs are extremely high, the rearrangements are not going to be counted. On the other hand, if their cost is very low (equal or slightly above that of the nucleotide-level rearrangements), rearrangements can be frequent (depending on the seed\_block\_len and seed\_length settings).

When DNA sequence data is combined with morphological data, the cost for morphological character transformations is customarily is set to be the same as for substitutions.



\chapter{\poy Tutorials}
These tutorials are intended to provide guidance for more sophisticated applications of \poy that involve multiple steps and a combination of different commands. Each tutorial contains a \poy script that is followed by detailed commentaries explaining the rationale behind each step of the analysis. Although these analyses can be conducted interactively using the \emph{Interactive Console} or running separate sequential analyses using the \emph{Graphical User Interphace}, the most practical way to do this is to use \poy scripts (see \emph{ POY4 Quick Start} for more information on \poy scripts).

It is important to remember that the numerical values for most command arguments will differ substantially depending on type, complexity, and size of the data. Therefore, the values used here should not be taken as optimal parameters.

The tutorials use sample datasets that are provided with \poy installation but can also be downloaded from the \poy site at
\begin{center}
\texttt{http://research.amnh.org/scicomp/projects/poy.php}
\end{center}
The minimal required items to run the tutorial analyses are the \poy application and the sample datafiles. Running these analyses requires some familiarity with \poy interface and command structure that can be found in the preceding chapters.

\section{Combining  search strategies}{\label{tutorial4.1}}
The following script implements a strategy for a thorough search. This is accomplished by generating a large number of independent initial trees by random addition sequence and combining different search strategies that aim at thoroughly exploring local tree space and escape the effect of composite optima by effectively traversing the tree space. In addition, this script shows how to output the status of the search to a log file and calculate the duration of the search. 

\begin{verbatim}
(* search using all data *)
read("9.fas","31.ss", aminoacids:("41.aa"))
(* We select as root the taxon with name t1. If we wanted
the taxon Locusta_migratoria, we would write:
root:"Locusta_migratoria" *)
set(seed:1,log:"all_data_search.log",root:"t1")
report(timer:"search start")
transform(tcm:(1,2),gap_opening:1)
build(250)
swap(threshold:5.0)
select()
perturb(transform(static_approx),iterations:15,ratchet:(0.2,3))
select()
fuse(iterations:200,swap())
select()
report("all_trees",trees:(total),"constree",graphconsensus,
"diagnosis",diagnosis)
report(timer:"search end")
set(nolog)
exit()
\end{verbatim}

\begin{itemize}
\item \texttt{(* search using all data *)} This first line of the script is a comment. While comments are optional and do not affect the analyses, they provide are useful for housekeeping purposes.
\item \texttt{read("9.fas","31.ss", aminoacids:("41.aa"))}
This command imports all the nucleotide sequence datafiles (all files with the extension \texttt{.seq}), a morphological datafile \texttt{morph.ss} in Hennig86 format, and an aminoacid datafile \texttt{myosin.aa}.
\item \texttt{set(seed:1,log:"all\_data\_search.log",root:"t1")} The \poycommand{set} command specifies conditions prior to tree searching. The \poyargument{seed} is used to ensure that the subsequent randomization procedures (such as tree building and selecting) are reproducible. Specifying the log produces a file, \texttt{all\_data\_search.log} that provides an additional means to monitor the process of the search. The outgroup (\texttt{taxon1}) is designated by the \poyargument{root}, so that all the reported trees have the desired polarity. By default, the analysis is performed using direct optimization.
\item \texttt{report(timer:"search start")} In combination with \texttt{report(timer:\\"search end")}, this commands reports the amount of time that the execution of commands enclosed by \poyargument{timer} takes. In this case, it reports how long it takes for the entire search to finish. Using timer is useful for planning a complex search strategy for large datasets that can take a long time to complete: it is instructive, for example, to know how long a search would last with a single replicate (one starting tree) before starting a search with multiple replicates.
\item \texttt{transform(tcm:(1,2),gap\_opening:1)} This command sets the \\transformation cost matrix for molecular data to be used in calculating the cost of the tree. Note, that in addition to the substitution and indel costs, the \poycommand{transform} specifies the cost for gap opening.
\item \texttt{build(250)} This commands begins tree-building step of the search that generates 250 random-addition trees. A large number of independent starting point insures that a large portion of tree space have been examined.
\item \texttt{swap(threshold:5.0)} \poycommand{swap} specifies that each of the 250 trees is subjected to alternating SPR and TBR branch swapping routine (the default of \poy). In addition to the most optimal trees, all the suboptimal trees found within 5\% of the best cost are thoroughly evaluated. This step ensures that the local searches settled on the local optima.
\item \texttt{select()} Upon completion of branch swapping, this command retains only optimal and topologically unique trees; all other trees are discarded from memory. 
\item \texttt{perturb(transform(static\_approx),iterations:15,ratchet:\\(0.2,3))} This command subjects the resulting trees to 15 rounds of ratchet, re-weighting 20\% of characters by a factor of 2. During ratcheting, the dynamic homology characters are transformed into static homology characters, so that the fraction of nucleotides (rather than of sequence fragments) is being re-weighted. This step, that begins at multiple local maxima, is intended to further traverse the tree space in search of a global optimum.
\item \texttt{fuse(iterations:200,swap())} In this step, up to 200 swappings of subtrees identical in terminal composition but different in topology, are performed between pairs of best trees recovered in the previous step. This is another strategy for further exploration of tree space. Each resulting tree is further refined by local branch swapping under the default parameters of \poycommand{swap}.
\item \texttt{select()} Upon completion of branch swapping, this command retains only optimal and topologically unique trees; all other trees are discarded from memory.
\item \texttt{report("all\_trees",trees:(total),"constree",\\graphconsensus,"diagnosis",diagnosis)} This command produces a series of outputs of the results of the search. It includes a file containing best trees in parenthetical notation and their costs (\texttt{all\_trees}), a graphical representation (in PDF format) of the strict consensus (\texttt{constree}), and the diagnoses for all best trees (\texttt{diagnosis}).
\item \texttt{report(timer:"search end")} This command stops timing the duration of search, initiated by the command \texttt{report(timer:"search start")}.
\item \texttt{set(nolog)} This command stops reporting any output to the log file, \texttt{all\_data\_search.log}.
\item \texttt{exit()} This commands ends the \poy session.
\end{itemize}

\section{Searching under iterative pass}{\label{tutorial4.2}}
The following script implements a strategy for a thorough search under iterative pass optimization. The iterative pass optimization is a very time consuming procedure that makes it impractical to conduct under this kind of optimization (save for very small datasets that can be analyzed within reasonable time). The iterative pass, however, can be used for the most advanced stages of the analysis for the final refinement, when a potential global optimum has been reached through searches under other kinds of optimization (such as direct optimization). Therefore, this tutorial begins with importing an existing tree (rather than performing tree building from scratch) and subjecting it to local branch swapping under iterative pass.

\begin{verbatim}
(* search using all data under ip *)
read("9.fas","31.ss",aminoacids:("41.aa"))
read("inter_tree.tre")
transform(tcm:(1,2),gap_opening:1)
set(iterative:approximate:2)
swap(around)
select()
report("all_trees",trees:(total),"constree", graphconsensus,
"diagnosis",diagnosis)
transform ((all, static_approx))
report ("phastwinclad.ss", phastwinclad)
exit()
\end{verbatim}

\begin{itemize}
\item \texttt{(* search using all data under ip *)} This first line of the script is a comment. While comments are optional and do not affect the analyses, they provide are useful for housekeeping purposes.
\item \texttt{read("9.fas","31.ss",aminoacids:("41.aa"))} This command imports all the nucleotide sequence datafiles (all files with the extension \texttt{.seq}), a morphological datafile \texttt{morph.ss} in Hennig86 format, and an aminoacid datafile \texttt{myosin.aa}.
\item \texttt{read("inter\_tree.tre")} This command imports a tree file, \texttt{inter\_tree.tre}, that contains the most optimal tree from prior analyses. 
\item \texttt{transform(tcm:(1,2),gap\_opening:1)} This command sets the transformation cost matrix for molecular data to be used in calculating the cost of the tree. Note, that in addition to the substitution and indel costs, the \poycommand{transform} specifies the cost for gap opening.
\item \texttt{set(iterative:approximate:2)} This command sets the optimization procedure
    to iterative pass such that approximated three dimensional alignments generated using pairwise alignments will be considered.  The program will iterate either two times, or until no further tress cost improvements can be made.
\item \texttt{swap(around)} This commands specifies that the the imported tree is subjected to alternating SPR and TBR branch swapping routine (the default of \poy) following the trajectory of search that completely evaluates the neighborhood of the tree (by using \poyargument{around}).
\item \texttt{select()} Upon completion of branch swapping, this command retains only optimal and topologically unique trees; all other trees are discarded from memory.
\item \texttt{report("all\_trees",trees:(total),"constree",\\graphconsensus,"diagnosis",diagnosis)} This command produces a series of outputs of the results of the search. It includes a file containing best trees in parenthetical notation and their costs (\texttt{all\_trees}), a graphical representation (in PDF format) of the strict consensus (\texttt{constree}), and the diagnoses for all best trees (\texttt{diagnosis}).
\item \texttt{transform ((all, "static\_approx"))} This command transforming all data into static homology characters corresponding to their implied alignments is necessary before reporting the data in the Hennig86 format.
\item \texttt{report ("phastwinclad.ss", phastwinclad)}  This command produces a file in the Hennig86 format which can be imported into other programs, such as WinClada.
\item \texttt{exit()} This commands ends the \poy session.
\end{itemize}

\section{Bremer support}{\label{tutorial4.3}}

This tutorial builds on the previous tutorials to illustrate Bremer support 
calculation on trees constructed using dynamic homology characters
    
   \begin{verbatim}
(* Bremer support part 1: generating trees *)
read("18s.fas","28s.fas")
set(root:"Americhernus")
build(200)
swap(all,visited:"tmp.trees", timeout:3600)
select()
report("my.tree",trees)
exit()

(* Bremer support part 2: Bremer calculations *)
read("18s.fas","28s.fas","my.tree")
report("support_tree.pdf",graphsupports:bremer:"tmp.trees")
exit()
\end{verbatim}

\begin{itemize}
\item \texttt{(* Bremer support part1: generating trees *)} This first line of the script is a comment. While comments are optional and do not affect the analyses, they provide are useful for housekeeping purposes. 
\item \texttt{read("18s.fas","28s.fas")} This command imports the nucleotide sequence files \texttt{18s.fas, 28s.fas}.
\item \texttt{set(root:"Americhernus")} The \poycommand{set} command specifies conditions prior to tree searching. The outgroup (\texttt{Americhernus}) is designated by the \poyargument{root}, so that all the reported trees have the desired polarity.     
\item \texttt{build(200)} This commands initializes tree-building and generates 200 random-addition trees.      
\item \texttt{swap(all,visited:"tmp.trees", timeout:3600)} The \poycommand{swap} command specifies that each of the trees be subjected to an alternating SPR and TBR branch swapping routine (the default of \poy).  The \poyargument{all} argument turns offf all swap heuristics. The \poyargument{visited:"tmp.trees"} argument stores every visited tree in the file specified.  Although the visited tree file is compressed to accommodate the large number of trees it will accumulate, the argument \poyargument{timeout} can be used to limit the number of seconds allowed for swapping also limiting the size of the file.  Alternately  the  \poycommand{swap} command can be performed as a separate analysis and terminated at the users discretion to maximize the number of trees generated.
\item \texttt{select()} Upon completion of branch swapping, this command retains only optimal and topologically unique trees; all other trees are discarded from memory. 
\item \texttt{report("my.tree",trees)} This command will save the swapped tree, \\ \texttt{my.tree} to a file. 
\item \texttt{exit()} This commands ends the \poy session.

\item \texttt{(* Bremer support part 2: Bremer calculations *)}  A comment indicating the intent of the commands which follow.
\item \texttt{read("18s.fas","28s.fas","my.tree")} This command imports the nucleotide sequence files \texttt{18s.fas, 28s.fas} and the tree file, \texttt{my.tree} for which the support values will be generated.  It is important to only read the selected \texttt{"my.tree"} file rather than the expansive  \texttt{"tmp.trees"} file which will be used in bremer calculations.
\item \texttt{report("support\_tree.pdf",graphsupports:bremer:"tmp.trees")} \\The \poycommand{report} command in combination with a file name and the \\ \poyargument{graphsupports} generates a pdf file designated by the name \texttt{support\_tree.pdf} with bremer values for the selected trees held in \texttt{tmp.trees}.  It is strongly recommended that this more exhaustive approach is used for calculating Bremer supports rather than simply using the \\ \poyargument{graphsupports} default s.  
\item \texttt{exit()} This commands ends the \poy session.
\end{itemize}

\section{Jackknife support}{\label{tutorial4.4}}

This tutorial illustrates calculating Jackknife support values for trees constructed with static homology characters.  Although it is possible to calculate both Jackknife and Bootstrap support values for trees constructed using dynamic homology characters, it is not recommended because resampling of dynamic characters occurs at the fragment (rather than nucleotide) level. Alternately dynamic homology characters can be converted to static characters using the transform argument \poyargument{static\_approx}, as it is common for biologists to want to  compute support values by resampling the characters from a fixed alignment.  In jackknife support a specified number of pseudo-repicates are performed independently such that in each one a percentage of characters is selected at random, without replacement.  The frequency of clade occurrence is its jackknife value.  

\begin{verbatim}
(* Jackknife support using static nucleotide characters *)
read ("28s.fas")
search (max_time:0:0:5)
select ()
report ("tree_for_supports.tre", trees)
transform(static_approx)
calculate_support (jackknife:(remove:0.50, resample:1000))
report (supports:jackknife)
report ("jacktree", graphsupports:jackknife)
exit()
\end{verbatim}
\begin{itemize}
\item \texttt{(* Jackknife support using static nucleotide characters *)} This first line of the script is a comment. While comments are optional and do not affect the analyses, they provide are useful for housekeeping purposes.
\item \texttt{read("28s.fas")} This command imports nucleotide sequence file \texttt{28s.fas}.
\item \texttt{search(max\_time:0:0:5)} This command performs a search (i.e. build, swap, perturb, fuse) of the data \texttt{28s.fas} for a maximum of 5 min (note that the short search time was selected for demonstration purposes to expedite the tutorial and not as a general time recommendation for actual data analyses).   
\item \texttt{select()} This command retains only optimal and topologically unique trees; all other trees are discarded from memory. 
\item \texttt{report("tree\_for\_supports.tre", trees)}  This command outputs a parenthetical representation of the tree \texttt{"tree\_for\_supports.tre"}.
\item \texttt{transform(static\_approx)} This command transforms the data (i.e. build, swap, perturb, fuse) of the data \texttt{28s.fas} for a maximum of 2 hours.
\item \texttt{calculate\_support(jackknife,(remove:0.50,resample:1000)} The \poycommand{calculate\_support} command generates support values as specified by the \poyargument{jackknife} argument for each tree held in memory. During each pseudoreplicate half of the characters will be deleted as specified in the argument\poyargument{remove:0.50}. 
\item \texttt{report(supports:jackknife)}  This command outputs a parenthetical representation of a tree with the support values previously calculated with the \poycommand{calculate\_support} command. 
\item \texttt{report("jacktree",graphsupports:jackknife)}  The \poycommand{report} command in combination with a file name and the \poyargument{graphsupports} generates a pdf file with jackknife values designated by the name specified (\emph{i.e.} \texttt{jacktree}). 
\item \texttt{exit()} This commands ends the \poy session.
\end{itemize}

\section{Sensitivity analysis}{\label{tutorial4.5}}

This tutorial demonstrates how data for parameter sensitivity analysis is generated. Sensitivity analysis \cite{wheeler1995} is a method of exploring the effect of relative costs of substitutions (transitions and transversions) and indels (insertions and deletions), either with or without taking gap extension cost into account. The approach consists of multiple iterations of the same search strategy under different parameters, (\emph{i.e} combinations of substitution and indel costs. Obviously, such analysis might become time consuming and certain methods are shown here how to achieve the results in reasonable time. This tutorial also shows the utility of the command \poycommand{store} and how transformation cost matrixes are imported and used.

\poy does not comprehensively display the results of the sensitivity analysis or implements the methods to select a parameter set that produces the optimal cladogram, but the output of a \poy analysis (such as the one presented here) generates all the necessary data for these additional steps.

For the sake of simplicity, this script contains commands for generating the data under just two parameter  sets. Using a larger number of parameter sets can easily be achieved by replicating the repeated parts of the script and substituting the names of input cost matrixes.

\begin{verbatim}
(* sensitivity analysis *)
read("9.fas")
set(root:"t1")
store("original_data")
transform(tcm:"111.txt")
build(100)
swap(timeout:3600)
select()
report("111.tre",trees:(total) ,"111con.tre",consensus,
"111con.pdf",graphconsensus)
use("original_data")
transform(tcm:"112.txt")
build(100)
swap(timeout:3600)
select()
report("112.tre",trees:(total),"112con.tre",consensus,
"112con.pdf",graphconsensus)
exit()
\end{verbatim}

\begin{itemize}
\item \texttt{(* sensitivity analysis *)} This first line of the script is a comment. While comments are optional and do not affect the analyses, they provide are useful for housekeeping purposes.
\item \texttt{read("9.fas")} This command imports all dynamic homology nucleotide data.
\item \texttt{set(root:"t1")} The outgroup (\texttt{taxon1}) is designated by the \poyargument{root}, so that all the reported trees have the desired polarity.
\item \texttt{store("original\_data")} This commands stores the current state of analysis in memory in a temporary file, \texttt{original\_data}.
\item \texttt{transform(tcm:"111.txt")} This command applies a transformation cost matrix from the file \texttt{111.txt} to for subsequent tree searching.
\item \texttt{build(100)} This commands begins tree-building step of the search that generates 250 random-addition trees. A large number of independent starting point insures that thee large portion of tree space have been examined.
\item \texttt{swap(timeout:3600)} \poycommand{swap} specifies that each of the 100 trees build in the previous step is subjected to alternating SPR and TBR branch swapping routine (the default of \poy). The argument \poyargument{timeout} specifies that 3600 seconds are allocated for swapping and the search is going to be stopped after reaching this limit. Because sensitivity analysis consists of multiple independent searches, it can take a tremendous amount of time to complete each one of them. In this example, \poyargument{timeout} is used to prevent the searches from running too long. Using \poyargument{timeout} is optional and can obviously produce suboptimal results due to insufficient time allocated to searching. A reasonable timeout value can be experimentally obtained by the analysis under one cost regime and monitoring time it takes to complete the search (using the argument \poyargument{timer} of the command \poycommand{set}). The advantage of using \poyargument{timeout} is saving time in cases where a local optimum is quickly reached and the search is trapped in its neighborhood.
\item \texttt{select()} Upon completion of branch swapping, this command retains only optimal and topologically unique trees; all other trees are discarded from memory.
\item \texttt{report("111.tre",trees:(total) ,"111con.tre",consensus,\\"111con.pdf", graphconsensus)} This command produces a file containing best tree(s) in parenthetical notation and their costs (\texttt{111.tre}), a a file containing the strict consensus in parenthetical notation \\(\texttt{111con.tre}), and a graphical representation (in PDF format) of the strict consensus (\texttt{111con.pdf}).
\item \texttt{use("original\_data")} This command restored the original (non-trans\-formed) data from the temporary file \texttt{original\_data} generated by \poycommand{store}.
\item \texttt{transform(tcm:"112.txt")} This command applies a different transformation cost matrix from the file \texttt{112.txt} to for another round of tree searching under this new cost regime.
\item \texttt{build(100)} This commands begins tree-building step of the search that generates 100 random-addition trees. A large number of independent starting point insures that thee large portion of tree space have been examined.
\item \texttt{swap(timeout:3600)} \poycommand{swap} specifies that each of the 100 trees build in the previous step is subjected to alternating SPR and TBR branch swapping routine (the default of \poy) to be interrupted after 3600 seconds (see the description in the previous iteration of the command above).
\item \texttt{select()} Upon completion of branch swapping, this command retains only optimal and topologically unique trees; all other trees are discarded from memory.
\item \texttt{report("112.tre",trees:(total),"112con.tre",consensus,\\"112con.pdf", graphconsensus)} This command produces a set of the same kinds of outputs as generated during the first search (see above) but under a new cost regime.
\item \texttt{exit()} This commands ends the \poy session.
\end{itemize}

\section{Chromosome analysis: unannotated sequences}{\label{tutorial4.6}}

This tutorial illustrates the analysis of chromosome-level transformations using 
unannotated sequences, i.e., contiguous strings of sequences without prior 
identification of independent regions. Prior to attempting an analysis of  
unannotated chromosomes it is necessary to enable the \texttt {"long sequences"}
option when compiling the \texttt{POY4} program. 

\begin{verbatim}
(* Chromosome analysis of unannotated sequences *)
read(chromosome:("ua15.fas"))
transform((all,dynamic_pam:(locus_breakpoint:20,locus_indel:
(10,1.5),circular:true,median:2, min_seed_length:15, 
min_rearrangement_len:45, min_loci_len:50,median:2,swap_med:1)))
build()
swap()
select()
report("chrom",diagnosis)
report("consensustree",graphconsensus)
exit()
\end{verbatim}

\begin{itemize}
\item \texttt{(* Chromosome analysis of unannotated sequences *)} This first line of the script is a comment. While comments are optional and do not affect the analyses, they provide are useful for housekeeping purposes.
\item \texttt{read(chromosome:("ua15.fas"))} This command imports the unannotated chromosomal sequence file \texttt{ua15.fas}. The argument \poyargument{chromosome} specifies the characters as unannotated chromosomes.
\item \texttt{transform((all,dynamic\_pam:(locus\_breakpoint:20,locus\_indel:\\(10,1.5),circular:true,seed\_length:15,rearranged\_len:50,sig\_block\_len:50,median:2,swap\_med:1)))}  The \poycommand{trans\-form} followed by the argument \poyargument{dynamic\_pam} specifies the conditions to be applied when calculating chromosome-level HTUs (medians).  The argument \poyargument{locus\_breakpoint:20} applies a breakpoint distance between chromosome loci with the integer value determining the rearrangement cost. The argument \poyargument{locus\_indel:10,1.5} specifies the indel costs for the chromosomal segments, whereby the integer 10 sets the gap opening cost and the float 1.5 sets the gap extension cost.  As the type of chromosomal sequences being analyzed are of mitochondrial origin, the argument \poyargument{circular:true} treats each chromosome sequence as a continuous rather than linear. The argument \poyargument{min\_seed\_length:15} sets the minimum length of identical continuous fragments (seeds) at 15.  As seeds are the foundation for larger homologous blocks setting the seed length to an integer appropriate for the data is critical to optimizing the efficiency with which the program correctly identifies chromosomal fragments and detects rearrangements.  The \poyargument{min\_rearrangement\_len} argument sets the lower limit for number of nucleotides between two seeds such that each is considered independent of the other.  Independent seeds belong to separate homologous blocks such that rearrangement events between blocks can be detected.  The argument \poyargument{min\_loci\_len} provides the integer value determining the minimum number of nucleotides constituting an homologous block.  In this example, because the data are mitochondrial containing relatively short homologous tRNA sequences, both the \poyargument{min\_rearrangement\_len} and the \poyargument{min\_loci\_len} were set to values below the defaults for these arguments.  The \poyargument{median} specifies the number of best cost locus-rearrangements which will be considered for each HTU (median), while the \poyargument{swap\_med} argument specifies the maximum number of swapping iterations performed in searching for the best pairwise alignment between two chromosomes.  Because values for the \poyargument{median} and \poyargument{swap\_med} arguments set above the default (1) will significantly increase the calculation time, the default values are recommended for larger chromosomal data sets.
\item \texttt{build()} This commands begins the tree-building step of the search that generates by default 10 random-addition trees. It is highly recommended that a greater number of Wagner builds be implemented when analyzing data for purposes other than this demonstration.
\item \texttt{swap()} The \poycommand{swap} command specifies that each of the trees be subjected to an alternating SPR and TBR branch swapping routine (the default of \poy).
\item \texttt{select()} Upon completion of branch swapping, this command retains only optimal and topologically unique trees; all other trees are discarded from memory. 
\item \texttt{report("chrom",diagnosis)}  The \poycommand{report} command in combination with a file name and the \poyargument{diagnosis} outputs the optimal median states and edge values to a specified file (\texttt{chrom}). 
\item \texttt{report("consensustree",graphconsensus)}  The \poycommand{report} command in combination with a file name and the \poyargument{graphconsensus} generates a pdf strict consensus file of the trees generated (\texttt{consensustree}). 
\item \texttt{exit()} This commands ends the \poy session.
\end{itemize}

\section{Chromosome analysis: annotated sequences}{\label{tutorial4.7}}

This tutorial illustrates the analysis of chromosome-level transformations using 
annotated sequences, i.e., contiguous strings of sequences with prior 
identification of independent regions delineated by pipes  \texttt{"|"}. 

\begin{verbatim}
(* Chromosome analysis of annotated sequences *)
read(annotated:("aninv2"))
transform((all,dynamic_pam:(locus_inversion:20,locus_indel:(10,
1.5),circular:false,median:1,swap_med:1)))
build()
swap()
select()
report("Annotated",diagnosis)
report("consensustree",graphconsensus)
exit()
\end{verbatim}

\begin{itemize}
\item \texttt{(* Chromosome analysis of annotated sequences  *)} This first line of the script is a comment. While comments are optional and do not affect the analyses, they provide are useful for housekeeping purposes.
\item \texttt{read(annotated:("aninv2"))} This command imports the annotated chromosomal sequence file \texttt{aninv2}. The argument \poyargument{annotated} specifies the characters. 
\item \texttt{transform((all,dynamic\_pam:(inversion:20,locus\_indel:\\(10,1.5),median:1,swap\_med:1)))}  The \poycommand{transform} follow\-ed by the argument \poyargument{dynamic\_pam} specifies the conditions to be applied when calculating chromosome-level HTUs (medians).  The argument \poyargument{locus\_inversion:20} applies an inversion distance between chromosome loci with the integer value determining the rearrangement cost. The argument \poyargument{locus\_indel:10,1.5} specifies the indel costs for the chromosomal segments, whereby the integer 10 sets the gap opening cost and the float 1.5 sets the gap extension cost.  The default values are applied to the arguments \poyargument{circular}  \poyargument{median} and \poyargument{swap\_med} arguments to minimize the time require for these nested search options.   To more exhaustively perform these calculations trees generated from initial builds can be imported to the program and reevaluated with values greater than 1 designated for the \poyargument{median} and \poyargument{swap\_med} arguments.
\item \texttt{build()} This commands begins the tree-building step of the search that generates by default 10 random-addition trees.  It is highly recommended that a greater number of Wagner builds be implemented when analyzing data for purposes other than this demonstration.
\item \texttt{swap()} The \poycommand{swap} command specifies that each of the trees be subjected to an alternating SPR and TBR branch swapping routine (the default of \poy).
\item \texttt{select()} Upon completion of branch swapping, this command retains only optimal and topologically unique trees; all other trees are discarded from memory. 
\item \texttt{report("Annotated",diagnosis)}  The \poycommand{report} command in combination with a file name and the \poyargument{diagnosis} outputs the optimal median states and edge values to a specified file (\texttt{Annotated}). 
\item \texttt{exit()} This commands ends the \poy session.
\end{itemize}

\section{Custom alphabet break inversion characters}{\label{tutorial4.8}}

This tutorial illustrates the analysis of the break inversion character type.  Break inversion characters are generated by transforming user-defined \poyargument {custom\_alphabet} characters.  
For example, observations of developmental stages could be represented in a corresponding array such that for each terminal taxon there is a sequence of observed developmental stages which are represented by a user-defined alphabet.  To allow rearrangement as well as indel events to be considered among alphabet elements, requires either reading in the data with the \poyargument {breakinv} argument or transforming the \poyargument {custom\_alphabet} sequences read to \poyargument {breakinv} characters. 

\begin{verbatim}
(* Custom Alphabet to Breakinv characters *)
read(custom_alphabet:("ca1.fas","m1.fas"))
transform((all,custom_to_breakinv:()))
transform((all,dynamic_pam:(locus_breakpoint:20,
locus_indel:(10,1.5),median:1,swap_med:1)))
build()
swap()
select()
report("breakinv",diagnosis)
report("consensustree",graphconsensus)
exit()
\end{verbatim}

\begin{itemize}
\item \texttt{(* Custom Alphabet to Breakinv characters  *)} This first line of the script is a comment. While comments are optional and do not affect the analyses, they provide are useful for housekeeping purposes.
\item \texttt{read(custom\_alphabet:("ca1.fas","m1.fas"))} This command imports the user-defined \poyargument {custom\_alphabet} character file \texttt{ca1.fas} and the accompanying transformation matrix \texttt{m1.fas}.
\item \texttt{transform((all,custom\_to\_breakinv:()))} This command transforms \poyargument {custom\_alphabet} characters to \poyargument {breakinv} characters which allow for rearrangement operations.
\item \texttt{transform((all,dynamic\_pam:(locus\_breakpoint:20,locus\_\\indel:(10,1.5),median:1,swap\_med:1)))}  The \poycommand{transform} follow\-ed by the argument \poyargument{dynamic\_pam} specifies the conditions to be applied when calculating medians. The argument \poyargument{locus\_breakpoint:20} applies a breakpoint distance calculation where the integer value specifies the rearrangement cost of \poyargument {breakinv} elements. The argument \poyargument{locus\_indel:10,1.5} specifies the indel costs for each \poyargument {breakinv} element, whereby the integer 10 sets the gap opening cost and the float 1.5 sets the gap extension cost.  The default values are applied to the \poyargument{median} and \poyargument{swap\_med} arguments to minimize the time require for these nested search options.   To more exhaustively perform these calculations trees generated from initial builds can be imported to the program and reevaluated with values greater than 1 designated for the \poyargument{median} and \poyargument{swap\_med} arguments
\item \texttt{build()} This commands begins the tree-building step of the search that generates by default 10 random-addition trees.  It is highly recommended that a greater number of Wagner builds be implemented when analyzing data for purposes other than this demonstration.
\item \texttt{swap()} The \poycommand{swap} command specifies that each of the trees be subjected to an alternating SPR and TBR branch swapping routine (the default of \poy).
\item \texttt{select()} Upon completion of branch swapping, this command retains only optimal and topologically unique trees; all other trees are discarded from memory. 
\item \texttt{report ("breakinv",diagnosis)}  The \poycommand{report} command in combination with a file name and the \poyargument{diagnosis} outputs the optimal median states and edge values to a specified file (\texttt{breakinv}). 
\item \texttt{exit()} This commands ends the \poy session.
\end{itemize}

\section{Genome analysis: multiple chromosomes}{\label{tutorial4.9}}

This tutorial illustrates the analysis of genome-level transformations using data from multiple chromosomes. 
Prior to attempting an analysis of unannotated chromosomes it is necessary to enable the \texttt {"long sequences"}
option when compiling the \texttt{POY4} program. 

\begin{verbatim}
(* Genome analysis of multiple chromosomes *)
read (genome:("gen5bp"))
transform((all,dynamic_pam:(chrom_breakpoint:80, chrom_indel:
(15,2.5),locus_inversion:20,locus_indel:(10,1.5), median:1,
swap_med:1)))
build()
swap()
select()
report("genome",diagnosis)
report("genconsensus",graphconsensus)
exit()
\end{verbatim}

\begin{itemize}
\item \texttt{(* Genome analysis of multiple chromosomes*)} This first line of the script is a comment. While comments are optional and do not affect the analyses, they provide are useful for housekeeping purposes.
\item \texttt{read(genome:("gen5bp"))} This command imports the genomic sequence file \texttt{mit5.txt}. The argument \poyargument{genome} specifies the characters as data consisting of multiple chromomsomes.
\item \texttt{transform((all,dynamic\_pam:(chrom\_breakpoint:80,chrom\_indel:(15, 2.5),locus\_breakpoint:20,locus\_indel:(10,1.5),median:1,swap\_med:1)))}  The \poycommand{transform} followed by the argument \poyargument{dynamic\_pam} specifies the conditions to be applied when calculating genome-level HTUs (medians). The argument \poyargument{chrom\_breakpoint:80} applies a breakpoint distance between chromosomes with the integer value determining the rearrangement cost. The argument \poyargument{chrom\_indel:15,1.5} specifies the indel costs for each entire chromosome, whereby the integer sets the gap opening cost and the float sets the gap extension cost.  The argument \poyargument{locus\_inversion:20} applies an inversion distance between loci with the integer value determining the rearrangement cost. The argument \poyargument{locus\_indel:10,1.5} specifies the indel costs for the chromosomal segments, whereby the integer 10 sets the gap opening cost and the float 1.5 sets the gap extension cost.  The default values are applied to the \poyargument{median} and \poyargument{swap\_med} arguments to minimize the time require for these nested search options.   To more exhaustively perform these calculations trees generated from initial builds can be imported to the program and reevaluated with values greater than 1 designated for the \poyargument{median} and \poyargument{swap\_med} arguments
\item \texttt{build()} This commands begins the tree-building step of the search that generates by default 10 random-addition trees.  It is highly recommended that a greater number of Wagner builds be implemented when analyzing data for purposes other than this demonstration.
\item \texttt{swap()} The \poycommand{swap} command specifies that each of the trees be subjected to an alternating SPR and TBR branch swapping routine (the default of \poy).
\item \texttt{select()} Upon completion of branch swapping, this command retains only optimal and topologically unique trees; all other trees are discarded from memory. 
\item \texttt{report("genome",diagnosis)}  The \poycommand{report} command in combination with a file name and the \poyargument{diagnosis} outputs the optimal median states and edge values to a specified file (\texttt{genome}). 
\item \texttt{report("genconsens",graphconsensus)}  The \poycommand{report} command in combination with a file name and the \poyargument{graphconsensus} generates a pdf strict consensus file of the trees generated (\texttt{genconsensus}). 
\item \texttt{exit()} This commands ends the \poy session.
\end{itemize}


\addcontentsline{toc}{section}{Bibliography}%%
\bibliography{doc/poylibrary}
\bibliographystyle{plain}

%\addtocontents{toc}{General Index}%%
\printindex{general}{General Index}
\printindex{poy3}{POY 3.0 Commands Index}
\index{general}{jack2hen|see{clades}}
\index{poy3}{agree|see{constraint}}
\index{poy3}{bremer|see{calculatesupports}}
\index{poy3}{bremerspr|see{calculatesupports, swap}}
\index{poy3}{build|see{build}}
\index{poy3}{topodiagnoseonly|see{read}}
\index{poy3}{buildmaxtrees|see{trees}}
\index{poy3}{buildslop|see{threshold}}
\index{poy3}{buildspr|see{spr}}
\index{poy3}{buildtbr|see{tbr}}
\index{poy3}{characterweights|see{report}}
\index{poy3}{commandfile|see{run}}
\index{poy3}{commandfiledir|see{cd}}
\index{poy3}{datadir|see{cd}}
\index{poy3}{defaultweight|see{weight}}
\index{poy3}{diagnose|see{report}}
\index{poy3}{disagree|see{constraint}}
\index{poy3}{driftequallaccept|see{drifting}}
\index{poy3}{driftlengthbase|see{drifting}}
\index{poy3}{driftspr|see{drifting}}
\index{poy3}{drifttbr|see{drifting}}
\index{poy3}{drifttrees|see{drifting}}
\index{poy3}{dropconstraints|see{constraint}}
\index{poy3}{extensiongap|see{gapopening}}
\index{poy3}{extensiongap|see{tcm}}
\index{poy3}{gap|see{tcm}}
\index{poy3}{gap|see{gapopening}}
\index{poy3}{gc|see{memory}}
\index{poy3}{holdmaxtrees|see{trees}}
\index{poy3}{hypancfile|see{diagnosis}}
\index{poy3}{hypancname|see{diagnosis}}
\index{poy3}{impliedalignment|see{implied\_alignment}}
\index{poy3}{iafiles|see{implied\_alignment}}
\index{poy3}{noiafiles|see{report}}
\index{poy3}{indices|see{treestats}}
\index{poy3}{intermediate|see{trajectory}}
\index{poy3}{jackboot|see{jackknife}}
\index{poy3}{jackfrequencies|see{jackknife}}
\index{poy3}{jackoutgroup|see{outgroup}}
\index{poy3}{jackstart|see{jackknife}}
\index{poy3}{poytreefile|see{trees}}
\index{poy3}{poybintreefile|see{trees}}
\index{poy3}{poystrictconsensustreefile|see{consensus}}
\index{poy3}{leading|see{trailing\_insertion}}
\index{poy3}{maxtrees|see{trees}}
\index{poy3}{molecularmatrix|see{tcm}}
\index{poy3}{newstates|see{fixed\_states}}
\index{poy3}{numdriftchanges|see{repeat}}
\index{poy3}{numdriftspr|see{repeat}}
\index{poy3}{numdrifttbr|see{repeat}}
\index{poy3}{phastwincladfile|see{phastwinclad}}
\index{poy3}{plotfile|see{graphtrees}}
\index{poy3}{plotfrequencies|see{graphtrees}}
\index{poy3}{plotechocommandline|see{echo}}
\index{poy3}{plotmajority|see{graphconsensus}}
\index{poy3}{plotoutgroup|see{outgroup}}
\index{poy3}{plotstrict|see{graphconsensus}}
\index{poy3}{plottrees|see{graphtrees}}
\index{poy3}{printtree|see{asciitrees}}
\index{poy3}{random|see{trees}}
\index{poy3}{replicates|see{trees}}
\index{poy3}{ratchetoverpercent|see{ratchet}}
\index{poy3}{ratchetpercent|see{ratchet}}
\index{poy3}{ratchetseverity|see{ratchet}}
\index{poy3}{ratchetslop|see{perturb}}
\index{poy3}{ratchetspr|see{perturb}}
\index{poy3}{ratchettbr|see{perturb}}
\index{poy3}{ratchettrees|see{perturb}}
\index{poy3}{ratchetinseq|see{perturb}}
\index{poy3}{slop|see{threshold}}
\index{poy3}{sprmaxtrees|see{trees}}
\index{poy3}{staticapprox|see{static\_approx}}
\index{poy3}{staticapproxbuild|see{build}}
\index{poy3}{tbrmaxtrees|see{trees}}
\index{poy3}{topofile|see{read}}
\index{poy3}{topology|see{read}}
\index{poy3}{topolist|see{trees}}
\index{poy3}{topooutgroup|see{outgroup}}
\index{poy3}{treefuse|see{fuse}}
\index{poy3}{treefusespr|see{fuse}}
\index{poy3}{treefusetbr|see{fuse}}
\index{poy3}{cat\_commandbrowsing|see{help}}
\index{poy3}{cat\_helptopics|see{help}}
\index{poy3}{replicatebuild|see{trees}}
\index{poy3}{replicaterefinement|see{trees}}
\index{poy3}{finalrefinement|see{swap}}
\index{poy3}{trailinggap|see{trailinginsertion}}
\index{poy3}{trailinggap|see{trailingdeletion}}
% --
% -2n2reorder
% -allchroms
% -alltrees
% -approxbuild
% -approxdrift
% -approxquickspr
% -approxquicktbr
% -basefreq
% -cmptreefilefile
% -cmp2treefiles
% -cmp1treefile
% -bayes
% -bayesroundup
% -buildchrom
% -catchfile
% -catchslaveoutput
% -change
% -checkfrequency
% -checkslop
% -chromegap
% -chromelocusgap
% -chromesizegap
% -chromfilter
% -chromosome
% -circular
% -crashslave
% -crashcontroller
% -crashreserve
% -compressstates
% -controllers
% -cutswap
% -discrepancies
% -distcost
% -dogaptie
% -enabletmpfiles
% -estimateparamsfirst
% -estimatep
% -estimateq
% -exact
% -onversionconflict
% -fitchtrees
% -freqmodel
% -fuselimit
% -fusemaxtrees
% -minterminals
% -fusemingroup
% -fusingrounds
% -gammaalpha
% -gammaclasses
% -genome
% -getnewmatrix
% -goloboff
% -horizontalcost
% -horizontalrecombination
% -horizontalsizecost
% -horizontaltranslocationcost
% -incremental
% -insertpipes
% -invdist
% -invariantsitesadjust
% -iterativeinitfinal
% -iterativeinitsingle
% -iterativekeepbetter
% -iterativelowmem
% -iterativepass
% -iterativepassfinal
% -iterativerandom
% -jackexclusion
% -jackcharfile
% -jacktree
% -jackpseudotrees
% -jackpseudoconsensustrees
% -jackftree
% -jackfpseudotrees
% -jackfpseudoconsensustrees
% -jackwincladfile
% -poystrictconsensuscharfile
% -poymctreefile
% -poymccharfile
% -mrcutoff
% -mrcutpoymctreefile
% -mrcutpoymccharfile
% -mrcutplotmajority
% -mrcutplotfrequencies
% -topocutoff
% -mrcutjacktree
% -mrcutjackftree
% -mrcutjackcharfile
% -mrcutjackfrequencies
% -jobspernode
% -mtx_partition
% -kbasefreq
% -kencoding
% -keventfreq
% -kmdl
% -kfile
% -kmod
% -kp
% -kmatrix
% -kunit
% -kfactor
% -likelihood
% -likelihoodconvergancevalue
% -likelihoodestimationsize
% -likelihoodesttranseachtime
% -likelihoodextensiongap
% -likelihoodmaxnumiterations
% -likelihoodnoncodinggap
% -likelihoodroundingmultiplier
% -likelihoodstep
% -likelihoodstepinterval
% -likelihoodtrailinggap
% -linear
% -locusextgap
% -locussizegap
% -locusgap
% -locusswap
% -locusswap3d
% -lowmemthreshold
% -maxiterations
% -maxprocessors
% -numslaveprocesses
% -minstop
% -buildsperreplicate
% -multibuild
% -nomultibuild
% -multidrift
% -multiplier
% -multirandom
% -multiratchet
% -n2reorder
% -n2reorderorders
% -onan
% -onannum
% -dpmonannum
% -dpmsolospawn
% -onechroms
% -onereorder
% -oneasis
% -overwritedataprotection
% -overwriteprotection
% -pack_topology_rep
% -pairmatrix
% -parallel
% -plotencoding
% -plotwidth
% -prealigned
% -printhypanc
% -printlotshypanc
% -printqmat
% -fastashortname
% -nofastashortname
% -deletegapsfrominput
% -nodeletegapsfrominput
% -polyaddconverttorange
% -nopolyaddconverttorange
% -printccode
% -editnames
% -noprintccode
% -qmatrix
% -quick
% -quote
% -randomizeoutgroup
% -randomizeslaves
% -rearrange
% -recode
% -repintermediate
% -rerootafterbuild
% -reversible
% -showchromsearch
% -showiterative
% -slave
% -solospawn
% -somechroms
% -spewbinary
% -statematrix
% -statenumber
% -statetax
% -stats
% -stopat
% -submodel
% -theta
% -time
% -topopickrandom
% -topopwpickrandom
% -toposkipidentical
% -notoposkipidentical
% -totallikelihood
% -translocationcost
% -translocationsizecost
% -transpositioncost
% -transpositionsizecost
% -transrecombination
% -treeweightscale
% -trullytotallikelihood
% -unpack_binary
% -upincremental
% -verbose
% -noiafexcludeexcluded
%%

\end{document}
