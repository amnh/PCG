\documentclass{article}
\usepackage{amsmath}
\usepackage{algorithm}
\usepackage{algpseudocode}

\begin{document}
%%------------------------------------------------------------------------%%
%%% See resolve_all_trees.py for legible, working subset generation code %%%
%%------------------------------------------------------------------------%%
\newcommand{\comment}[1]{\hskip1em// #1}
\alglanguage{pseudocode}

\begin{algorithm}
\caption{Enumerate all trees for given taxon set}\label{create_trees}
\begin{algorithmic}[1]

\Function{resolve}{${root}$}
\Require phylogenetic node, ${root}$, the root of an unresolved phylogenetic tree
\Ensure phylogenetic forest: all possible resolutions of tree rooted at ${root}$
%\comment {Input: phylogenetic node; Output: phylogenetic forest. }
    \State ${forest} \gets \emptyset$
    \If {$root$ is leaf}
        \State ${forest} \gets {list}(root)$

        \comment{$root$ is already the root of all possible subtrees}
    \Else % For each pair of subsets generate a pair of forests, return the new forest
        \State \textit{subsetPairs} $\gets$ \Call{generate\_subsets}{$root_{descendents}$}
        \ForAll{$(l, r) \in {subsetPairs}$}
            \State ${subtree_l} \gets$ \Call{resolve}{l}
            \State ${subtree_r} \gets$ \Call{resolve}{r}
            \State ${forest} \gets$ \Call{create\_node}{$subtree_l$, $subtree_r$}
        \EndFor
    \EndIf
    \State \Return ${forest}$
\EndFunction
\end{algorithmic}
\end{algorithm}


\begin{algorithm}
\caption{Enumerate subset bifurcations of ordered input set}\label{generate_subsets}
\begin{algorithmic}[1]
\Function{generate\_subsets}{${forest, length}$}\label{subsets}
    \Require Phylogenetic forest (essentially an \textit{ordered} set of nodes) of size $>$ 2, as well as size of said forest
    \Ensure  List of tuple pairs of all bifurcations of forest set


    \If {${length} < 3$}            \Comment{This is degenerate case. Returning this just to match types.}
        \State \textit{subsetPairs} $\gets (\emptyset, n)$

    \ElsIf {${length} == 3$}        \comment{This is the base case.}
        \State \textit{subsetPairs} $\gets$

               \quad \quad \quad \quad \quad  [ ([forest[0]], forest[1:]),

               \quad \quad \quad \quad \quad    ([forest[1]], [forest[0], forest[2]]),

               \quad \quad \quad \quad \quad    ([forest[2]], forest[:2])  ]
               \comment{Split notation above follows Python.}
    \Else                           \comment{Call recursively from here.}
        \State ${subsetPairs} \gets \emptyset$
        \For {$i \gets 0, n$}
            \comment{For each single member, $m$, create tuple $(m, {forest} - m)$.}
            \State ${subsetPairs} \gets {subsetPairs}$ \textbf{cons}
            \If {${length} - i - 1 > 2$ and $i < {length} / 2$}
                \For {$(lhs, rhs) \in$ \Call{generate\_subsets}{${forest_descendents[]}$last ${length - i}$ descendents of forest, length - i - 1}}


                \EndFor
            \EndIf
        \EndFor
    \EndIf

    \Return \textit{subsetPairs}


% \Return $trees_{final}$
\EndFunction

\end{algorithmic}
\end{algorithm}
\end{document}
