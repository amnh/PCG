\documentclass[11pt]{article}
\usepackage{setspace}
\usepackage{graphicx}
\usepackage{subfigure}
\usepackage{lscape}
\usepackage{flafter}  % Don't place floats before their definition
\usepackage{bm}  % Define \bm{} to use bold math fonts
\usepackage{amsmath}
\usepackage{amsfonts}
\usepackage{amssymb}
\usepackage{MnSymbol}
\usepackage{url}
\usepackage{natbib}
%\usepackage{fullpage}
\bibliographystyle{cbe}
\citestyle{aa}
%\usepackage{algorithmic}
%\usepackage[vlined,algochapter,ruled]{algorithm2e}
\usepackage[vlined,ruled]{algorithm2e}
\SetKwComment{Comment}{$\triangleright\ $}{}

%ROOT PRIORS?
%Add Giribet and Wheeler arth, ratchet cite, Arango sea spiders

\title{Algorithmic Descriptions and Pseudo-Code for Rerooting Logic in a Phylogenetic Network.}
\author{  Alex Washburn
       \\ Division of Invertebrate Zoology,
       \\ American Museum of Natural History,
       \\ Central Park West @ 79th Street,
       \\ New York, NY 10024-5192,
       \\ USA,
       \\ alex.washburn@recursion.ninja
       }
	%\date{}
\begin{document}

\maketitle
\begin{abstract}
    Description of memoized implementation of rerooting logic on an unrooted binary tree and extension to memoized rerooting on a phylogentic network. The network extension requires resolving potential cycles.
\end{abstract}
%\newpage
%\tableofcontents
%\newpage

%\doublespacing
\section{Introduction}
Rerooting is the method of considering every edge in a rooted tree to be the potential root, determining the cost and character states for the tree if that edge was selected as the root. A precondition is that a postorder scoring of the tree has already occurred with an arbitrary edge in the unrooted tree selected as the root.

\section{Memoized rerooting of an unrooted binary tree}
 
\section{Unary set of directed memoization choices on unrooted binary tree}
 
\section{Trinary set of directed memoization choices on unrooted phylogenetic network}

\section{Pruning cyclic subtree resolutions on unrooted phylogenetic network}

\end{document}
